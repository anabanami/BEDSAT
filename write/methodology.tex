\chapter{Methodology}

The first phase of the project (objective 1) is to derive the BedSAT method. This will involve the integration of the Budd~\cite{Budd_1970} mathematical model relating ice surface elevation and bed topography into ISSM, and the development of a methodology for the data assimilation into ISSM. I will use a regional catchment in Antarctica for which relatively more radar data are available and has an indicative range of topography features, e.g. the Aurora Subglacial Basin, East Antarctica, whose margins have been extensively surveyed by the ICECAP project for airborne geophysics~\cite{Young_2011}. The second phase of the project (objective 2) will apply the methodology developed in objective 1 to the whole Antarctic continent, deriving a continent-wide bed topography dataset. Using covariance properties from existing radar surveys, I will generate a number of realisations of bed topography with unique high-resolution, and statistically-consistent topographic roughness. The third phase of the project will use the new bed topography datasets to conduct a sensitivity analysis of ice sheet model projections to 2300 CE, investigating the impact of the new topography and different realisations of roughness on ice and subglacial hydrological flow and ice mass loss from Antarctica.\\


\section*{Currently available data and Framework}\label{data}
The project will make use of a number of new remote sensing datasets, namely the Reference Elevation Model of Antarctica (REMA), ice surface velocities from NASA’s ITS\_LIVE, and the state-of-the-art Ice-sheet and Sea-level System Model (ISSM).

\begin{enumerate}
    \item\textbf{Reference Elevation Model of Antarctica (REMA)}\\
    REMA provides a high-resolution (2-metre) terrain map of nearly the entire continent, allowing for precise measurements of elevation changes over time. REMA supports various remote sensing activities, such as image orthorectification and interferometry, and aids in geodynamic and ice flow modeling, grounding line mapping, and surface process studies. Constructed from hundreds of thousands of Digital Elevation Models (DEMs) derived from high-resolution Maxar satellite imagery (including WorldView and GeoEye data), REMA is calibrated with Cryosat-2 and ICESat altimetry, ensuring high elevation accuracy with uncertainties of less than 1 meter over most areas\cite{REMA}.

    \item\textbf{ITS\_LIVE Antarctic surface velocities and elevation}\\
    The NASA-administered ITS\_LIVE website provides automated, high-resolution datasets of Antarctic surface velocities and ice surface elevation change, derived from satellite observations. The datasets are available on annual timesteps from 1985 to present. ITS\_LIVE employs various statistical and computational methods to process data from satellites including Landsat and Sentinel, ensuring precise and timely updates for scientific research~\cite{itslive}.

    \item\textbf{BedMachine}\\
    \cite{Fremand_2023}.

    \item\textbf{Ice-sheet and Sea-level System Model (ISSM)}~\cite{ISSM}\\
    ISSM is a finite-element numerical ice sheet model. It has been used to simulate the Antarctic Ice Sheet’s response to various climate scenarios and assess future mass loss contributions to sea level rise [9, 10]. The mesh can be refined to better capture variations in ice flow and driving stresses, enhancing the simulation’s accuracy of surface elevation changes and ice dynamics. This project will involve numerical modeling using advanced mathematical approaches, including the Blatter-Pattyn approximation to the full Stokes equations for ice flow (i.e. conservation of momentum equations). The Blatter-Pattyn approximation strikes a balance between the computationally intensive full Stokes equations and the simpler shallow ice approximation (SIA), retaining vertical shearing and longitudinal stress gradients. This makes it ideal for modeling the dynamics of fast-flowing ice streams and ice shelves at the continental scale, enhancing simulation accuracy while being computationally feasible. Additionally, data assimilation, machine learning, and geostatistics will be employed, with the full Stokes equations used if necessary.
\end{enumerate}


