\chapter*{Methodology}

The project will make use of a number of new remote sensing datasets, namely the Reference Elevation Model of Antarctica (REMA), ice surface velocities from NASA’s ITS\_LIVE, and the state-of-the-art Ice-sheet and Sea-level System Model (ISSM).

The first phase of the project (objective 1) is to derive the BedSAT method. This will involve the integration of the Budd~\cite{Budd_1970} mathematical model relating ice surface elevation and bed topography into ISSM, and the development of a methodology for the data assimilation into ISSM. I will use a regional catchment in Antarctica for which relatively more radar data are available, e.g. the Aurora Subglacial Basin, East Antarctica, whose margins have been extensively surveyed by the ICECAP project for airborne geophysics~\cite{Young2011}. The second phase of the project (objective 2) will apply the methodology developed in objective 1 to the whole Antarctic continent, deriving a continent-wide bed topography dataset. Using covariance properties from existing radar surveys, I will generate a number of realisations of bed topography with unique high-resolution, and statistically-consistent topographic roughness. The third phase of the project will use the new bed topography datasets to conduct a sensitivity analysis of ice sheet model projections to 2300 CE, investigating the impact of the new topography and different realisations of roughness on ice and subglacial hydrological flow and ice mass loss from Antarctica.\\