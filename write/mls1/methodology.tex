\chapter{Methods}
\section{Aims}
 My research plan is structured around these three broad research questions:
\begin{enumerate}

    \item How does the bed topography manifest on the ice surface?

    \item To what extent do interpolation uncertainties in bed topography datasets affect the accuracy of Antarctic Ice Sheet evolution simulations under different climate change scenarios?

    \item What is the impact of variable bed conditions and topography on the rate of grounding line (GL) retreat in continental ice sheets?

\end{enumerate}

Underpinning these research questions are the following objectives (O):
\begin{itemize}
    \item{O1:} Develop an ice sheet modelling approach to assimilate satellite remote sensing datasets to improve knowledge of the bed (BedSAT) informed by mathematical models of ice flow over topography;

    \item{O2:} Derive a new bed topography for Antarctica using BedSAT;

    \item{O3:} Evaluate the impact of the improved bed topography on projections of ice mass loss from Antarctica under climate warming through sensitivity analyses. 
\end{itemize}

\section{Research plan methodology}

In order to achieve these objectives, each will be addressed in sequential phases. My primary focus is currently on O1: Deriving the BedSAT method. As the initial phase of O1, I am working on an investigation on the influence of different combinations of rheological and sliding law assumptions in ice sheet modeling. The goal of this investigation is to systematically understand the forward problem (how the bed affects the surface under different physical rules), and then use that knowledge to build a better inverse model (BedSAT). This foundational study will be the basis for my first peer-reviewed paper.

\subsection{Foundational Analysis of Bed-to-Surface Signal Transfer}\label{paper1}

The first critical step is to systematically quantify how fundamental physical assumptions influence the expression of subglacial topography at the ice surface.
This directly addresses my first research question: "How does the bed topography manifest on the ice surface?". This work will leverage the Ice-sheet and Sea-level System Model (ISSM) with a custom-built computational framework based on a synthetic bed topography database. See Chapter~\ref{progress} for detailed information. This systematic study will verify and validate the necessary set of constraints on bed-to-surface transfer functions that account for realistic ice dynamics. This comprehensive analysis will form the basis of the the first peer-reviewed manuscript of this PhD.

\subsection{Development of the BedSAT Inversion Framework}

By understanding how rheology and sliding conditions alter the surface expression of the bed, I can develop more physically robust transfer functions for the inversion process. The inversion model will be developed and tested using a regional catchment in Antarctica with extensive radar data, such as the Aurora Subglacial Basin (this data can be found in works such as~\cite{Young_2011}). The model will be constrained by available observations of surface velocity, thermal distribution, and ice thickness, this will allow for direct validation of the inversion results against known bed configurations. Furthermore, the robustness of the model will be ensured through grid independence testing and a sensitivity analysis of model assumptions.

\subsection{Derive a new bed topography for Antarctica using BedSAT}

I will apply the validated BedSAT methodology from O1 to the entire Antarctic continent, deriving a new continent-wide bed topography dataset. Using covariance properties from existing radar surveys, I will generate multiple realisations of the bed, each with unique and statistically-consistent topographic roughness.

\subsection{Evaluate the impact of the improved bed topography}

The new bed topography datasets will be used to conduct a sensitivity analysis of ice sheet model projections to 2300 CE. This will investigate the impact of the improved topography and different roughness realisations on ice dynamics, subglacial hydrology, and overall ice mass loss from Antarctica, directly addressing the project's main research questions.
\\
\textit{Note: Detailed methodological outlines for O2 and O3 will be developed following the completion and refinement of the BedSAT method in O1.}


