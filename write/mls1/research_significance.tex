\chapter{Research Significance}
The polar regions are losing ice, and their oceans are changing rapidly~\cite{O_C_in_changingClimate}. The consequences of this extend to the whole planet and it is crucial for us to understand them to be able to evaluate the timing and magnitude of potential mitigation strategies. 
The accelerating loss of continental ice sheets in both Greenland and Antarctica are major contributors to global sea level rise~\cite{O_C_in_changingClimate}. Impacts extend beyond direct ice loss: as fresh water from melting ice sheets is added into the ocean, it increases ocean stratification disrupting global thermohaline circulation~\cite{Jacobs_2004}. In addition, cold freshwater can dissolve larger amounts of $\mathrm{CO_2}$ than regular ocean water creating corrosive conditions for marine life~\cite{O_C_in_changingClimate}.
While there is high confidence in current ice loss and retreat observations in many areas, there is more uncertainty about the mechanisms driving these changes and their future progression~\cite{Fox-Kemper_2021}. Uncertainty increases in regions with variable bed conditions, where characteristics like bed slipperiness and roughness are difficult to verify via direct observations. Other problematic areas involve the ice sheet grounding line (GL): The zone that delineates where the ice sheet is grounded on bedrock. The GL retreat rate depends crucially on topographical features like pinning points~\cite{Fox-Kemper_2021}, which lead to increased retardation of upstream flow. Although this mechanism is established, major knowledge gaps persist in mapping bed topography across Antarctic ice sheet margins, with over half of all margin areas having insufficient data within $5\!\,$km of the grounding zone~\cite{RINGS_2022}. Addressing this data gap through both systematic mapping and improved interpolation —utilising auxiliary data streams with more complete coverage— would significantly improve both our understanding of current ice dynamics and the accuracy of ice-sheet models projecting future changes.

\section{Writing Contributions}
I have helped evaluating existing methodologies to address critical data gaps in Antarctic bed topography products. 

The review titled ``Synthetic bed topographies for Antarctica and their utility in ice sheet modelling'' establishes the theoretical context for a case study on the Aurora Subglacial Basin. This manuscript is framed as a comparison between different types of synthetic beds (``elevation-preserving'' vs. ``texture-preserving'') underscoring how methodological choices in bed generation can significantly influence projections of future sea level contributions. A key outcome of this investigation was the identification of persistent limitations in widely-used interpolation techniques. Many established methods struggle to provide robust uncertainty estimates, avoid systematic biases, or realistically capture the spatial correlation of errors.

In the manuscript ``Antarctic bed topography estimation using a Stochastic Meshless Uncertainty Gridding (SMUG) method'', we establish the scientific rationale for SMUG: an algorithm to generate robust estimates of the bed topography using Taylor series expansions with robust uncertainty estimates. The paper sets stage for introducing SMUG as a method designed to overcome specific challenges faced by commonly used interpolation methods. This research effort, moves from a comprehensive assessment of existing tools to the justification and development of a next-generation approach for Antarctic bed mapping.



