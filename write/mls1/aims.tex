\chapter{Aims}
Antarctica's bed topography data currently has local uncertainties of hundreds of metres in elevation due to sparse and unevenly distributed radar surveys, significantly limiting our ability to predict ice sheet behaviour and sea level rise contributions. 

Through BedSAT, I aim to develop a novel modelling approach that integrates remote sensing data and airborne-derived estimates, with mathematical and numerical ice flow models to substantially improve bed topography resolution and accuracy. One of my goals is to derive a continent-wide bed topography dataset. I plan to use this dataset to conduct sensitivity analyses of dynamic ice loss to different realisations of topographic roughness through 2300CE.

My work will quantify how bed topography uncertainties affect ice mass loss projections. The resulting open-source dataset and model will provide more reliable sea-level rise predictions, benefiting the broader scientific community and informing climate change mitigation strategies.

\section{Objectives}
\begin{itemize}
    \item{O1:} Develop an ice sheet modelling approach to assimilate satellite remote sensing datasets to improve knowledge of the bed informed by mathematical models of ice flow over topography;
    \item{O2:} Derive a new bed topography for Antarctica using BedSAT;
    \item{O3:} Evaluate the impact of the improved bed topography on projections of ice mass loss from Antarctica under climate warming through sensitivity analyses. 
\end{itemize}