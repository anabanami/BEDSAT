\chapter{Aims}
Antarctica's bed topography currently has local uncertainties of hundreds of metres in elevation due to sparse and unevenly distributed radar surveys. BedSAT will integrate remote sensing data and airborne-derived estimates, with ice flow models can improve bed topography resolution and accuracy. Quantifying how bed topography uncertainties affect ice mass loss projections via sensitivity analyses with different realisations of topographic roughness through 2300CE. Providing more reliable sea level rise predictions. Following open-source approach and FAIR data principles, these improvements benefit the broader scientific community and support more effective climate change mitigation planning.

\section{Research questions}
\begin{enumerate}
    \item How does the bed topography manifest on the ice surface?
    \item To what extent do interpolation uncertainties in bed topography datasets affect the accuracy of Antarctic Ice Sheet evolution simulations under different climate change scenarios?
    \item What is the impact of variable bed conditions and topography on the rate of grounding line (GL) retreat in continental ice sheets?
\end{enumerate}
\section{Objectives}
\begin{itemize}
    \item{O1:} Develop an ice sheet modelling approach to assimilate satellite remote sensing datasets to improve knowledge of the bed informed by mathematical models of ice flow over topography;
    \item{O2:} Derive a new bed topography for Antarctica using BedSAT;
    \item{O3:} Evaluate the impact of the improved bed topography on projections of ice mass loss from Antarctica under climate warming through sensitivity analyses. 
\end{itemize}