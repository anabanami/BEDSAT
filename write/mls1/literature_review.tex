\chapter{Preliminary Literature review}\label{review}
Bed topography is one of the most crucial boundary conditions that influences ice flow and loss from the Antarctic Ice Sheet (AIS)~\cite{Morlighem_2020}. Bed topography datasets are typically generated from airborne radar surveys, which are sparse and unevenly distributed across the Antarctic continent. Interpolation schemes—such as those used in Bedmap ice bed, surface and thickness gridded datasets (1,2,3) for Antarctica~\cite{Lythe_2001, Fretwell_2013, Pritchard_2025}, or BedMachine~\cite{Morlighem_2017}—to ``gap fill'' these sparse datasets yield bed topography estimates that have uncertanties of multiple hundreds of metres in elevation~\cite{Morlighem_2020} which propagate through simulations of AIS evolution under climate change~\cite{Castleman_2022}. Given the logistical challenges of accessing large parts of the Antarctic continent, there is a crucial need for alternative approaches that integrate diverse and possibly more spatially complete data streams – including satellite data.

\section{Approaches to Bed Topography Reconstruction}
To address the critical gap in understanding subglacial conditions, my research goal is to develop a method that combines forward and inverse modeling, leveraging high-resolution satellite surface data that significantly improve bed topography estimates in regions where direct radar measurements are sparse.
Bed topography research is based on two primary modeling approaches to infer subglacial bed topography and its influence on ice dynamics: forward and inverse models. Forward models aim to understand how different bed properties affect ice flow by running simulations on large ensembles of statistically generated bed topographies. These methods, often using geostatistical techniques like kriging~\cite{Mackie_2020} or flexural modelling~\cite{Jamieson_2023} and are valuable for investigating how uncertainties in the bed can impact outcomes like simulated ice mass loss. However, their fundamental limitation is that they rely on assumptions about the bed conditions rather than direct observation, exploring possibilities rather than determining the actual topography. While inverse models work by inferring bed properties, such as topography or basal slipperiness, from known surface observations like ice elevation and velocity. Techniques like control method inversion~\cite{deRydt_2013} and 4dvar assimilation minimise mismatches between observed and simulated data, on the other hand mass conservation approaches use physical laws to reconstruct the bed, especially where measurements are sparse~\cite{Morlighem_2017, Morlighem_2020}. Despite their power, these methods face significant challenges. Variational approaches often require regularisation to prevent non-physical results and over-fitting~\cite{Morlighem_Goldberg_2024}, and time-dependent methods like 4dvar are computationally expensive. Probabilistic methods such as Markov Chain Monte Carlo (MCMC) are powerful for quantifying uncertainty but remain too computationally intensive~\cite{Morlighem_Goldberg_2024} for large-scale models. The core problem for all these methods is the difficulty of fully utilising abundant satellite surface data to overcome the primary challenge in glaciology: a limited understanding of subglacial conditions due to sparse direct measurements.

\section{From Signal Transfer to Tractable Inversion}\label{theoretical_frameworks}
A robust theoretical framework is essential for connecting the observable surface features to the hidden subglacial bed topography underneath. The core principle is that the ice sheet acts as a physical filter, modifying the expression of bed features as they propagate to the surface. The work by Budd (1970)~\cite{Budd_1970} established that ice preferentially dampens short-wavelength bed undulations while features with a wavelength approximately $3.3$ times the ice thickness are most clearly expressed at the surface. This process also introduces a phase lag, and can be described mathematically using frequency-dependent ``transfer functions''. More recent studies have built directly on this foundation. For instance, Gudmundsson and Raymond (2008) ~\cite{Gudmundsson_2008} refined the transfer function concept for ice streams, and Ockenden et al. (2023)~\cite{Ockenden_2023} applied it in reverse, using full-Stokes transfer functions to invert high-resolution satellite observations of surface elevation and velocity to infer bed properties. These studies collectively establish that the physical relationship between the bed and the surface provides a viable pathway for subglacial mapping.

Despite the success of these approaches, their practical application has been limited by simplifying assumptions about ice physics. A primary limitation is the reliance on a linear (Newtonian) ice rheology, where stress is directly proportional to the strain rate (i.e., a ``constant viscosity''). This assumption is often utilised to make the inversion mathematically tractable, but it contrasts with the widely accepted non-linear Glen's Flow Law, where the stress exponent is typically 
$n \approx 3$ or even $n = 4$. As my preliminary modeling work shows in section~\cite{study1}, the choice of rheology is a critical control on the bed-to-surface signal transfer; a non-linear rheology ($n = 4$) produces significantly different surface expressions compared to a linear one ($n = 1$). By largely ignoring non-linear rheology and complex basal sliding conditions, past inversion methods have introduced uncertainties and may not be robustly applicable across all dynamic regimes of the ice sheet. This leaves a critical gap: the need for an inversion method that honours more realistic ice dynamics.

The critical opportunity lies in the exploiting the vast wealth of underutilised high-resolution satellite surface observations including NASA's ITS\_LIVE~\cite{itslive} velocities and REMA elevations~\cite{REMA}. My approach with BedSAT will harness these data streams by building upon established transfer function theory while addressing the fundamental limitation that has restricted past approaches: the mathematical intractability of non-linear ice physics. BedSAT will connect surface observations with bed topography using more realistic rheological assumptions ($n = 4$) and complex sliding conditions, rather than the simplified linear physics commonly used by traditional inversions. The key step to making this process tractable is Physics-Informed Machine Learning (Physics-ML), which solves the computational bottleneck that has forced previous methods to rely on unrealistic simplifications. By leveraging NVIDIA PhysicsNeMo—designed to blend governing physics (PDEs) with training data\cite{NVIDIA_NeMo_2025}—BedSAT can learn the non-linear mapping between surface expressions and bed topography without linearising the physics. This approach transforms what was previously computationally intractable into a fast, accurate inverse solver. My systematic forward modeling study (Section~\ref{study1}) directly informs this by establishing how different rheological and sliding assumptions alter bed-to-surface transfer functions, providing the physical constraints needed to train the Physics-ML model.
Through an iterative process where initially inverted bed topography is integrated into forward models for progressive refinement, BedSAT will deliver physically consistent reconstructions validated against independent datasets. This represents a fundamental advance: where previous methods had to choose between physical realism and computational feasibility, Physics-ML enables both simultaneously.