\chapter{Antarctica's Landscape}\label{why}
\section{Climate Impacts and Global Significance}

The polar regions are losing ice, and their oceans are changing rapidly~\cite{O_C_in_changingClimate}. The consequences of this extend to the whole planet and it is crucial for us to understand them to be able to evaluate the costs and benefits of potential mitigation. 

Changes in different kinds of polar ice affect many connected systems. Of particular concern is the accelerating loss of continental ice sheets (glacial ice masses on land) in both Greenland and Antarctica, which has become a major contributor to global sea level rise~\cite{O_C_in_changingClimate}. Impacts extend beyond direct ice loss: as fresh water from melting ice sheets is added into the ocean, it increases ocean stratification disrupting global thermohaline circulation~\cite{Jacobs_2004}. In addition, cold freshwater can dissolve larger amounts of $\mathrm{CO_2}$ than regular ocean water creating corrosive conditions for marine life~\cite{O_C_in_changingClimate}.
 
While there is high confidence in current ice loss and retreat observations in many areas, there is more uncertainty about the mechanisms driving these changes and their future progression~\cite{Fox-Kemper_2021}. Uncertainty increases in regions with variable bed conditions, where characteristics like ``slipperiness'' and ``roughness'' are difficult to verify via direct observations. Other problematic areas involve the ice sheet's grounding line (GL), the zone that delineates ice grounded on bedrock from ice shelves floating over the ocean. The retreat rate depends crucially on topographical features like pinning points~\cite{Fox-Kemper_2021}, which lead to increased buttressing by the ice shelf on the upstream ice sheet. Although this mechanism is established, major knowledge gaps persist in mapping bed topography across Antarctic ice sheet margins - with over half of all margin areas having insufficient data within 5 km of the grounding zone~\cite{RINGS_2022}. Addressing this data gap through both systematic mapping and improved interpolation utilising auxiliary data streams with more complete coverage would significantly improve both our understanding of current ice dynamics and the accuracy of ice-sheet models projecting future changes.

\chapter{Topography of Antarctica}\label{review}

Bed topography is one of the most crucial boundary conditions that influences ice flow and loss from the Antarctic Ice Sheet (AIS)~\cite{Morlighem_2020}. Bed topography datasets are typically generated from airborne radar surveys, which are sparse and unevenly distributed across the Antarctic continent (see figure \ref{fig:BedMAP}). Interpolation schemes to ``gap fill'' these sparse datasets yield bed topography estimates that have high uncertainties (i.e. multiple hundreds of metres in elevation uncertainty; Morlighem et al., 2020) which propagate through simulations of AIS evolution under climate change~\cite{Castleman_2022}. Given the logistical challenges of accessing large parts of the Antarctic continent, there is a crucial need for alternative approaches that integrate diverse and possibly more spatially complete data streams – including satellite data.
\begin{figure}[H] % Forces the figure exactly HERE
    \includegraphics[scale=0.38]{bedmap.png}
    \caption{Distribution of BedMAP\{1,2,3\} data tracks (Source: bedmap.scar.org).}
    \label{fig:BedMAP}
\end{figure}

\section{Approaches to Bed Topography Reconstruction}

A key objective of this study is to understand the bed topography itself and how it influences ice dynamics. There are two ways to can infer information about this relationship: Through forward modelling, with assumptions of the bed conditions; and through inverse modelling that relies on surface observations.
\begin{itemize}
    \item\textbf{Forward models}\\
    The aim of forward models is to see how bed properties impact ice dynamics. A key example is using a large ensemble of bed topographies to investigate how bed uncertainties impact simulated ice mass loss. In this example geostatistical methods can be used to generate bed topographies that either preserve elevation or texture:    
    \begin{itemize}
            \item\textbf{Geostatistics} Statistical methods specialized for analyzing spatially correlated data. In glaciology, this approach is used to interpolate between sparse measurements and characterise spatial patterns in bed properties, often employing techniques like kriging~\cite{Mackie_2020}.

    \end{itemize}

    \item\textbf{Inversion models}\\
    The aim of these models is to understand bed properties through knowledge of surface or other variables. A key example is the retrieval of bed topography or basal slipperiness from surface elevation and velocities.

        \begin{itemize}
            \item\textbf{Control method inversion}: A variational approach that minimizes mismatches between observed and simulated fields through a cost function approach. Remote sensing data and theoretical ice flow models are used to obtain basal conditions~\cite{deRydt_2013}. Often needs regularization terms to prevent non-physical features or over-fitting~\cite{Morlighem_Goldberg_2024}.

            \item\textbf{4dvar}: Four-dimensional variational data assimilation - Similar to the control method inversion algorithm, but adds a time dimension. Used to optimize model parameters and initial conditions~\cite{Morlighem_Goldberg_2024}. Can handle time-varying data and evolving glacier states, making it more suitable for dynamic systems unlike control methods, this makes them more computationally demanding~\cite{Morlighem_Goldberg_2024}.

            \item\textbf{Mass conservation}: Used to constrain inversion models and fill data gaps by employing physical conservation laws, particularly effective for reconstructing bed topography where direct measurements are sparse~~\cite{Morlighem_2017, Morlighem_2020}. Requires (contemporary) measurements of ice thickness at the inflow boundary to properly constrain the system~\cite{Morlighem_Goldberg_2024}.

            \item\textbf{Markov Chain Monte Carlo (MCMC)}: A probabilistic method that generates sample distributions to quantify uncertainties in ice sheet parameters and models~\cite{Morlighem_Goldberg_2024}. While powerful for uncertainty quantification, these methods remain computationally intensive for continental-scale ice sheet models~\cite{Morlighem_Goldberg_2024}.

            \item\textbf{EnKF} Ensemble Kalman Filter. A sequential data assimilation method that uses an ensemble of model states to estimate uncertainty and update model parameters based on observations~\cite{Morlighem_Goldberg_2024}.
        \end{itemize}
    
\end{itemize} 
This study aims to develop an integrated method combining forward and inverse modeling to improve bed topography estimates by leveraging high-resolution satellite surface data in regions where radar data is sparse. Despite revolutionary advances in satellite technology that provide unprecedented surface detail, a key challenge in glaciology remains: how to fully utilize this wealth of information in regions where our understanding of subglacial conditions is limited. Our approach will integrate more comprehensive dynamical ice models with modern computational capabilities to develop better bed topography.

\newpage
\section{Theoretical Frameworks}
 Understanding how bed features manifest in surface observations requires a theoretical framework that connects these two domains. The modelling approach used in this project relies on two different theoretical frameworks that relate bed topography and surface features. Using synthetic data, observations and these modelling frameworks, my goal is understanding the limitations of each approach and how they can be improved.

\subsection{Ice Flow Over Bedrock Perturbations - Budd 1970}
The first framework was originally developed by Budd~\cite{Budd_1970}. This model relates ice flow over bedrock perturbations to surface expressions using a two-dimensional biharmonic stress equation. The modelling carried out in~\cite{Budd_1970} determined ice-sliding velocities for wide ranges of roughness, normal stress, and shear stress relevant to real glaciers~\cite{Budd_1970}. Budd's framework fits in my project as a means of verifying the validity of the physics in my ice-sheet model, since it describes important effects of bedrock disturbances on the transient evolution of the transferred basal disturbances onto the surface. The theory makes several key predictions that have been confirmed through spectral analysis of real ice cap profiles: 
\begin{enumerate}
    \item A wavelength of minimum damping occurs at approximately 3.3 times the ice thickness, 
    \item surface undulations exhibit a $\pi/2$ phase lag relative to bedrock features with steepest surface slopes occurring over the highest bedrock points, and 
    \item the amplitude reduction depends systematically on ice speed, viscosity, thickness, and wavelength. 
\end{enumerate}

Importantly, Budd's theory demonstrates that energy dissipation and basal stress patterns are maximized for bedrock irregularities with wavelengths several times the ice thickness, while smaller-scale bedrock variations decay exponentially with distance into the ice and have minimal impact on overall ice motion. This selective filtering of bedrock signals provides crucial insights for understanding which scales of bed topography most significantly influence ice dynamics.

A critical aspect of Budd's theoretical framework is understanding how ice rheology affects the bed-to-surface transfer relationships. Glen's flow law typically employs a stress exponent $n\approx 3$ for ice under most natural conditions, reflecting the strongly nonlinear relationship between stress and strain rate. Budd's analysis revealed that under certain low-stress conditions, ice deformation can behave more linearly $n\approx 1$ than conventional wisdom suggests. This rheological distinction has profound implications for bed-to-surface transfer functions: linear rheology $(n = 1)$ produces different amplitude dampening and phase relationships compared to nonlinear rheology $(n = 3)$, particularly for wavelengths around the critical 3.3 times ice thickness scale. My current experimental framework systematically explores this in 1D by generating forward models for multiple synthetic bedrock profiles across four scenarios combining rheological assumptions $(n = 1$ vs $n = 3)$ with basal boundary conditions (no-slip vs sliding), enabling direct comparison of how these physical assumptions affect the detectability and reconstruction of bed features from surface observations. Understanding these differences is essential for developing robust inversion methods, as the choice of rheological model fundamentally determines the mathematical relationship between observable surface expressions and the underlying bed topography I seek to reconstruct.

Crucially, Budd's work established the concept of frequency-dependent transfer functions that act as "filters" between bed and surface topography. This transfer function approach, expressed as $\psi(\omega) = \frac{\text{surface amplitude}}{\text{bed amplitude}}$ for wavelength $\lambda = 2\pi/\omega$, provides a direct mathematical framework for inversion. By inverting these transfer functions, one can theoretically reconstruct bed topography from surface observations, particularly for wavelengths where the damping factor is minimal and the signal-to-noise ratio is optimal.

\subsection{Ice flow perturbation analysis - Ockenden 2023}

The second framework in my analysis builds upon these foundational concepts through some of the recent work by Ockenden et al. in~\cite{Ockenden_2023}, which uses observed surface perturbations (in velocity and elevation) to invert for unknown basal perturbations. Ockenden et al. improve from their previous work in~\cite{Ockenden_2022} by using full-Stokes transfer functions, which greatly improves their method when dealing with steep topography where the shallow-ice-stream approximation breaks down. Ockenden et al. find this is crucial for better resolving the topographic features they are interested in. The core principle of the method by Ockenden et al. (2023) relies on the fact that variations in basal topography, slipperiness, and roughness cause measurable disturbances to the surface flow of the ice. Through linear perturbation analysis, they establish a systematic relationship between surface observations and bed conditions. This relationship can be expressed as $y=f(x)$, where $y$ represents surface measurements (velocity and topography), $x$ represents bed properties (topography and slipperiness), and $f$ is the forward model transfer function refined by Gudmundsson and Raymond in 2008~\cite{Gudmundsson_2008} this by deriving analytical transfer functions. 

In their work Ockenden et al apply this framework in reverse $x=f^{-1}(y)$, to infer the bedrock from modern, high-resolution satellite data estimates. A restrictive assumption in the modeling design by Ockenden et al might be their assumption of ``constant viscosity", in glaciology this is equivalent to assuming a linear rheology, where the stress exponent, $n$, is equal to 1. This means that the strain rate is directly proportional to the stress. This is in contrast to the more commonly used non-linear Glen's Flow Law, where $n$ is typically around 3 or even 4. This earliest phase in my PhD project has as a goal to determine whether treating the rheology of ice as  linear is adequate. Ockenden et al account for a non-linear sliding law at the base of the ice, mentioning the ``sliding law parameter m''. However, the transfer of stress through the body of the ice—the core of the perturbation analysis—relies on the linear rheology (constant viscosity) assumption from the foundational work of Gudmundsson and Raymond 2008.

% How will you use the insights from Budd's original analysis (e.g., on non-linear rheology) to inform your new inversion method?

\subsection{Bridging Classical and Modern Approaches}

While both frameworks address the fundamental bed-to-surface relationship, they operate at different levels of complexity and make different assumptions. Budd's approach provides the fundamental physical understanding of how specific wavelengths propagate through ice, establishing theoretical limits on what bed features can be detected from surface observations. Ockenden's method extends this to practical applications using real satellite data but relies on linearised assumptions that may break down under certain conditions. My research aims to bridge these approaches by combining Budd's rigorous transfer function analysis with comprehensive forward modeling that relaxes some of the restrictive assumptions inherent in linear perturbation methods. By systematically exploring how different rheological models $(n = 1$ vs $n = 3)$ and basal conditions affect the bed-to-surface transfer functions, this work aims to develop a robust inversion method that can better handle the nonlinear physics of ice flow while maintaining the theoretical rigor established by Budd's foundational analysis.

\section{Current Opportunities}

Current Antarctic bed topography reconstruction methods fail to utilize the wealth of presently available satellite-derived surface data. While mathematical models linking bed to surface through ice dynamics (such as those by Ockenden and Budd) provide a foundation for inferring bed topography from satellite data, they have significant limitations. My approach with BedSAT builds upon theoretical foundations and recent inversion methods to better understand how bed conditions—including slipperiness, roughness, and pinning points—affect both grounding line retreat rates and their surface expressions. BedSAT will connect surface observations with bed topography using more realistic rheological and geometric assumptions through an iterative process: initially inverted bed topography will feed into ice dynamics models with these improved assumptions, allowing comparison between model predictions and established datasets like NASA's ITS\_LIVE. I expect to utilise Machine learning methods to systematize this process, enhancing the analytical capabilities for the project's final phase.

% especially in the context of sparse data availability in space and time. 