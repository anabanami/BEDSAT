\chapter{Research problem}\label{review}
Bed topography is one of the most crucial boundary conditions that influences ice flow and loss from the Antarctic Ice Sheet (AIS)~\cite{Morlighem_2020}. Bed topography datasets are typically generated from airborne radar surveys, which are sparse and unevenly distributed across the Antarctic continent (see figure \ref{fig:BedMAP}). Interpolation schemes to ``gap fill'' these sparse datasets yield bed topography estimates that have high uncertainties (i.e. multiple hundreds of metres in elevation uncertainty; Morlighem et al. 2020) which propagate through simulations of AIS evolution under climate change~\cite{Castleman_2022}. Given the logistical challenges of accessing large parts of the Antarctic continent, there is a crucial need for alternative approaches that integrate diverse and possibly more spatially complete data streams – including satellite data.
\begin{figure}[H]
    \includegraphics[scale=0.31]{bedmap.png}
    \caption{Distribution of BedMAP\{1,2,3\} data tracks (Source: bedmap.scar.org).}
    \label{fig:BedMAP}
\end{figure}
\newpage
\section{Approaches to Bed Topography Reconstruction}
An objective of my research is to understand the bed topography itself and how it influences ice dynamics. There are two ways to infer information about this relationship: Through forward modelling, with assumptions of the bed conditions; and through inverse modelling that relies on surface observations.
\begin{itemize}
    \item\textbf{Forward models}\\
    The aim of forward models is to see how bed properties impact ice dynamics. A key example is using a large ensemble of bed topographies to investigate how bed uncertainties impact simulated ice mass loss. The generated bed topographies preserve elevation or texture:    
    \begin{itemize}
            \item\textbf{Geostatistics} statistical methods specialized for analyzing spatially correlated data. This approach is used to interpolate between sparse measurements and characterise spatial patterns in bed properties~\cite{Mackie_2020}.
    \end{itemize}
    \item\textbf{Inversion models}\\
    The aim of these models is to understand bed properties through knowledge of surface or other variables. A key example is the retrieval of bed topography or basal slipperiness from surface elevation and velocities.
        \begin{itemize}
            \item\textbf{Control method inversion}: A variational approach that minimizes mismatches between observed and simulated fields through a cost function approach. Remote sensing data and theoretical ice flow models are used to obtain basal conditions~\cite{deRydt_2013}. Often needs regularization terms to prevent non-physical features or over-fitting~\cite{Morlighem_Goldberg_2024}.
            \item\textbf{4dvar}: Four-dimensional variational data assimilation - Similar to the control method inversion algorithm, but adds a time dimension. Used to optimize model parameters and initial conditions~\cite{Morlighem_Goldberg_2024}. Can handle time-varying data and evolving glacier states, making it more suitable for dynamic systems unlike control methods. The trade-off for extra functionallity is increased computational cost~\cite{Morlighem_Goldberg_2024}.
            \item\textbf{Mass conservation}: Used to constrain inversion models and fill data gaps by employing physical conservation laws, particularly effective for reconstructing bed topography where direct measurements are sparse~\cite{Morlighem_2017, Morlighem_2020}. Requires (contemporary) measurements of ice thickness at the inflow boundary to properly constrain the system~\cite{Morlighem_Goldberg_2024}.
            \item\textbf{Markov Chain Monte Carlo (MCMC)}: A probabilistic method that generates sample distributions to quantify uncertainties in ice sheet parameters and models~\cite{Morlighem_Goldberg_2024}. While powerful for uncertainty quantification, these methods remain computationally intensive for continental-scale ice sheet models~\cite{Morlighem_Goldberg_2024}.
            \item\textbf{EnKF} Ensemble Kalman Filter. A sequential data assimilation method that uses an ensemble of model states to estimate uncertainty and update model parameters based on observations~\cite{Morlighem_Goldberg_2024}.
        \end{itemize}
\end{itemize} 
My research aims to develop an integrated method combining forward and inverse modeling to improve bed topography estimates by leveraging high-resolution satellite surface data in regions where radar data is sparse.

\newpage
\section{Theoretical Frameworks}\label{theoretical_frameworks}
Understanding how bed features manifest in surface observations requires a theoretical framework that connects these two domains.

\subsection{Ice Flow Over Bedrock Perturbations - Budd 1970}
The model by Budd~\cite{Budd_1970} relates ice flow over bedrock perturbations to surface expressions. Budd's theory makes several key predictions that have been confirmed through spectral analysis of real ice cap profiles: 
\begin{enumerate}
    \item A basal disturbance wavelength of minimum damping occurs at approximately 3.3 times the ice thickness, 
    \item Surface undulations exhibit a $\pi/2$ phase lag relative to bedrock features with steepest surface slopes occurring over the highest bedrock points, and 
    \item The amplitude reduction depends systematically on ice speed, viscosity, thickness, and wavelength. 
\end{enumerate}
This theory demonstrates that energy dissipation and basal stress patterns are maximized for bedrock irregularities with wavelengths several times the ice thickness, while smaller-scale bedrock variations decay exponentially with distance into the ice and have minimal impact on overall ice motion. This selective filtering of bedrock signals provides crucial insights for understanding which scales of bed topography most significantly influence ice dynamics.
A critical aspect of Budd's theoretical framework is understanding how ice rheology affects the bed-to-surface transfer relationships. Glen's flow law typically employs a stress exponent $n\approx 3$ for ice under most natural conditions, reflecting the strongly nonlinear relationship between stress and strain rate. However, more recent research suggests that $n = 4$ may better represent ice flow in some locations~\cite{Getraer_2025}.  Budd's analysis revealed that under certain low-stress conditions, ice deformation can behave more linearly ($n\approx 1$) than conventional wisdom suggests. 
This rheological distinction has profound implications for bed-to-surface transfer functions: because linear rheology $(n = 1)$ may produce different amplitude dampening and phase relationships compared to nonlinear rheology $(n = 4)$, particularly for wavelengths around the critical 3.3 times ice thickness scale.
My current modelling work systematically explores this by generating forward models for multiple synthetic bedrock profiles across four scenarios combining rheological assumptions $(n = 1$ vs $n = 4)$ with basal boundary conditions (no-slip vs sliding), enabling direct comparison of how these physical assumptions affect the detectability and reconstruction of bed features from surface observations. Understanding these differences is essential for developing robust inversion methods, as the choice of rheological model fundamentally determines the mathematical relationship between observable surface expressions and the underlying bed topography I seek to reconstruct.
Crucially, Budd's work established the concept of frequency-dependent transfer functions that act as "filters" between bed and surface topography. This transfer function approach, expressed as $\psi(\omega) = \frac{\text{surface amplitude}}{\text{bed amplitude}}$ for wavelength $\lambda = 2\pi/\omega$, provides a direct mathematical framework for inversion. By inverting these transfer functions, one can theoretically reconstruct bed topography from surface observations, particularly for wavelengths where the damping factor is minimal and the signal-to-noise ratio is optimal.

\subsection{Ice flow perturbation analysis - Ockenden 2023}
Ockenden et al. (2023) use observed surface perturbations (in velocity and elevation) to invert for unknown basal perturbations. Ockenden et al. improve from their previous work in~\cite{Ockenden_2022} by using full-Stokes transfer functions, which greatly improves their method when dealing with steep topography where the shallow-ice-stream approximation breaks down. They find this is crucial for better resolving the topographic features they are interested in. The core principle relies on the fact that variations in basal topography, slipperiness, and roughness cause measurable disturbances to the surface flow of the ice. Through linear perturbation analysis, they establish a systematic relationship between surface observations and bed conditions. The relationship is based on the forward model transfer function refined by Gudmundsson and Raymond in 2008~\cite{Gudmundsson_2008}. 
Ockenden et al apply this framework in reverse to infer the bedrock from modern, high-resolution satellite data estimates. A restrictive assumption in the modeling design by Ockenden et al might be their assumption of ``constant viscosity'', this means that the strain rate is directly proportional to the stress. This is in contrast to the more commonly used non-linear Glen's Flow Law, where $n$ is typically around 3 or even 4.

\subsection{Bridging Classical and Modern Approaches}
Budd's approach provides fundamental physical understanding of how specific wavelengths propagate through ice, establishing theoretical limits on what bed features can be detected from surface observations. Ockenden's method extends this to practical applications using real satellite data but relies on linearised assumptions that may break down under certain conditions. By systematically exploring how different rheological models $(n = 1$ vs $n = 4)$ and basal conditions affect the bed-to-surface transfer functions, my work aims to develop a robust inversion method that can better handle the nonlinear physics of ice flow.
While simultaneously utilising the wealth of presently available satellite-derived surface data. My approach with BedSAT builds upon theoretical foundations and recent inversion methods to better understand how bed conditions—including slipperiness, roughness, and pinning points—affect both grounding line retreat rates and their surface expressions. BedSAT will connect surface observations with bed topography using more realistic rheological and geometric assumptions through an iterative process: Initially inverted bed topography integrated into forward models allowing for progressive improvement via comparison with established datasets like NASA's ITS\_LIVE.

\subsection{Physics-Informed Machine Learning for Bed Topography Inversion}\label{ML}
In the rheology and sliding study in section~\ref{study1}, I am establishing a ``forward problem'' investigation—how bed topography influences surface expression—under various physical assumptions. However, the goal of BedSAT is to solve the ``inverse problem'': Inferring bed topography from surface observations. I plan to use Physics-Informed Machine Learning (Physics-ML), leveraging NVIDIA PhysicsNeMo which is designed to create high-fidelity, deep learning models, blending the governing physics of a system—Partial Differential Equations (PDEs)—with training data~\cite{NVIDIA_NeMo_2025}. 

\begin{enumerate}
\item{Forward model training}: PhysicsNeMo can learn the relationship between ice sheet surface velocity and elevation to bed topography, basal sliding and ice rheology ($n=1$ vs $n=4$). The model can then generate vast amounts of synthetic training data—including variations with statistically realistic aeolian noise—orders of magnitude faster than a traditional ISSM solver.

\item{Solving the Inverse Problem}: PhysicsNeMo is explicitly designed to solve inverse problems by using observational data to infer unknown system parameters~\cite{NVIDIA_NeMo_2025}. BedSAT will rely on the PhysicsNeMo data-driven architecture to learn the mapping from surface expression to bed topography, effectively creating a fast and accurate inverse solver.
\end{enumerate}

By integrating PhysicsNeMo, BedSAT will develop into a model capable of near real-time inference, satisfying my project's third objective: Allowing for rapid sensitivity analyses of ice mass loss projections to different realisations of topographic roughness. This approach goes beyond traditional inversion methods, enhancing computational efficiency and physical realism of Antarctic bed topography reconstruction.