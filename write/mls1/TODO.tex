\chapter{TODO list}

\begin{enumerate}
\item Complete the missing literature review section

\item Add interpretation of phase analysis results: 
Given current findings (what does this mean for my inversion approach???)
    An isolated bump differs from the 90° prediction. Because the derivation in the Budd's 1970 is based on a periodic, harmonic bedrock (continuous, infinitely repeating cosine wave). The bump in IsmipF is a fundamentally different geometry. It does not have a single wavelength and introduces edge effects not accounted for in this periodic model.
    I want to include a periodic cosine wave to test budd more directly but the simulations are tricky (need to figure out correct resolution factors)

\item HOW CAN I USE the sliding study in the inversion method?
    \begin{itemize}
       \item different transfer functions for different rheological assumptions? 
       \item test which rheology best matches observed data?
    \end{itemize}

\item Consider whether timeline is realistic or if scope adjustment is needed

\item Include more discussion of limitations and potential challenges
\end{enumerate}

% Machine learning connection: You mention on page 20 that the bedrock database will be used to "train an image recognition (machine learning) model," but this isn't well-integrated with your earlier methodology description. Consider adding a brief mention of this in Section 3.2.2.