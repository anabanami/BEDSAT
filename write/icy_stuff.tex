\chapter{Ice is weird}
\section{Temperate Ice Sliding: An Empirical Study}

% Static and dynamic friction are different.
% Limiting static shear stress: is the tipping point where pressure overcomes friction and sliding begins. The interface of ice on bedrock has very high static friction.

% how does the ice respond to different surfaces (smooth, rough, different material compositions)?
% how does the sliding speed varies from one surface to another?
% Rough surfaces requires more force to overcome static friction.

% Glaciers carve away landscapes via friction and the landscape makes the ice slide.

The main objectives described in\cite{Budd_Keage_Blundy_1979} are to describe the relationship between forces and ice movement (sliding) over different surfaces, and how the moving ice affects the surfaces over which they slide.

\begin{enumerate}
\item How did the researchers apply normal and shear stresses to the ice blocks in their experiments?

\item Describe the relationship between limiting static shear stress and normal load observed in the experiments.

\item How did sliding velocity vary with shear stress and normal stress at low normal loads? % The experiments showed a direct proportionality between limiting static shear stress and normal load, indicating a constant coefficient of limiting friction specific to each slab.

\item What was the significance of the product (TmVb) in the experiments?
% The product TmVb, representing the transition point from steady-state sliding to acceleration, tended towards a constant value, suggesting a critical threshold for dynamic instability.

\item How did the relationship between sliding velocity and shear stress change at high normal loads?
% At low normal loads, sliding velocity increased proportionally with shear stress and decreased proportionally with both normal load and surface roughness.
% "The sliding speed at high stresses is not linear, velocity increases with the cube of the shear stress*???* small force increase can lead to a lot of speed!
% At high normal loads, the relationship shifted from linear to cubic, with sliding velocity increasing proportionally to the cube of shear stress and inversely proportionally to normal stress.


\item What effect does an increase in the water table have on the effective normal stress acting on the glacier base?
% An increase in the water table elevates the basal water pressure, which counteracts the normal stress from the overlying ice, effectively reducing the effective normal stress acting on the glacier base.

\item Explain how the study's findings might help explain the high velocities observed in fast-outlet polar glaciers.
% The study demonstrated that sliding velocity is highly sensitive to effective normal stress. For fast-outlet polar glaciers, where the base is often below sea level, the buoyancy effect of seawater can significantly reduce the effective normal stress, potentially leading to higher sliding velocities.

\item What was the observed relationship between erosion and the experimental parameters (normal stress, shear stress, and velocity)?
% Erosion was found to increase with higher normal stress, shear stress, and velocity, suggesting a combined effect of these parameters on the rate of material removal from the slab surface.

\item Why did the researchers conclude that ice deformation, rather than regelation, plays a dominant role in the observed sliding behaviour?
\end{enumerate}% The observed cubic relationship between sliding velocity and shear stress at high normal loads pointed towards ice deformation, rather than regelation, as the dominant mechanism governing sliding behaviour at the experimental scales.
    

Essay Questions
\begin{enumerate}
\item Discuss the limitations of the experimental setup used in the study and how these limitations might affect the applicability of the findings to real-world glacier systems.
\item Compare and contrast the roles of regelation and ice deformation in glacier sliding, drawing upon the findings of the study to support your arguments.
\item Analyze how the study's results contribute to a better understanding of the dynamics of glacier surges, focusing on the factors that lead to the onset and propagation of surge events.
\item Explain the concept of effective normal stress in the context of glacier sliding and discuss how variations in basal water pressure can influence glacier flow.
\item Critically evaluate the significance of the study's findings for modelling glacier behaviour and predicting future glacier response to climate change.
\end{enumerate}
    
    

