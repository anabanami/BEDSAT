\chapter{Ice is weird}
A Newtonian fluid (like water) has a constant viscosity regardless of the flow conditions, while a non-Newtonian fluid's viscosity changes based on factors like strain-rate. The viscosity of ice is not constant.depends on how much it is deforming (shear rate).
Ice is a slow, shear-thinning fluid. ``shear-thinning" means that the viscosity of ice decreases with increasing strain rate. This means that under more strain forces ice becomes "softer" and flows more easily.``Slow" means that the flow of ice occurs at very low velocities
\begin{equation}
\rho(\vec{u_t} + \vec{u} \cdot \nabla \vec{u}) = 0,
\end{equation}
i.e. the change in flow velocity (time dependent and convective) is approximately zero.
This assumption greatly simplifies the Navier-Stokes equations for ice flow.
The equations we use to are the incompressibility condition 
\begin{equation}
\nabla \vec{u} = 0,
\end{equation}
i.e. the divergence of the velocity field is zero. The force balance equation
\begin{equation}
-\nabla p + \nabla \cdot \tau_{ij} + \rho g, = 0
\end{equation}
i.e. the pressure gradient, the divergence of the stress tensor (viscous forces within the ice) and the gravitational body force acting on the ice all cancel out, dnd finally Glen's flow law
\begin{equation}
D_{ij} = A\tau^{n} \tau_{ij},
\end{equation}
where the strain rate tensor $D_{ij}$, which describes how fast the ice is deforming is proportional to the deviatoric stress tensor $\tau_{ij}$ and it's magnitude $\tau$, which accounts for the stress caused by deformation (as opposed to isotropic stress like pressure). The flow law exponent $n$ determines how strongly the flow rate depends on stress. For ice, Glen's law uses $n=3$, which implies a nonlinear relationship between stress and strain rate, meaning the flow rate accelerates rapidly with increased stress\cite{modelling_ppt}.

This model does not have time derivatives anymore, this means that a time-stepping ice sheet program recomputes the full velocity field at every time step and does not require velocity information from the previous time step.


\chapter{Shallow Ice Approximation}
(Bons et al 2018)
Glen's law exponent n can range from 1 to 5???woah!?


low driving stresses make SIA fail?



% \section{Ice flow over bedrock perturbations}
% This 1970's rheology paper by Budd explains how bedrock irregularities beneath a glacier affect the surface shape of the ice mass. Budd develops a mathematical model to describe the flow of ice over these undulations, considering the ice as a viscous fluid that deforms under stress. The model predicts that the surface shape of the glacier will mirror the bedrock undulations, but shifted out of phase by approximately $\frac{\pi}{2}$ radians. The paper analyzes the damping of different wavelengths of bedrock undulations, finding that waves with a length roughly three times the ice thickness are minimally damped, while shorter or longer waves are significantly damped out. Finally, the implications of this theory proposes the potential for ice to flow uphill, concluding that bedrock undulations with wavelengths several times the ice thickness are most important in controlling ice motion.

% % \section{Temperate Ice Sliding: An Empirical Study}
% % % Static and dynamic friction are different.
% % % Limiting static shear stress: is the tipping point where pressure overcomes friction and sliding begins. The interface of ice on bedrock has very high static friction.

% % % how does the ice respond to different surfaces (smooth, rough, different material compositions)?
% % % how does the sliding speed varies from one surface to another?
% % % Rough surfaces requires more force to overcome static friction.

% % % Glaciers carve away landscapes via friction and the landscape makes the ice slide.

% % The main objectives described in\cite{Budd_Keage_Blundy_1979} are to describe the relationship between forces and ice movement (sliding) over different surfaces, and how the moving ice affects the surfaces over which they slide.

% % \begin{enumerate}
% % \item How did the researchers apply normal and shear stresses to the ice blocks in their experiments?

% % \item Describe the relationship between limiting static shear stress and normal load observed in the experiments.

% % \item How did sliding velocity vary with shear stress and normal stress at low normal loads? % The experiments showed a direct proportionality between limiting static shear stress and normal load, indicating a constant coefficient of limiting friction specific to each slab.

% % \item What was the significance of the product (TmVb) in the experiments?
% % % The product TmVb, representing the transition point from steady-state sliding to acceleration, tended towards a constant value, suggesting a critical threshold for dynamic instability.

% % \item How did the relationship between sliding velocity and shear stress change at high normal loads?
% % % At low normal loads, sliding velocity increased proportionally with shear stress and decreased proportionally with both normal load and surface roughness.
% % % "The sliding speed at high stresses is not linear, velocity increases with the cube of the shear stress*???* small force increase can lead to a lot of speed!
% % % At high normal loads, the relationship shifted from linear to cubic, with sliding velocity increasing proportionally to the cube of shear stress and inversely proportionally to normal stress.


% % \item What effect does an increase in the water table have on the effective normal stress acting on the glacier base?
% % % An increase in the water table elevates the basal water pressure, which counteracts the normal stress from the overlying ice, effectively reducing the effective normal stress acting on the glacier base.

% % \item Explain how the study's findings might help explain the high velocities observed in fast-outlet polar glaciers.
% % % The study demonstrated that sliding velocity is highly sensitive to effective normal stress. For fast-outlet polar glaciers, where the base is often below sea level, the buoyancy effect of seawater can significantly reduce the effective normal stress, potentially leading to higher sliding velocities.

% % \item What was the observed relationship between erosion and the experimental parameters (normal stress, shear stress, and velocity)?
% % % Erosion was found to increase with higher normal stress, shear stress, and velocity, suggesting a combined effect of these parameters on the rate of material removal from the slab surface.

% % \item Why did the researchers conclude that ice deformation, rather than regelation, plays a dominant role in the observed sliding behaviour?
% % \end{enumerate}% The observed cubic relationship between sliding velocity and shear stress at high normal loads pointed towards ice deformation, rather than regelation, as the dominant mechanism governing sliding behaviour at the experimental scales.
    

% % Essay Questions
% % \begin{enumerate}
% % \item Discuss the limitations of the experimental setup used in the study and how these limitations might affect the applicability of the findings to real-world glacier systems.
% % \item Compare and contrast the roles of regelation and ice deformation in glacier sliding, drawing upon the findings of the study to support your arguments.
% % \item Analyze how the study's results contribute to a better understanding of the dynamics of glacier surges, focusing on the factors that lead to the onset and propagation of surge events.
% % \item Explain the concept of effective normal stress in the context of glacier sliding and discuss how variations in basal water pressure can influence glacier flow.
% % \item Critically evaluate the significance of the study's findings for modelling glacier behaviour and predicting future glacier response to climate change.
% % \end{enumerate}
    
    
