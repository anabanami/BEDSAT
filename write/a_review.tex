\chapter{Topographic Models of Antarctica: A Review}\label{n2}
Numerical modelling and sedimentary sequence interpretation suggest cyclical periods of ice-sheet expansion and retreat\cite{Young_2011}. Using ice-penetrating radar data to generate a new basal bed topography of the Aurora Subglacial Basin (ASB) in east Antarctica is characterised by a fjord landscape (this land is under $\sim$ under $2-4.5$ km of ice). The ASB has a potentially significant influence on the east Antarctic ice-sheet (EAIS), however there is high uncertainty in estimates of past and present global sea level changes due to the scarcity of bed data\cite{Young_2011}. This uncertainty also limits the accuracy of models used to predict future ice sheet growth or decay.\\
{\large Methods in\cite{Young_2011}}
\begin{enumerate}
    \item A ski-equipped airplane (DC-3T) carried a radar system (HiCARS), which can see through ice. HiCARS sends signals that bounce back to show the thickness of the ice and the shape of the land beneath it.
    \item The plane flew back and forth over a large area, covering distances of around 1,000 km. The flights took place over two different periods in 2008–2009 and 2009–2010.
    \item The radar data was cleaned up (processed) to improve accuracy, and they used a special radar system that helps reduce distortions (errors) in the measurements. \textbf{[HOW?]}
    \item Thickness of the ice was measured using the time it took for the radar signals to travel through the ice and back, assuming the radar signals move through the ice at a specific speed (169 meters per microsecond).
    \item The height of the land below the ice was calculated by looking at the radar-determined surface of the ice. \textbf{[WHAT?]}
    \item The radar data was combined with other existing datasets (BEDMAP) to improve the overall picture. They used a computer algorithm to fill in gaps where they didn’t have direct measurements. \textbf{[WHICH?]}
    \item Determining how rough or uneven the land under the ice was, by using a statistical measure called the ''root mean squared (rms) deviation."
\end{enumerate}
In short, Young et al. used advanced radar technology on an airplane to map the ice thickness and the landscape beneath it in a region of Antarctica, combining this data with previous maps for a better overall picture.
