% latexmk -pvc -pdf
\documentclass[12pt, a4paper, openany]{book}
\usepackage[margin=2.5cm]{geometry}
\usepackage{titling}
\usepackage{titlesec}
\usepackage{amsmath,amsthm,amsfonts,amssymb}
\usepackage{dsfont}
\usepackage{mathtools}
\usepackage{braket}
\usepackage[font=scriptsize,labelfont=bf]{caption}
\usepackage[english]{babel}
\usepackage{float} 
\usepackage{pdflscape} 
\usepackage{cite}

\usepackage[T1]{fontenc}
\usepackage[utf8]{inputenc}

\usepackage{datetime}
\newdateformat{monthyeardate}{%
  \monthname[\THEMONTH], \THEYEAR}

\usepackage{graphicx}
\newenvironment{Figure}
    {\par\medskip\noindent\minipage{\linewidth}}
    {\endminipage\par\medskip}
\usepackage{tabularx}

\newenvironment{abstract}
{\clearpage \thispagestyle{empty} \null \vfill \begin{center} \bfseries \large Abstract \end{center}}
{\vfill \null \clearpage}

% Colours:
\usepackage[table]{xcolor} % for setting colors
\definecolor{purple}{RGB}{117,77,226}

\usepackage[bookmarksopen,
  pagebackref,
  pdfpagelayout=TwoPageRight,
  colorlinks=true,
  urlcolor=purple,
  citecolor=purple,
  filecolor=purple,
  linkcolor=purple,
  ]
{hyperref}

\usepackage{listings}
\usepackage{xcolor} % Necessary to define custom colors

\definecolor{softblue}{rgb}{0.3, 0.5, 0.8}
\definecolor{darkergreen}{rgb}{0, 0.5, 0}

\lstset{
  basicstyle=\ttfamily\small, % Set the typeface to typewriter and small size
  breaklines=true, % Enable line breaking
  postbreak=\mbox{\textcolor{red}{$\hookrightarrow$}\space}, % Marker for line breaks
  numbers=left, % Line numbers on left
  numberstyle=\tiny, % Set the size of the numbers
}

\lstdefinestyle{mystyleC}{
    language=C,      
    commentstyle=\color{softblue},
    morecomment=[l]{//},  % Line comment
}

\lstdefinestyle{mystylePython}{
    language=Python,      
    commentstyle=\color{darkergreen},
    morecomment=[l]{//},  % Line comment
}

\titleformat{\chapter}[display]
  {\normalfont\bfseries}{}{0pt}{\Huge}


% Required packages
\usepackage{catchfile}

% Command to execute texcount and capture the word count
\newcommand\wordcount[1]{
    \immediate\write18{texcount #1.tex | grep "Words in text" | cut -d: -f2 > #1.wordcount.tmp}
    \CatchFileDef{\mywordcount}{#1.wordcount.tmp}{}
    \immediate\write18{rm #1.wordcount.tmp} % to remove the temporary file
    \mywordcount % to print the word count
}

\begin{document}

\begin{titlepage}
\begin{center}
    {\Huge BEDSAT: Antarctica}\\ [1cm] 
    {\Large Exploring what lies beneath using big data and modelling}\\
    \vspace{5cm}
    {\large Ana Fabela Hinojosa\footnote{ana.fabelahinojosa1@monash.edu}}\\
    \monthyeardate\today\\ [1cm]
    Supervisors:\\
    Dr. Felicity McCormack\\
    Dr Jason Roberts\\
    Dr Richard Jones\\ [2cm]
    Panel:\\
    Dr. ??? \\
    Dr ??? \\
    Dr ??? \\ [3.5cm]
    \includegraphics[scale=0.2]{logos.png}

    \end{center}

\end{titlepage}


%----------------------------------------------------------------------------------------
%   QUOTATION PAGE
%----------------------------------------------------------------------------------------
% \vspace*{0.2\textheight}

% \noindent{``El alacrán clavándose el aguijón, harto de ser un alacrán pero necesitando de su alacranidad para acabar con el alacrán''.}\bigbreak
% % \vspace{-2cm}
% \hfill Julio Cortázar\bigbreak

% \begin{abstract}\label{abstracc}
% Antarctica has been losing ice mass over recent decades, contributing to rising sea levels (SLR). There is significant uncertainty regarding the extent and timing of this ice loss. One key factor influencing ice flow and loss is bed topography, typically derived from sparse airborne radar surveys. The uncertainty in the data can impact simulations of the Antarctic Ice Sheet (AIS) evolution under climate change.
% We need alternative approaches to surveying Antarctica and given the logistical challenges we plan on modelling the bed topography.
% Our model's accuracy (surely?) depends on the spacing of ice thickness measurements and uncertainties in ice velocity (SOMETHING I AM WORKING ON understanding rn) and surface mass balance. 
% BedSAT, aims to leverage the mathematical relationship between ice surface elevation (data we have?) and bed topography to estimate the actual bed topography of a surrounding spatial region (in 2D?). 
% \end{abstract}

% \chapter*{Acknowledgements}

% \tableofcontents

% introduction
\chapter{Antarctica's Landscape}\label{why}
\section{Climate Impacts and Global Significance}

The polar regions are losing ice, and their oceans are changing rapidly\cite{O_C_in_changingClimate}. The consequences of this extend to the whole planet and it is crucial for us to understand them to be able to evaluate the costs and benefits of potential mitigation. 

Changes in different kinds of polar ice affect many connected systems. Of particular concern is the accelerating loss of continental ice sheets (glacial ice masses on land) in both Greenland and Antarctica, which has become a major contributor to global sea level rise\cite{O_C_in_changingClimate}. Impacts extend beyond direct ice loss: as fresh water from melting ice sheets is added into the ocean, it increases ocean stratification disrupting global thermohaline circulation\cite{Jacobs_2004}. In addition, cold freshwater can dissolve larger amounts of $\mathrm{CO_2}$ than regular ocean water creating corrosive conditions\cite{O_C_in_changingClimate}.
 
While there is high confidence in current ice loss and retreat observations in many areas, there is more uncertainty about the mechanisms driving these changes and their future progression\cite{Fox-Kemper_2021}. Uncertainty increases in regions with variable bed conditions, where characteristics like ``slipperiness'' and ``roughness'' are difficult to verify via direct observations. Other problematic areas involve the ice sheet's grounding line (GL), the zone that delineates ice grounded on bedrock from ice shelves floating over the ocean. The retreat rate depends crucially on topographical features like pinning points\cite{Fox-Kemper_2021}, which lead to increased buttressing by the ice shelf on the upstream ice sheet. However, major knowledge gaps persist in mapping bed topography across Antarctic ice sheet margins - with over half of all margin areas having insufficient data within 5 km of the grounding zone\cite{RINGS_2022}. Addressing this data gap through both systematic mapping and improved interpolation utilising auxiliary data streams with more complete coverage would significantly improve both our understanding of current ice dynamics and the accuracy of ice-sheet models projecting future changes.

\chapter{Topography of Antarctica}\label{review}

Bed topography is one of the most crucial boundary conditions that influences ice flow and loss from the Antarctic Ice Sheet (AIS)\cite{Morlighem_2020}. Bed topography datasets are typically generated from airborne radar surveys, which are sparse and unevenly distributed across the Antarctic continent (see figure \ref{fig:BedMAP}). Interpolation schemes to ``gap fill'' these sparse datasets yield bed topography estimates that have high uncertainties (i.e. multiple hundreds of metres in elevation uncertainty; Morlighem et al., 2020) which propagate through simulations of AIS evolution under climate change\cite{Castleman_2022}. Given the logistical challenges of accessing large parts of the Antarctic continent, there is a crucial need for alternative approaches that integrate diverse and possibly more spatially complete data streams – including satellite data.
\begin{figure}[H] % Forces the figure exactly HERE
    \includegraphics[scale=0.4]{bedmap.png}
    \caption{Distribution of BedMAP\{1,2,3\} data tracks (Source: bedmap.scar.org).}
    \label{fig:BedMAP}
\end{figure}

\section{Approaches to Bed Topography Reconstruction}

A key objective of this study is to understand how bed topography influences ice dynamics, and the bed topography itself. There are two ways that we can infer information about this relationship: Through forward modelling, with assumptions of the bed conditions; and through inverse modelling that relies on surface observations.
\begin{itemize}
    \item\textbf{Forward models}
    The aim of forward models is to see how bed properties impact ice dynamics. A key example is using a large ensemble of bed topographies to investigate how bed uncertainties impact simulated ice mass loss. In this example geostatistical methods can be used to generate bed topographies that either preserve elevation or texture:    
    \begin{itemize}
            \item\textbf{Geostatistics} Statistical methods specialized for analyzing spatially correlated data. In glaciology, this approach is used to interpolate between sparse measurements and characterise spatial patterns in bed properties, often employing techniques like kriging\cite{Mackie_2020}.

    \end{itemize}

    \item\textbf{Inversion models}
    The aim of these models is to understand bed properties through knowledge of surface or other variables. A key example is the retrieval od bed topography or basal slipperiness from surface elevation and velocities.

        \begin{itemize}
            \item\textbf{Control method inversion}: A variational approach that minimizes mismatches between observed and simulated fields through a cost function approach. Remote sensing data and theoretical ice flow models are used to obtain basal conditions\cite{deRydt_2013}. Often needs regularization terms to prevent non-physical features or over-fitting\cite{Morlighem_Goldberg_2024}.

            \item\textbf{Mass conservation}: Used to constrain inversion models and fill data gaps by employing physical conservation laws, particularly effective for reconstructing bed topography where direct measurements are sparse~\cite{Morlighem_2017, Morlighem_2020}. Requires (contemporary) measurements of ice thickness at the inflow boundary to properly constrain the system\cite{Morlighem_Goldberg_2024}.

            \item\textbf{Markov Chain Monte Carlo (MCMC)}: A probabilistic method that generates sample distributions to quantify uncertainties in ice sheet parameters and models\cite{Morlighem_Goldberg_2024}. While powerful for uncertainty quantification, these methods remain computationally intensive for continental-scale ice sheet models\cite{Morlighem_Goldberg_2024}.

            \item\textbf{4dvar}: Four-dimensional variational data assimilation - Minimizes the difference between model predictions and observations across a time window. Mainly used to optimize model parameters and initial conditions\cite{Morlighem_Goldberg_2024}. Can handle time-varying data and evolving glacier states, making it more suitable for dynamic systems unlike control methods, this makes them more computationally demanding\cite{Morlighem_Goldberg_2024}.

            \item\textbf{EnKF} Ensemble Kalman Filter. A sequential data assimilation method that uses an ensemble of model states to estimate uncertainty and update model parameters based on observations\cite{Morlighem_Goldberg_2024}.
        \end{itemize}
    
\end{itemize} 
This study aims to develop an integrated method combining forward and inverse modeling to improve bed topography estimates by leveraging high-resolution satellite surface data in regions where radar data is sparse. Despite revolutionary advances in satellite technology providing unprecedented surface detail, a key challenge in glaciology remains: fully utilizing this wealth of information where subglacial understanding is limited. Our approach will integrate more comprehensive models with modern computational capabilities, with the specific methodology chosen based on research objectives, data availability, and computational resources.

\newpage
\section{Theoretical Frameworks}
 Understanding how bed features manifest in surface observations requires a theoretical framework that connects these two domains. The modelling approach used on this project relies on two different theoretical frameworks that relate bed topography and surface features. Using observations and these modelling frameworks, my goal is understanding the limitations of each approach and how they can be improved

The first framework was originally developed by Budd \cite{Budd_1970}. This model relates ice flow over bedrock perturbations to surface expressions using a two-dimensional biharmonic stress equation. 

The model's foundation rests on two key simplifications:
\begin{itemize}
    \item Most shear deformation occurs at the base of the ice sheet
    \item Explicit consideration of longitudinal stresses and strain-rates
\end{itemize}

The modelling carried out in\cite{Budd_1970} determined ice-sliding velocities for wide ranges of roughness, normal stress, and shear stress relevant to real glaciers\cite{Budd_1970}. Despite its robustness, Budd's mathematical framework remains notably underutilized in modern ice sheet modeling. 

The second framework in my plan analyses how shape and mechanical properties of the ice bed significantly influence how ice flows, with changes at the bed potentially leading to large differences in predicted ice loss rates\cite{Ockenden_2022}. Recent work by Ockenden et al. demonstrates both the capabilities and limitations of current inversion approaches in addressing this problem.
The core principle of the method by Ockenden et al. (2022) relies on the fact that variations in basal topography, slipperiness, and roughness cause measurable disturbances to the surface flow of the ice. Through linear perturbation analysis, they establish a systematic relationship between surface observations and bed conditions. This relationship can be expressed as $y=f(x)$, where $y$ represents surface measurements (velocity and topography), $x$ represents bed properties (topography and slipperiness), and $f$ is the forward model\cite{Gudmundsson_2008}. The inversion process, $x=f^{-1}(y)$, estimates bed conditions from surface observations.%Could add a sentence here that there is frequently a horizontal offset between the bed and surface expression of features.

The method works best when analyzing perturbations that are small relative to mean properties, under specific conditions including:
\begin{enumerate}
    \item A linear viscous medium ($n=1$)
    \item Non-linear sliding law ($m>0$)
    \item Steady-state conditions
    \item Spatially constant zero-order solutions
\end{enumerate}

Using high-resolution datasets (REMA surface elevation at 8m resolution and NASA ITS\_LIVE velocity at 120m resolution), their approach performs well for:
\begin{itemize}
    \item Areas with moderate topographic gradients
    \item Features not aligned with ice flow direction
    \item Medium-wavelength (5-50km) bedrock features
\end{itemize}

However, significant limitations emerge when:
\begin{itemize}
    \item Dealing with steep topography where the shallow-ice-stream approximation breaks down
    \item Handling variable slipperiness parameters
    \item Attempting to validate slipperiness predictions due to lack of ground-truth data
\end{itemize}


\section{Current Opportunities}

The review of current approaches to Antarctic bed topography reconstruction reveals significant methodological limitations. While theoretically robust, methods such as the inversion technique employed by Ockenden et al. often rely on simplifying assumptions that inadequately capture complex ice-bed interactions, particularly in regions with steep topography where the shallow-ice-stream approximation breaks down.
With BedSAT, I aim to address these limitations by building upon the theoretical foundations established by Budd and recent inversion methods. My primary objective is to better understand how variable bed conditions such as ``slipperiness,'' ``roughness,'' and pinning points influence both GL retreat rates and their surface manifestations. BedSAT will bridge the disconnect between surface observations and bed topography by implementing a more realistic set of rheological and geometric assumptions. The methodology will follow an iterative inversion-forward modeling validation cycle: initially inverted bed topography will be used in ice dynamics (forward) models with BedSAT's improved assumptions, allowing comparison between resulting surface predictions and established datasets like NASA's ITS\_LIVE (see Chapter \ref{resources}). This process will be systematized through machine learning methods, ultimately enhancing the analytical capabilities for the final phase of my project.
 
% INCLUDES research significance
\wordcount{literature_review} words in this section.
% planning
\chapter{Objectives}

The overall aim of this project is to derive a new Antarctic bed topography using remote sensing data, airborne derived-estimates of the bed and ice sheet modelling. Using the new bed topography to improve understanding of the impact of fine-scale topographic roughness on ice and subglacial hydrological flow, and projections of ice mass loss under climate warming.\\
\\The specific objectives are:
\begin{enumerate}
    \item Develop an ice sheet modelling approach to assimilate satellite remote sensing datasets to improve knowledge of the bed (BedSAT) informed by mathematical models of ice flow over topography;
    \item Derive a new bed topography for Antarctica using BedSAT;
    \item Conduct sensitivity analyses to understand the impact of the improved bed topography on projections of ice mass loss from Antarctica under climate warming.
\end{enumerate}
\wordcount{objectives} words in this section.
\chapter{Objectives and Methodology}

The overall aim of this project is to derive a new Antarctic bed topography using remote sensing data, airborne derived-estimates of the bed and ice sheet modelling. Using the new bed topography to improve understanding of the impact of fine-scale topographic roughness on ice and subglacial hydrological flow, and projections of ice mass loss under climate warming.\\
\\The specific objectives are:
\begin{enumerate}
    \item Develop an ice sheet modelling approach to assimilate satellite remote sensing datasets to improve knowledge of the bed (BedSAT) informed by mathematical models of ice flow over topography;
    \item Derive a new bed topography for Antarctica using BedSAT;
    \item Conduct sensitivity analyses to understand the impact of the improved bed topography on projections of ice mass loss from Antarctica under climate warming.
\end{enumerate}

The first phase of the project (objective 1) is to derive the BedSAT method. This will involve the integration of the Budd~\cite{Budd_1970} mathematical model relating ice surface elevation and bed topography into ISSM, and the development of a methodology for the data assimilation into ISSM. I will use a regional catchment in Antarctica for which relatively more radar data are available and has an indicative range of topography features, e.g. the Aurora Subglacial Basin, East Antarctica, extensively surveyed by the ICECAP project for airborne geophysics~\cite{Young_2011}. The second phase of the project (objective 2) will apply the methodology developed in objective 1 to the whole Antarctic continent, deriving a continent-wide bed topography dataset. Using covariance properties from existing radar surveys, I will generate a number of realisations of bed topography with unique high-resolution, and statistically-consistent topographic roughness. The third phase of the project will use the new bed topography datasets to conduct a sensitivity analysis of ice sheet model projections to 2300 CE, investigating the impact of the new topography and different realisations of roughness on ice and subglacial hydrological flow and ice mass loss from Antarctica.\\

\section*{Plan:}
\subsection*{Objective 1}
\begin{enumerate}
\item Develop a method to interpolate topography that ensures consistent surface expressions with observations
\item Reduce RMS error between observations and model predictions
\end{enumerate}
\subsection*{Key Investigation Areas}
\begin{enumerate}
\item\textbf{Model Development Strategy}
    \begin{itemize}
    \item We will maintain invariant bed traction throughout our modeling timeframe to isolate topographical effects in our inversion approach, with validation through sensitivity tests in regions where bed properties are well-known.
    
    \item Our model will incorporate available thermal distribution, velocity field, and ice thickness data, as these parameters are essential for accurate ice flow representation and can be constrained using radar observations.
    
    \item Through spectral analysis of surface expressions, we will identify the topographical features that most strongly influence surface patterns, using available high-resolution surface elevation data for validation.
    
    \item To account for variations in ice behavior with thickness, we will simulate scenarios ranging from thick ice with slippery base to thin ice with sticky base, using observed velocity patterns as constraints.
    \end{itemize}

\item\textbf{Transfer Functions}
    \begin{itemize}
    \item We will develop efficient transfer function methods for rapid bed topography inversion, validating against known bed configurations from radar data.
    
    \item Our transfer functions will be tested across various ice thickness and flow conditions, with validation against different glacial systems.
    
    \item Cross-validation against radargrams will provide direct verification of our inversion results, including uncertainty quantification through comparison with measured bed elevations.
    
    \item Spatial covariance analysis of existing radar data will inform our statistical framework and error propagation through the inversion process.
    
    \item We will account for friction roughness and high-amplitude variations in our analysis, using observed surface velocity patterns as constraints.
    \end{itemize}

\item\textbf{Model Validation}
    \begin{itemize}
    \item We will apply quantitative error reduction metrics and compare systematically against existing bed topography products.
    
    \item Model limitations and breaking points will be identified through systematic testing across extreme scenarios, constrained by physical principles.
    
    \item Singular Value Decomposition (SVD) analysis will help identify key modes of variability in our solutions.
    
    \item Grid independence testing will ensure solution robustness across different spatial resolutions.
    
    \item Sensitivity analysis will examine the impact of our model assumptions, particularly regarding basal conditions and ice rheology.
    \end{itemize}
\end{enumerate}


\subsection*{Confounding Factors to Consider}

Basal friction at the ice-bed interface plays a crucial role in how bed topography is expressed at the surface. Understanding how different factors affect basal friction is essential for accurately interpreting surface expressions and inverting them to determine bed topography:

\begin{itemize}
    \item Sliding behavior: The relationship between basal stress and sliding velocity affects how ice flows over the bed. Areas with enhanced sliding can mask bed features in surface expressions, while areas with stronger friction tend to show more pronounced surface expressions of bed topography.
    
    \item Rheological properties: Ice viscosity varies with temperature and stress state, affecting how efficiently bed topographic signals propagate to the surface. Softer ice tends to dampen bed topography signals more than stiffer ice.
    
    \item Thermomechanical responses: The temperature-dependent nature of ice deformation means that warmer, softer ice near the bed behaves differently from colder, stiffer ice above. This vertical variation in ice properties affects how bed topography signals are transmitted to the surface.
    
    \item Slippery spots: Localized areas of reduced friction, often due to the presence of water or deformable sediments, can create surface expressions that might be misinterpreted as bed topography features.
\end{itemize}

These factors directly impact our ability to invert surface data for bed topography, as they can either enhance or mask the relationship between bed features and their surface expressions. Our methodology must account for these effects to avoid misinterpreting surface features.


% \section*{Plan:}
% \subsection*{Objective 1}
% \begin{enumerate}
% \item Develop a method to interpolate topography that ensures consistent surface expressions with observations
% \item Reduce RMS error between observations and model predictions
% \end{enumerate}

% \subsection*{Key Investigation Areas}
% % this is great, but perhaps make each point a sentence, and include: why it's relevant to your work, citation, how you might constrain it
% \begin{enumerate}
% \item\textbf{Model Development Strategy}
%     \begin{itemize}
%     \item Focus on invariant bed traction over the time-frame
%     \item Consider thermal distribution, velocity field, and ice thickness
%     \item Perform analysis to identify which topographical feartures affecting surface expression amplitude and positions
%     \item Note: Ice behavior varies with thickness. Simulate the following different scenarios: (thick ice → slippery base, thin ice → sticky base)
%     \end{itemize}
% \item\textbf{Transfer Functions}
%     \begin{itemize}
%     \item Develop rapid transfer function construction methods
%     \item Use for various conditions and domains
%     \item Validate against ICECAP cross-section radargrams
%     \item Analyze spatial covariance of existing data
%     \item Consider friction roughness and high amplitude variations
%     \end{itemize}
% \item\textbf{Model Validation}
%     \begin{itemize}
%     \item Quantify error reduction
%     \item Identify model breaking points
%     \item Use SVD (Singular Value Decomposition)
%     \item Test grid independence
%     \item Evaluate sensitivity to assumptions
%     \end{itemize}
% \end{enumerate}


% \subsection*{Confounding Factors to Consider}
% % this needs some more description - recommend writing a short paragraph describing how these each influence basal friction, and what that means for what we might expect in terms of surface expressions, and how that relates to your aims of generating bed topography

% Basal friction coefficient effects on:
% \begin{itemize}
%     \item Sliding behavior
%     \item Rheological properties
%     \item Thermomechanical responses (stiff vs soft ice)
%     \item Slippery spots
%     \end{itemize}

% \subsection*{Workflow Plan}
% \begin{enumerate}
% \item\textbf{Initial Modeling Phase}
%     \begin{itemize}
%     \item Exclude surface elevation initially (reserve for control model)
%     \item Use 2D Gaussian with $3\times$ ice thickness
%     \item Analyze REMA spectral components at various frequencies
%     \item Determine reasonable Signal-to-Noise ratio levels
%     \end{itemize}

% \item\textbf{Inversion Development}
%     Parameters to consider:
%     \begin{itemize}
%         \item Basal traction
%         \item Internal temperature distribution
%         \item Heat flux
%         \item Additional rheological parameters
%     \end{itemize}

% \item\textbf{Model Testing}
%     \begin{itemize}
%     \item Test with ensemble of topographical conditions
%     \item Experiment with Gaussian features of different sizes
%     \item Focus on surface expressions with meaningful results
%     \item Consider Gausberg domain
%     \item Use Jameson cross-flow features (ranging from shallow to deep ice, sticky to sliding bed)
%     \item Incorporate ICECAP data
%     \end{itemize}
% \end{enumerate}

\wordcount{methodology} words in this section.
\chapter{Resources}\label{resources}
I require high performance computing resources (including compute and storage) from the National Computing Infrastructure (NCI). These resources are already available via a Flagship between NCI and the Monash-led Australian Research Council project Securing Antarctica’s Environmental Future (SAEF).

\section*{Currently available data and Framework}\label{data}
The project will make use of a number of new remote sensing datasets and software tools.

\begin{enumerate}
% recommend to include 1-2 sentences on each of these that say how you will use them / how they come into your study
    \item\textbf{Reference Elevation Model of Antarctica (REMA)}\\
   Constructed from hundreds of thousands of Digital Elevation Models (DEMs) derived from high-resolution Maxar satellite imagery, REMA is calibrated with Cryosat-2 and ICESat altimetry\cite{REMA}.

    \item\textbf{ITS\_LIVE Antarctic surface velocities and elevation}\\
    The NASA-administered ITS\_LIVE website provides automated, high-resolution datasets of Antarctic surface velocities and ice surface elevation change, derived from satellite observations~\cite{itslive}.

    \item\textbf{BedMAP}\\ 
    A suite of gridded products describing surface elevation, ice-thickness and the seafloor and subglacial bed elevation of Antarctica, based on a compilation of data collected by a large number of researchers using a variety of techniques, with the aim of representing a snap-shot of understanding of the Antarctic region\cite{Fretwell_2013}. BEDMAP lacks detailed information on bedrock type, sediment layers, or geothermal heat flux, all of which affect ice dynamics intruducing model uncertainty.

    \item\textbf{BedMachine Antarctica}\\
    A high-resolution map of Antarctic subglacial bed topography that provides unprecedented detail of basal features. The dataset combines multiple ice thickness measurements with mass conservation principles, satellite-derived ice flow velocities, and surface mass balance from regional atmospheric models\cite{Morlighem_2020}. Similarly to BedMAP, BedMachine does not explicitly model basal properties.

    \item\textbf{ICECAP}\\ 
    Collaborative Exploration of the Cryosphere through Airborne Profiling. Since 2012, the project has obtained extensive data on ice thickness mapping and surface elevation in regions of the East Antarctic grounding zone, also comprehensive gravity mapping in areas beneath the Totten Glacier cavity\cite{ICECAP}.

    \item\textbf{Ice-sheet and Sea-level System Model (ISSM)}\\
    ISSM is a finite-element numerical ice sheet model. It has been used to simulate the Antarctic Ice Sheet’s response to various climate scenarios and assess future mass loss contributions to sea level rise\cite{deRydt_2013, Morlighem_2020, ISSM}.
\end{enumerate}

\section*{Data management and archiving}

Data will be published adhering to FAIR principles (Findable, Accessible, Interoperable, Reusable), ensuring transparency and accessibility. The final bed topography datasets will be published at the Australian Antarctic Data Centre (AADC) under an open source licence. All production model outputs will be published with unique DOIs at repositories aligned with the corresponding journal articles. Model outputs will be archived to tape at NCI using existing SAEF resources, as well as backed up to storage available through Monash MASSIVE M3 account aligned with project supervisor Dr McCormack. All journal articles published through this project will be open source, and tier 1 journals will be targeted.

\section*{Risk}

The project is highly feasible and low risk, given that it is a desk-based modelling and data assimilation project. All the data to be used in this project are freely available for download, and project supervisors are experts in ice sheet modelling using ISSM. 

\subsection*{Fieldwork}
Fieldwork is not necessary to achieve the objectives of the project; however, there may be the opportunity to participate in fieldwork through the ICECAP airborne geophysics project (led CI of ICECAP is project supervisor Dr Jason Roberts, Australian Antarctic Division), which will be instrumental in training of geophysical instruments and in developing broader expertise in the field.
\wordcount{resources} words in this section.

% progress/THEORY?
% \chapter{Ice is weird}
A Newtonian fluid (like water) has a constant viscosity regardless of the flow conditions, while a non-Newtonian fluid's viscosity changes based on factors like strain-rate. The viscosity of ice is not constant.depends on how much it is deforming (shear rate).
Ice is a slow, shear-thinning fluid. ``shear-thinning" means that the viscosity of ice decreases with increasing strain rate. This means that under more strain forces ice becomes "softer" and flows more easily.``Slow" means that the flow of ice occurs at very low velocities
\begin{equation}
\rho(\vec{u_t} + \vec{u} \cdot \nabla \vec{u}) = 0,
\end{equation}
i.e. the change in flow velocity (time dependent and convective) is approximately zero.
This assumption greatly simplifies the Navier-Stokes equations for ice flow.
The equations we use to are the incompressibility condition 
\begin{equation}
\nabla \vec{u} = 0,
\end{equation}
i.e. the divergence of the velocity field is zero. The force balance equation
\begin{equation}
-\nabla p + \nabla \cdot \tau_{ij} + \rho g, = 0
\end{equation}
i.e. the pressure gradient, the divergence of the stress tensor (viscous forces within the ice) and the gravitational body force acting on the ice all cancel out, dnd finally Glen's flow law
\begin{equation}
D_{ij} = A\tau^{n} \tau_{ij},
\end{equation}
where the strain rate tensor $D_{ij}$, which describes how fast the ice is deforming is proportional to the deviatoric stress tensor $\tau_{ij}$ and it's magnitude $\tau$, which accounts for the stress caused by deformation (as opposed to isotropic stress like pressure). The flow law exponent $n$ determines how strongly the flow rate depends on stress. For ice, Glen's law uses $n=3$, which implies a nonlinear relationship between stress and strain rate, meaning the flow rate accelerates rapidly with increased stress\cite{modelling_ppt}.

This model does not have time derivatives anymore, this means that a time-stepping ice sheet program recomputes the full velocity field at every time step and does not require velocity information from the previous time step.


\chapter{Shallow Ice Approximation}
(Bons et al 2018)
Glen's law exponent n can range from 1 to 5???woah!?


low driving stresses make SIA fail?



% \section{Ice flow over bedrock perturbations}
% This 1970's rheology paper by Budd explains how bedrock irregularities beneath a glacier affect the surface shape of the ice mass. Budd develops a mathematical model to describe the flow of ice over these undulations, considering the ice as a viscous fluid that deforms under stress. The model predicts that the surface shape of the glacier will mirror the bedrock undulations, but shifted out of phase by approximately $\frac{\pi}{2}$ radians. The paper analyzes the damping of different wavelengths of bedrock undulations, finding that waves with a length roughly three times the ice thickness are minimally damped, while shorter or longer waves are significantly damped out. Finally, the implications of this theory proposes the potential for ice to flow uphill, concluding that bedrock undulations with wavelengths several times the ice thickness are most important in controlling ice motion.

% % \section{Temperate Ice Sliding: An Empirical Study}
% % % Static and dynamic friction are different.
% % % Limiting static shear stress: is the tipping point where pressure overcomes friction and sliding begins. The interface of ice on bedrock has very high static friction.

% % % how does the ice respond to different surfaces (smooth, rough, different material compositions)?
% % % how does the sliding speed varies from one surface to another?
% % % Rough surfaces requires more force to overcome static friction.

% % % Glaciers carve away landscapes via friction and the landscape makes the ice slide.

% % The main objectives described in\cite{Budd_Keage_Blundy_1979} are to describe the relationship between forces and ice movement (sliding) over different surfaces, and how the moving ice affects the surfaces over which they slide.

% % \begin{enumerate}
% % \item How did the researchers apply normal and shear stresses to the ice blocks in their experiments?

% % \item Describe the relationship between limiting static shear stress and normal load observed in the experiments.

% % \item How did sliding velocity vary with shear stress and normal stress at low normal loads? % The experiments showed a direct proportionality between limiting static shear stress and normal load, indicating a constant coefficient of limiting friction specific to each slab.

% % \item What was the significance of the product (TmVb) in the experiments?
% % % The product TmVb, representing the transition point from steady-state sliding to acceleration, tended towards a constant value, suggesting a critical threshold for dynamic instability.

% % \item How did the relationship between sliding velocity and shear stress change at high normal loads?
% % % At low normal loads, sliding velocity increased proportionally with shear stress and decreased proportionally with both normal load and surface roughness.
% % % "The sliding speed at high stresses is not linear, velocity increases with the cube of the shear stress*???* small force increase can lead to a lot of speed!
% % % At high normal loads, the relationship shifted from linear to cubic, with sliding velocity increasing proportionally to the cube of shear stress and inversely proportionally to normal stress.


% % \item What effect does an increase in the water table have on the effective normal stress acting on the glacier base?
% % % An increase in the water table elevates the basal water pressure, which counteracts the normal stress from the overlying ice, effectively reducing the effective normal stress acting on the glacier base.

% % \item Explain how the study's findings might help explain the high velocities observed in fast-outlet polar glaciers.
% % % The study demonstrated that sliding velocity is highly sensitive to effective normal stress. For fast-outlet polar glaciers, where the base is often below sea level, the buoyancy effect of seawater can significantly reduce the effective normal stress, potentially leading to higher sliding velocities.

% % \item What was the observed relationship between erosion and the experimental parameters (normal stress, shear stress, and velocity)?
% % % Erosion was found to increase with higher normal stress, shear stress, and velocity, suggesting a combined effect of these parameters on the rate of material removal from the slab surface.

% % \item Why did the researchers conclude that ice deformation, rather than regelation, plays a dominant role in the observed sliding behaviour?
% % \end{enumerate}% The observed cubic relationship between sliding velocity and shear stress at high normal loads pointed towards ice deformation, rather than regelation, as the dominant mechanism governing sliding behaviour at the experimental scales.
    

% % Essay Questions
% % \begin{enumerate}
% % \item Discuss the limitations of the experimental setup used in the study and how these limitations might affect the applicability of the findings to real-world glacier systems.
% % \item Compare and contrast the roles of regelation and ice deformation in glacier sliding, drawing upon the findings of the study to support your arguments.
% % \item Analyze how the study's results contribute to a better understanding of the dynamics of glacier surges, focusing on the factors that lead to the onset and propagation of surge events.
% % \item Explain the concept of effective normal stress in the context of glacier sliding and discuss how variations in basal water pressure can influence glacier flow.
% % \item Critically evaluate the significance of the study's findings for modelling glacier behaviour and predicting future glacier response to climate change.
% % \end{enumerate}
    
    

% \wordcount{icy_stuff} words in this section.
% \include{numerics
% \wordcount{numerics} words in this section.

\begin{landscape}
\section*{Project timeline}
    \vspace{1cm}\hspace{-2.5em}
    \includegraphics[width=0.85\linewidth]{timeline.png}
\end{landscape}

% \include{dynamics}
% % \wordcount{dynamics} words in this section.
% %
% \include{Conclusion}
% % \wordcount{Conclusion} words in this section.
% %
\chapter{Glossary of Key Terms}\label{glossary}
% https://www.antarcticglaciers.org/glaciers-and-climate/numerical-ice-sheet-models/numerical-modelling-glossary/
\begin{enumerate}
\item Temperate ice: Ice at or near its pressure-melting point.
\item Sliding: The movement of a glacier over its bed by sliding rather than internal deformation.
\item Basal sliding: Sliding occurring at the base of a glacier.
\item Normal stress (N):The force acting perpendicular to a surface, per unit area. In the context of glaciers, it is primarily the weight of the overlying ice.
\item Shear stress (T): The force acting parallel to a surface, per unit area. In the context of glaciers, it is the force driving glacier motion.
\item Limiting static shear stress (TS): The minimum shear stress required to initiate sliding from a resting position.
\item Coefficient of static friction (µs): The ratio of the limiting static shear stress to the normal stress, indicating the resistance to sliding from rest.
\item Steady-state velocity (Vb): The constant velocity reached by a glacier or ice block when the driving shear stress is balanced by resisting forces.
\item Limiting dynamic shear stress (Tm): The shear stress at which a glacier or ice block transitions from steady-state sliding to accelerated sliding.
\item Coefficient of sliding friction (µ): The ratio of the shear stress to the normal stress during steady-state sliding.
\item Regelation: The process of melting under pressure and refreezing at lower pressure, potentially contributing to ice sliding.
\item Asperity: A protrusion or bump on a surface.RoughnessThe unevenness of a surface, characterized by the size and distribution of asperities.
\item Shallow Ice Approximation (SIA): A simplified ice flow model that considers only vertical shear stresses and neglects horizontal stress gradients. Suitable for slow-moving ice in the interior of ice sheets.
\item Shallow Shelf Approximation (SSA): A simplified ice flow model that neglects vertical shear stresses and assumes depth-independent horizontal velocity. Appropriate for modelling floating ice shelves and fast-flowing ice streams
\item Blatter-Pattyn Approximation (BP): A higher-order ice flow model that incorporates longitudinal stresses, making it suitable for simulating fast-flowing ice streams and regions with significant vertical shear.
\item Full-Stokes (FS): The most comprehensive and computationally expensive ice flow model, accounting for all stress components. Essential for accurately simulating ice flow near grounding lines.
\item Ice Sheet System Model (ISSM): A finite element, thermomechanical ice flow model that incorporates SIA, SSA, BP, and FS formulations to simulate ice sheet behaviour at various complexities and spatial resolutions. 
\item Finite Element Method (FEM): A numerical method for solving partial differential equations by dividing the domain into smaller elements and approximating the solution within each element.
\item Adaptive Mesh Refinement: A technique used to refine the mesh in regions of high variability or complexity, enhancing model accuracy and efficiency.
\item Data Assimilation: The process of incorporating observational data into a model to improve its accuracy and predictive capabilities.
\item Basal Drag Coefficient: A parameter representing the frictional resistance between the ice sheet base and the underlying bedrock.
\item Grounding Line: The boundary where the ice sheet transitions from grounded ice to floating ice (ice shelf).
\item Calving: The process by which icebergs break off from the edge of a glacier or ice sheet
\item InSAR: Interferometric Synthetic Aperture Radar, a remote sensing technique used to measure ice surface velocity.
\end{enumerate}

    WHAT ARE THESE
    \begin{itemize}
    \item{Channel incision}
    \item{Alpine style glaciation}
    \end{itemize}
    \vspace{1cm}
    TOOLS
    {\small \begin{itemize}
    \item{ICECAP aero geophysical programme}
    \item{BEDMAP}
    \end{itemize}
    }
    \vspace{1cm}
    MATHS
    {\small \begin{itemize}
    \item{Lagrangian interpolation}
    \item{natural-neighbour interpolation}
    \end{itemize}
    }




\wordcount{glossary} words in this section.
\appendix
% \include{appendix}
% % \wordcount{appendix} words in this section.

\bibliographystyle{unsrturl_mod}
\bibliography{mybib}
\end{document}


