% latexmk -pvc -pdf
\documentclass[12pt, a4paper, openany]{book}
\usepackage[margin=2.5cm]{geometry}
\usepackage{titling}
\usepackage{titlesec}
\usepackage{amsmath,amsthm,amsfonts,amssymb}
\usepackage{dsfont}
\usepackage{mathtools}
\usepackage{braket}
\usepackage[font=scriptsize,labelfont=bf]{caption}
\usepackage[english]{babel}
\usepackage{float} 
\usepackage{pdflscape} 
\usepackage{cite}

\usepackage[T1]{fontenc}
\usepackage[utf8]{inputenc}

\usepackage{datetime}
\newdateformat{monthyeardate}{%
  \monthname[\THEMONTH], \THEYEAR}

\usepackage{graphicx}
\newenvironment{Figure}
    {\par\medskip\noindent\minipage{\linewidth}}
    {\endminipage\par\medskip}
\usepackage{tabularx}

\newenvironment{abstract}
{\clearpage \thispagestyle{empty} \null \vfill \begin{center} \bfseries \large Abstract \end{center}}
{\vfill \null \clearpage}

\newenvironment{impactstatement}
{\clearpage \thispagestyle{empty} \null \vfill \begin{center} \bfseries \large Impact Statement \end{center}}
{\vfill \null \clearpage}


% Colours:
\usepackage[table]{xcolor} % for setting colors
\definecolor{purple}{RGB}{117,77,226}

\usepackage[bookmarksopen,
  pagebackref,
  pdfpagelayout=TwoPageRight,
  colorlinks=true,
  urlcolor=purple,
  citecolor=purple,
  filecolor=purple,
  linkcolor=purple,
  ]
{hyperref}

\usepackage{listings}
\usepackage{xcolor} % Necessary to define custom colors

\definecolor{softblue}{rgb}{0.3, 0.5, 0.8}
\definecolor{darkergreen}{rgb}{0, 0.5, 0}

\lstset{
  basicstyle=\ttfamily\small, % Set the typeface to typewriter and small size
  breaklines=true, % Enable line breaking
  postbreak=\mbox{\textcolor{red}{$\hookrightarrow$}\space}, % Marker for line breaks
  numbers=left, % Line numbers on left
  numberstyle=\tiny, % Set the size of the numbers
}

\lstdefinestyle{mystyleC}{
    language=C,      
    commentstyle=\color{softblue},
    morecomment=[l]{//},  % Line comment
}

\lstdefinestyle{mystylePython}{
    language=Python,      
    commentstyle=\color{darkergreen},
    morecomment=[l]{//},  % Line comment
}

\titleformat{\chapter}[display]
  {\normalfont\bfseries}{}{0pt}{\Huge}


% Required packages
\usepackage{catchfile}

% Command to execute texcount and capture the word count
\newcommand\wordcount[1]{
    \immediate\write18{texcount #1.tex | grep "Words in text" | cut -d: -f2 > #1.wordcount.tmp}
    \CatchFileDef{\mywordcount}{#1.wordcount.tmp}{}
    \immediate\write18{rm #1.wordcount.tmp} % to remove the temporary file
    \mywordcount % to print the word count
}


\begin{document}

\begin{titlepage}
\begin{center}
    {\Huge BedSAT: Antarctica}\\ [1cm] 
    {\Large Exploring what lies beneath using big data and modelling}\\
    \vspace{5cm}
    {\large Ana Fabela Hinojosa\footnote{ana.fabelahinojosa1@monash.edu}}\\
    \monthyeardate\today\\ [1cm]
    Supervisors:\\
    Dr. Felicity McCormack\\
    Dr Jason Roberts\\
    Dr Richard Jones\\ [2cm]
    Panel:\\
    Dr. Fabio Capitanio \\
    Dr. Andrew Gunn \\
    Dr. Ariaan Purich \\ [3.5cm]
    \includegraphics[scale=0.2]{logos.png}

    \end{center}

\end{titlepage}


%----------------------------------------------------------------------------------------
%   QUOTATION PAGE
%----------------------------------------------------------------------------------------
% \vspace*{0.2\textheight}

% \noindent{``El alacrán clavándose el aguijón, harto de ser un alacrán pero necesitando de su alacranidad para acabar con el alacrán''.}\bigbreak
% % \vspace{-2cm}
% \hfill Julio Cortázar\bigbreak

\begin{impactstatement}\label{impactstatement}
Antarctica's bed topography data currently has uncertainties of hundreds of meters in elevation due to sparse and unevenly distributed radar surveys, significantly limiting our ability to predict ice sheet behavior and sea level rise contributions. Through the BedSAT project, we are developing a novel modelling approach that integrates remote sensing data and airborne derived estimates with mathematical and numerical ice flow models to substantially improve bed topography resolution and accuracy. We aim to derive a continent-wide bed topography dataset and conduct sensitivity analyses of dynamic ice loss to different realisations of topographic roughness through 2300.
Our work will quantify how bed topography uncertainties affect ice mass loss projections. This improved understanding can provide more reliable sea level rise predictions, and enable evidence-based policy decisions for climate adaptation strategies. Our open-source approach and FAIR data principles will ensure these improvements benefit the broader scientific community and support more effective climate change mitigation planning for vulnerable coastal communities.
\end{impactstatement}

% \chapter*{Acknowledgements}

\tableofcontents

% introduction
\chapter{Antarctica's Landscape}\label{why}
\section{Climate Impacts and Global Significance}

The polar regions are losing ice, and their oceans are changing rapidly\cite{O_C_in_changingClimate}. The consequences of this extend to the whole planet and it is crucial for us to understand them to be able to evaluate the costs and benefits of potential mitigation. 

Changes in different kinds of polar ice affect many connected systems. Of particular concern is the accelerating loss of continental ice sheets (glacial ice masses on land) in both Greenland and Antarctica, which has become a major contributor to global sea level rise\cite{O_C_in_changingClimate}. Impacts extend beyond direct ice loss: as fresh water from melting ice sheets is added into the ocean, it increases ocean stratification disrupting global thermohaline circulation\cite{Jacobs_2004}. In addition, cold freshwater can dissolve larger amounts of $\mathrm{CO_2}$ than regular ocean water creating corrosive conditions\cite{O_C_in_changingClimate}.
 
While there is high confidence in current ice loss and retreat observations in many areas, there is more uncertainty about the mechanisms driving these changes and their future progression\cite{Fox-Kemper_2021}. Uncertainty increases in regions with variable bed conditions, where characteristics like ``slipperiness'' and ``roughness'' are difficult to verify via direct observations. Other problematic areas involve the ice sheet's grounding line (GL), the zone that delineates ice grounded on bedrock from ice shelves floating over the ocean. The retreat rate depends crucially on topographical features like pinning points\cite{Fox-Kemper_2021}, which lead to increased buttressing by the ice shelf on the upstream ice sheet. However, major knowledge gaps persist in mapping bed topography across Antarctic ice sheet margins - with over half of all margin areas having insufficient data within 5 km of the grounding zone\cite{RINGS_2022}. Addressing this data gap through both systematic mapping and improved interpolation utilising auxiliary data streams with more complete coverage would significantly improve both our understanding of current ice dynamics and the accuracy of ice-sheet models projecting future changes.

\chapter{Topography of Antarctica}\label{review}

Bed topography is one of the most crucial boundary conditions that influences ice flow and loss from the Antarctic Ice Sheet (AIS)\cite{Morlighem_2020}. Bed topography datasets are typically generated from airborne radar surveys, which are sparse and unevenly distributed across the Antarctic continent (see figure \ref{fig:BedMAP}). Interpolation schemes to ``gap fill'' these sparse datasets yield bed topography estimates that have high uncertainties (i.e. multiple hundreds of metres in elevation uncertainty; Morlighem et al., 2020) which propagate through simulations of AIS evolution under climate change\cite{Castleman_2022}. Given the logistical challenges of accessing large parts of the Antarctic continent, there is a crucial need for alternative approaches that integrate diverse and possibly more spatially complete data streams – including satellite data.
\begin{figure}[H] % Forces the figure exactly HERE
    \includegraphics[scale=0.4]{bedmap.png}
    \caption{Distribution of BedMAP\{1,2,3\} data tracks (Source: bedmap.scar.org).}
    \label{fig:BedMAP}
\end{figure}

\section{Approaches to Bed Topography Reconstruction}

A key objective of this study is to understand how bed topography influences ice dynamics, and the bed topography itself. There are two ways that we can infer information about this relationship: Through forward modelling, with assumptions of the bed conditions; and through inverse modelling that relies on surface observations.
\begin{itemize}
    \item\textbf{Forward models}
    The aim of forward models is to see how bed properties impact ice dynamics. A key example is using a large ensemble of bed topographies to investigate how bed uncertainties impact simulated ice mass loss. In this example geostatistical methods can be used to generate bed topographies that either preserve elevation or texture:    
    \begin{itemize}
            \item\textbf{Geostatistics} Statistical methods specialized for analyzing spatially correlated data. In glaciology, this approach is used to interpolate between sparse measurements and characterise spatial patterns in bed properties, often employing techniques like kriging\cite{Mackie_2020}.

    \end{itemize}

    \item\textbf{Inversion models}
    The aim of these models is to understand bed properties through knowledge of surface or other variables. A key example is the retrieval od bed topography or basal slipperiness from surface elevation and velocities.

        \begin{itemize}
            \item\textbf{Control method inversion}: A variational approach that minimizes mismatches between observed and simulated fields through a cost function approach. Remote sensing data and theoretical ice flow models are used to obtain basal conditions\cite{deRydt_2013}. Often needs regularization terms to prevent non-physical features or over-fitting\cite{Morlighem_Goldberg_2024}.

            \item\textbf{Mass conservation}: Used to constrain inversion models and fill data gaps by employing physical conservation laws, particularly effective for reconstructing bed topography where direct measurements are sparse~\cite{Morlighem_2017, Morlighem_2020}. Requires (contemporary) measurements of ice thickness at the inflow boundary to properly constrain the system\cite{Morlighem_Goldberg_2024}.

            \item\textbf{Markov Chain Monte Carlo (MCMC)}: A probabilistic method that generates sample distributions to quantify uncertainties in ice sheet parameters and models\cite{Morlighem_Goldberg_2024}. While powerful for uncertainty quantification, these methods remain computationally intensive for continental-scale ice sheet models\cite{Morlighem_Goldberg_2024}.

            \item\textbf{4dvar}: Four-dimensional variational data assimilation - Minimizes the difference between model predictions and observations across a time window. Mainly used to optimize model parameters and initial conditions\cite{Morlighem_Goldberg_2024}. Can handle time-varying data and evolving glacier states, making it more suitable for dynamic systems unlike control methods, this makes them more computationally demanding\cite{Morlighem_Goldberg_2024}.

            \item\textbf{EnKF} Ensemble Kalman Filter. A sequential data assimilation method that uses an ensemble of model states to estimate uncertainty and update model parameters based on observations\cite{Morlighem_Goldberg_2024}.
        \end{itemize}
    
\end{itemize} 
This study aims to develop an integrated method combining forward and inverse modeling to improve bed topography estimates by leveraging high-resolution satellite surface data in regions where radar data is sparse. Despite revolutionary advances in satellite technology providing unprecedented surface detail, a key challenge in glaciology remains: fully utilizing this wealth of information where subglacial understanding is limited. Our approach will integrate more comprehensive models with modern computational capabilities, with the specific methodology chosen based on research objectives, data availability, and computational resources.

\newpage
\section{Theoretical Frameworks}
 Understanding how bed features manifest in surface observations requires a theoretical framework that connects these two domains. The modelling approach used on this project relies on two different theoretical frameworks that relate bed topography and surface features. Using observations and these modelling frameworks, my goal is understanding the limitations of each approach and how they can be improved

The first framework was originally developed by Budd \cite{Budd_1970}. This model relates ice flow over bedrock perturbations to surface expressions using a two-dimensional biharmonic stress equation. 

The model's foundation rests on two key simplifications:
\begin{itemize}
    \item Most shear deformation occurs at the base of the ice sheet
    \item Explicit consideration of longitudinal stresses and strain-rates
\end{itemize}

The modelling carried out in\cite{Budd_1970} determined ice-sliding velocities for wide ranges of roughness, normal stress, and shear stress relevant to real glaciers\cite{Budd_1970}. Despite its robustness, Budd's mathematical framework remains notably underutilized in modern ice sheet modeling. 

The second framework in my plan analyses how shape and mechanical properties of the ice bed significantly influence how ice flows, with changes at the bed potentially leading to large differences in predicted ice loss rates\cite{Ockenden_2022}. Recent work by Ockenden et al. demonstrates both the capabilities and limitations of current inversion approaches in addressing this problem.
The core principle of the method by Ockenden et al. (2022) relies on the fact that variations in basal topography, slipperiness, and roughness cause measurable disturbances to the surface flow of the ice. Through linear perturbation analysis, they establish a systematic relationship between surface observations and bed conditions. This relationship can be expressed as $y=f(x)$, where $y$ represents surface measurements (velocity and topography), $x$ represents bed properties (topography and slipperiness), and $f$ is the forward model\cite{Gudmundsson_2008}. The inversion process, $x=f^{-1}(y)$, estimates bed conditions from surface observations.%Could add a sentence here that there is frequently a horizontal offset between the bed and surface expression of features.

The method works best when analyzing perturbations that are small relative to mean properties, under specific conditions including:
\begin{enumerate}
    \item A linear viscous medium ($n=1$)
    \item Non-linear sliding law ($m>0$)
    \item Steady-state conditions
    \item Spatially constant zero-order solutions
\end{enumerate}

Using high-resolution datasets (REMA surface elevation at 8m resolution and NASA ITS\_LIVE velocity at 120m resolution), their approach performs well for:
\begin{itemize}
    \item Areas with moderate topographic gradients
    \item Features not aligned with ice flow direction
    \item Medium-wavelength (5-50km) bedrock features
\end{itemize}

However, significant limitations emerge when:
\begin{itemize}
    \item Dealing with steep topography where the shallow-ice-stream approximation breaks down
    \item Handling variable slipperiness parameters
    \item Attempting to validate slipperiness predictions due to lack of ground-truth data
\end{itemize}


\section{Current Opportunities}

The review of current approaches to Antarctic bed topography reconstruction reveals significant methodological limitations. While theoretically robust, methods such as the inversion technique employed by Ockenden et al. often rely on simplifying assumptions that inadequately capture complex ice-bed interactions, particularly in regions with steep topography where the shallow-ice-stream approximation breaks down.
With BedSAT, I aim to address these limitations by building upon the theoretical foundations established by Budd and recent inversion methods. My primary objective is to better understand how variable bed conditions such as ``slipperiness,'' ``roughness,'' and pinning points influence both GL retreat rates and their surface manifestations. BedSAT will bridge the disconnect between surface observations and bed topography by implementing a more realistic set of rheological and geometric assumptions. The methodology will follow an iterative inversion-forward modeling validation cycle: initially inverted bed topography will be used in ice dynamics (forward) models with BedSAT's improved assumptions, allowing comparison between resulting surface predictions and established datasets like NASA's ITS\_LIVE (see Chapter \ref{resources}). This process will be systematized through machine learning methods, ultimately enhancing the analytical capabilities for the final phase of my project.
 
% \wordcount{literature_review} words in this section.
% planning
\chapter{Objectives and Methodology}

The overall aim of this project is to derive a new Antarctic bed topography using remote sensing data, airborne derived-estimates of the bed and ice sheet modelling. Using the new bed topography to improve understanding of the impact of fine-scale topographic roughness on ice and subglacial hydrological flow, and projections of ice mass loss under climate warming.\\
\\The specific objectives are:
\begin{enumerate}
    \item Develop an ice sheet modelling approach to assimilate satellite remote sensing datasets to improve knowledge of the bed (BedSAT) informed by mathematical models of ice flow over topography;
    \item Derive a new bed topography for Antarctica using BedSAT;
    \item Conduct sensitivity analyses to understand the impact of the improved bed topography on projections of ice mass loss from Antarctica under climate warming.
\end{enumerate}

The first phase of the project (objective 1) is to derive the BedSAT method. This will involve the integration of the Budd~\cite{Budd_1970} mathematical model relating ice surface elevation and bed topography into ISSM, and the development of a methodology for the data assimilation into ISSM. I will use a regional catchment in Antarctica for which relatively more radar data are available and has an indicative range of topography features, e.g. the Aurora Subglacial Basin, East Antarctica, extensively surveyed by the ICECAP project for airborne geophysics~\cite{Young_2011}. The second phase of the project (objective 2) will apply the methodology developed in objective 1 to the whole Antarctic continent, deriving a continent-wide bed topography dataset. Using covariance properties from existing radar surveys, I will generate a number of realisations of bed topography with unique high-resolution, and statistically-consistent topographic roughness. The third phase of the project will use the new bed topography datasets to conduct a sensitivity analysis of ice sheet model projections to 2300 CE, investigating the impact of the new topography and different realisations of roughness on ice and subglacial hydrological flow and ice mass loss from Antarctica.\\

\section*{Plan:}
\subsection*{Objective 1}
\begin{enumerate}
\item Develop a method to interpolate topography that ensures consistent surface expressions with observations
\item Reduce RMS error between observations and model predictions
\end{enumerate}
\subsection*{Key Investigation Areas}
\begin{enumerate}
\item\textbf{Model Development Strategy}
    \begin{itemize}
    \item We will maintain invariant bed traction throughout our modeling timeframe to isolate topographical effects in our inversion approach, with validation through sensitivity tests in regions where bed properties are well-known.
    
    \item Our model will incorporate available thermal distribution, velocity field, and ice thickness data, as these parameters are essential for accurate ice flow representation and can be constrained using radar observations.
    
    \item Through spectral analysis of surface expressions, we will identify the topographical features that most strongly influence surface patterns, using available high-resolution surface elevation data for validation.
    
    \item To account for variations in ice behavior with thickness, we will simulate scenarios ranging from thick ice with slippery base to thin ice with sticky base, using observed velocity patterns as constraints.
    \end{itemize}

\item\textbf{Transfer Functions}
    \begin{itemize}
    \item We will develop efficient transfer function methods for rapid bed topography inversion, validating against known bed configurations from radar data.
    
    \item Our transfer functions will be tested across various ice thickness and flow conditions, with validation against different glacial systems.
    
    \item Cross-validation against radargrams will provide direct verification of our inversion results, including uncertainty quantification through comparison with measured bed elevations.
    
    \item Spatial covariance analysis of existing radar data will inform our statistical framework and error propagation through the inversion process.
    
    \item We will account for friction roughness and high-amplitude variations in our analysis, using observed surface velocity patterns as constraints.
    \end{itemize}

\item\textbf{Model Validation}
    \begin{itemize}
    \item We will apply quantitative error reduction metrics and compare systematically against existing bed topography products.
    
    \item Model limitations and breaking points will be identified through systematic testing across extreme scenarios, constrained by physical principles.
    
    \item Singular Value Decomposition (SVD) analysis will help identify key modes of variability in our solutions.
    
    \item Grid independence testing will ensure solution robustness across different spatial resolutions.
    
    \item Sensitivity analysis will examine the impact of our model assumptions, particularly regarding basal conditions and ice rheology.
    \end{itemize}
\end{enumerate}


\subsection*{Confounding Factors to Consider}

Basal friction at the ice-bed interface plays a crucial role in how bed topography is expressed at the surface. Understanding how different factors affect basal friction is essential for accurately interpreting surface expressions and inverting them to determine bed topography:

\begin{itemize}
    \item Sliding behavior: The relationship between basal stress and sliding velocity affects how ice flows over the bed. Areas with enhanced sliding can mask bed features in surface expressions, while areas with stronger friction tend to show more pronounced surface expressions of bed topography.
    
    \item Rheological properties: Ice viscosity varies with temperature and stress state, affecting how efficiently bed topographic signals propagate to the surface. Softer ice tends to dampen bed topography signals more than stiffer ice.
    
    \item Thermomechanical responses: The temperature-dependent nature of ice deformation means that warmer, softer ice near the bed behaves differently from colder, stiffer ice above. This vertical variation in ice properties affects how bed topography signals are transmitted to the surface.
    
    \item Slippery spots: Localized areas of reduced friction, often due to the presence of water or deformable sediments, can create surface expressions that might be misinterpreted as bed topography features.
\end{itemize}

These factors directly impact our ability to invert surface data for bed topography, as they can either enhance or mask the relationship between bed features and their surface expressions. Our methodology must account for these effects to avoid misinterpreting surface features.


% \section*{Plan:}
% \subsection*{Objective 1}
% \begin{enumerate}
% \item Develop a method to interpolate topography that ensures consistent surface expressions with observations
% \item Reduce RMS error between observations and model predictions
% \end{enumerate}

% \subsection*{Key Investigation Areas}
% % this is great, but perhaps make each point a sentence, and include: why it's relevant to your work, citation, how you might constrain it
% \begin{enumerate}
% \item\textbf{Model Development Strategy}
%     \begin{itemize}
%     \item Focus on invariant bed traction over the time-frame
%     \item Consider thermal distribution, velocity field, and ice thickness
%     \item Perform analysis to identify which topographical feartures affecting surface expression amplitude and positions
%     \item Note: Ice behavior varies with thickness. Simulate the following different scenarios: (thick ice → slippery base, thin ice → sticky base)
%     \end{itemize}
% \item\textbf{Transfer Functions}
%     \begin{itemize}
%     \item Develop rapid transfer function construction methods
%     \item Use for various conditions and domains
%     \item Validate against ICECAP cross-section radargrams
%     \item Analyze spatial covariance of existing data
%     \item Consider friction roughness and high amplitude variations
%     \end{itemize}
% \item\textbf{Model Validation}
%     \begin{itemize}
%     \item Quantify error reduction
%     \item Identify model breaking points
%     \item Use SVD (Singular Value Decomposition)
%     \item Test grid independence
%     \item Evaluate sensitivity to assumptions
%     \end{itemize}
% \end{enumerate}


% \subsection*{Confounding Factors to Consider}
% % this needs some more description - recommend writing a short paragraph describing how these each influence basal friction, and what that means for what we might expect in terms of surface expressions, and how that relates to your aims of generating bed topography

% Basal friction coefficient effects on:
% \begin{itemize}
%     \item Sliding behavior
%     \item Rheological properties
%     \item Thermomechanical responses (stiff vs soft ice)
%     \item Slippery spots
%     \end{itemize}

% \subsection*{Workflow Plan}
% \begin{enumerate}
% \item\textbf{Initial Modeling Phase}
%     \begin{itemize}
%     \item Exclude surface elevation initially (reserve for control model)
%     \item Use 2D Gaussian with $3\times$ ice thickness
%     \item Analyze REMA spectral components at various frequencies
%     \item Determine reasonable Signal-to-Noise ratio levels
%     \end{itemize}

% \item\textbf{Inversion Development}
%     Parameters to consider:
%     \begin{itemize}
%         \item Basal traction
%         \item Internal temperature distribution
%         \item Heat flux
%         \item Additional rheological parameters
%     \end{itemize}

% \item\textbf{Model Testing}
%     \begin{itemize}
%     \item Test with ensemble of topographical conditions
%     \item Experiment with Gaussian features of different sizes
%     \item Focus on surface expressions with meaningful results
%     \item Consider Gausberg domain
%     \item Use Jameson cross-flow features (ranging from shallow to deep ice, sticky to sliding bed)
%     \item Incorporate ICECAP data
%     \end{itemize}
% \end{enumerate}

% \wordcount{methodology} words in this section.
\chapter{Resources}\label{resources}
I require high performance computing resources (including compute and storage) from the National Computing Infrastructure (NCI). These resources are already available via a Flagship between NCI and the Monash-led Australian Research Council project Securing Antarctica’s Environmental Future (SAEF).

\section*{Currently available data and Framework}\label{data}
The project will make use of a number of new remote sensing datasets and software tools.

\begin{enumerate}
% recommend to include 1-2 sentences on each of these that say how you will use them / how they come into your study
    \item\textbf{Reference Elevation Model of Antarctica (REMA)}\\
   Constructed from hundreds of thousands of Digital Elevation Models (DEMs) derived from high-resolution Maxar satellite imagery, REMA is calibrated with Cryosat-2 and ICESat altimetry\cite{REMA}.

    \item\textbf{ITS\_LIVE Antarctic surface velocities and elevation}\\
    The NASA-administered ITS\_LIVE website provides automated, high-resolution datasets of Antarctic surface velocities and ice surface elevation change, derived from satellite observations~\cite{itslive}.

    \item\textbf{BedMAP}\\ 
    A suite of gridded products describing surface elevation, ice-thickness and the seafloor and subglacial bed elevation of Antarctica, based on a compilation of data collected by a large number of researchers using a variety of techniques, with the aim of representing a snap-shot of understanding of the Antarctic region\cite{Fretwell_2013}. BEDMAP lacks detailed information on bedrock type, sediment layers, or geothermal heat flux, all of which affect ice dynamics intruducing model uncertainty.

    \item\textbf{BedMachine Antarctica}\\
    A high-resolution map of Antarctic subglacial bed topography that provides unprecedented detail of basal features. The dataset combines multiple ice thickness measurements with mass conservation principles, satellite-derived ice flow velocities, and surface mass balance from regional atmospheric models\cite{Morlighem_2020}. Similarly to BedMAP, BedMachine does not explicitly model basal properties.

    \item\textbf{ICECAP}\\ 
    Collaborative Exploration of the Cryosphere through Airborne Profiling. Since 2012, the project has obtained extensive data on ice thickness mapping and surface elevation in regions of the East Antarctic grounding zone, also comprehensive gravity mapping in areas beneath the Totten Glacier cavity\cite{ICECAP}.

    \item\textbf{Ice-sheet and Sea-level System Model (ISSM)}\\
    ISSM is a finite-element numerical ice sheet model. It has been used to simulate the Antarctic Ice Sheet’s response to various climate scenarios and assess future mass loss contributions to sea level rise\cite{deRydt_2013, Morlighem_2020, ISSM}.
\end{enumerate}

\section*{Data management and archiving}

Data will be published adhering to FAIR principles (Findable, Accessible, Interoperable, Reusable), ensuring transparency and accessibility. The final bed topography datasets will be published at the Australian Antarctic Data Centre (AADC) under an open source licence. All production model outputs will be published with unique DOIs at repositories aligned with the corresponding journal articles. Model outputs will be archived to tape at NCI using existing SAEF resources, as well as backed up to storage available through Monash MASSIVE M3 account aligned with project supervisor Dr McCormack. All journal articles published through this project will be open source, and tier 1 journals will be targeted.

\section*{Risk}

The project is highly feasible and low risk, given that it is a desk-based modelling and data assimilation project. All the data to be used in this project are freely available for download, and project supervisors are experts in ice sheet modelling using ISSM. 

\subsection*{Fieldwork}
Fieldwork is not necessary to achieve the objectives of the project; however, there may be the opportunity to participate in fieldwork through the ICECAP airborne geophysics project (led CI of ICECAP is project supervisor Dr Jason Roberts, Australian Antarctic Division), which will be instrumental in training of geophysical instruments and in developing broader expertise in the field.
\wordcount{resources} words in this section.
\chapter{Progress}

Budd's sliding theory examines how stress propagates through an ice mass as it flows over an undulating bed. When ice encounters a bedrock bump, it creates a complex stress field that propagates upward through the ice column at an angle rather than vertically, (see figures \ref{fig:P} for the pressure and \ref{fig:Vx} horizontal velocity fields). The steady-state surface shape consists of waves similar to the bedrock perturbations but out of phase by approximately $\pi$. As a direct mathematical consequence, ``the steepest slope occurs over the highest bedrock''\cite{Budd_1970} not the surface undulation itself. To verify Budd's sliding law, this simulation implements a flowband model across a 10 km domain with 100 m resolution. The simulation features an 800 m mean ice thickness over bedrock at 1 km elevation, with a -0.1 radians downward slope in the x-direction and imposed cosine undulations (2.64 km wavelength, 0.1 km amplitude). The main model, executes a 300-year transient Full-Stokes simulation using 1-year time steps. The current implementation demonstrates the stress propagation through ice flowing over spatially varying basal friction according to a simplified version of Budd's sliding law (see figure \ref{fig:Friction}) and validates the non-linear rheology (n = 4.0) that I use.


% surface undulations phase-shifted by $\pi$/2
\begin{figure}[H]
    \includegraphics[scale=0.5]{basal_friction.png}
    \caption{Slope parallel visualisation of the computational mesh with basal nodes highlighted. The gray dotted lines represent the complete finite element mesh with multiple vertical layers conforming to the undulating geometry. Basal nodes (yellow to purple) follow the periodic undulated bed topography. The colour gradient along the basal nodes display the imposed variations in basal friction coefficient implemented through a simplified version of Budd's sliding law, with lighter colours indicating regions of higher basal drag es per Budd's sliding theory.}
    \label{fig:Friction}
\end{figure}


\begin{figure}
    \includegraphics[scale=0.45]{Pressure_300yrs_xz.png}
    \caption{Pressure field distribution at t = 300.00 years shown in original coordinates. The color scale (ranging from -2000 to 8000 Pa) reveals the spatially distributed pressure variations throughout the ice body, with higher pressures (yellow) concentrated before the peaks of the bed undulations. The triangular mesh elements display the finite element discretisation used for solving the Stokes equations. The development of low-pressure zones near the surface aligning with the undulating basal topography suggests stress transfer from the imposed variation in basal friction. Note: The white gaps visible within the mesh represent masked areas excluded from visualization processing, not physical features or data gaps in the original simulation.}
    \label{fig:P}
\end{figure}

\begin{figure}
    \includegraphics[scale=0.45]{Vx_300yrs_xz.png}
    \caption{Horizontal velocity field ($V_x$) at t = 300.00 years displayed in original coordinates. The color scale indicates velocity magnitude ($10^{-12}$ km/s, equivalent to $\approx 31.5$ mm/year at the maximum) with flow direction from left to right. The velocity pattern shows clear acceleration as the ice flows downslope, with highest velocities (yellow) occurring near the terminus. The fixed upstream boundary condition  constrains flow at the inlet, while the progressive acceleration downstream results from gravitational forcing along the sloped bed. Note: The white gaps visible within the mesh represent masked areas excluded from visualization processing, not physical features or data gaps in the original simulation.}
    \label{fig:Vx}
\end{figure}

\begin{figure}
    \includegraphics[scale=0.5]{xcorr.png}
    \caption{Cross-correlation between the base and surface slope signals across spatial lags. The maximum correlation corresponds to the lag distance of 0.7 km, which translates to a phase shift of 0.53$\pi$ (95.5 degrees), only 0.03$\pi$ radians away from Budd's theoretical value.}
    \label{fig:xcorr_filtered}
\end{figure}

\begin{figure}
    \includegraphics[scale=0.5]{direct.png}
    \caption{Visual comparison of topographic profile of bedrock compared to the surface slope signal in slope-parallel coordinates. Here it is clearly visible that the surface slope (orange) is shifted to the right of the peak in bed elevation (blue), with approximately the predicted phase shift. This result is strong evidence that my simple model is correctly capturing the physics described in Budd's paper. The surface slope is responding to bed undulations with the expected phase relationship - maximum surface slopes occur slightly downstream of the highest bed elevations.}
    \label{fig:direct}
\end{figure}













\wordcount{progress} words in this section.

% \chapter{Numerics}
\begin{Figure}
\includegraphics[width=0.9\linewidth]{steps.png}
\captionof{figure}{TODO}
\label{fig:first_sims}
\end{Figure}

\begin{Figure}
\includegraphics[width=0.9\linewidth]{numerical_modelling_of_glaciers.png}
\captionof{figure}{figure taken from\cite{modelling_ppt}}
\label{fig:parameters}
\end{Figure}

\section{ISSM: Continental-Scale Ice Sheet Modelling}
The Ice Sheet System Model (ISSM) was developed and is used for simulating ice sheet flow at continental scales. ISSM, a finite element, thermomechanical model, incorporates high-order stresses and high spatial resolution capabilities\cite{ISSM}. The Larour et al. (2012) ISSM paper discusses the different ice flow models within the software, including the full-Stokes, Blatter-Pattyn, Shallow-Shelf, and Shallow Ice Approximations, highlighting their individual strengths and limitations. It also explores numerical methods employed, such as static adaptive mesh refinement and inverse methods for parameter estimation. Finally, the study showcases the application of ISSM to the Greenland Ice Sheet, demonstrating its capacity to model the ice flow velocity with high accuracy, using data assimilation techniques to infer the basal drag coefficient.


% Study Guide
% Glossary of Key Terms

%     Shallow Ice Approximation (SIA): A simplified ice flow model that considers only vertical shear stresses and neglects horizontal stress gradients. Suitable for slow-moving ice in the interior of ice sheets.
%     Shallow Shelf Approximation (SSA): A simplified ice flow model that neglects vertical shear stresses and assumes depth-independent horizontal velocity. Appropriate for modeling floating ice shelves and fast-flowing ice streams.
%     Blatter-Pattyn Approximation (BP): A higher-order ice flow model that incorporates longitudinal stresses, making it suitable for simulating fast-flowing ice streams and regions with significant vertical shear.
%     Full-Stokes (FS): The most comprehensive and computationally expensive ice flow model, accounting for all stress components. Essential for accurately simulating ice flow near grounding lines.
%     Ice Sheet System Model (ISSM): A finite element, thermomechanical ice flow model that incorporates SIA, SSA, BP, and FS formulations to simulate ice sheet behaviour at various complexities and spatial resolutions.
%     Finite Element Method (FEM): A numerical method for solving partial differential equations by dividing the domain into smaller elements and approximating the solution within each element.
%     Adaptive Mesh Refinement: A technique used to refine the mesh in regions of high variability or complexity, enhancing model accuracy and efficiency.
%     Data Assimilation: The process of incorporating observational data into a model to improve its accuracy and predictive capabilities.
%     Basal Drag Coefficient: A parameter representing the frictional resistance between the ice sheet base and the underlying bedrock.
%     Grounding Line: The boundary where the ice sheet transitions from grounded ice to floating ice (ice shelf).
%     Calving: The process by which icebergs break off from the edge of a glacier or ice sheet.
%     InSAR: Interferometric Synthetic Aperture Radar, a remote sensing technique used to measure ice surface velocity.

% Quiz

% Instructions: Answer the following questions in 2-3 sentences.

%     What are the limitations of the Shallow Ice Approximation (SIA) and Shallow Shelf Approximation (SSA) in ice sheet modelling?
%     Why is the Full-Stokes (FS) model considered the most accurate representation of ice flow, and what makes its application challenging?
%     How does the Ice Sheet System Model (ISSM) integrate different ice flow approximations?
%     What is the importance of anisotropic adaptive mesh refinement in ISSM?
%     Describe the role of data assimilation in improving the accuracy of ice sheet models.
%     Why is the basal drag coefficient a crucial parameter in ice flow modeling?
%     Explain the concept of grounding line dynamics and its significance.
%     What is the process of calving, and why is it relevant to ice sheet mass balance?
%     How does ISSM handle the thermal regime of ice sheets?
%     What are some key challenges and future directions in ice sheet modeling using ISSM?

% Quiz Answer Key

%     SIA neglects horizontal stress gradients and is unsuitable for fast flow, while SSA ignores vertical shear, limiting accuracy near grounding lines.
%     FS accounts for all stress components, making it highly accurate, but its computational expense poses a challenge for large-scale applications.
%     ISSM offers SIA, SSA, BP, and FS formulations, allowing for different levels of complexity and computational efficiency.
%     It concentrates elements in dynamic regions like fast-flowing outlets, optimizing computational resources while preserving accuracy.
%     Data assimilation incorporates observations (e.g., surface velocity) to constrain model parameters and improve realism.
%     It governs basal friction, a key factor influencing ice flow velocity and the response to changes in temperature or basal conditions.
%     Grounding line dynamics refer to the movement of the grounding line, impacting ice discharge and contributing to sea level rise.
%     Calving is the breaking off of icebergs, a major process of mass loss from ice sheets, influencing their size and contribution to sea level.
%     ISSM includes a thermal model with heat conduction, advection, and a penalty scheme to ensure the temperature stays below the pressure melting point.
%     Challenges include improving grounding line dynamics, incorporating calving laws, and increasing spatial resolution, requiring more efficient numerical techniques and computational power.

% Essay Questions

%     Compare and contrast the four ice flow approximations (SIA, SSA, BP, and FS) implemented in ISSM, discussing their strengths, limitations, and appropriate applications.
%     Explain the process of data assimilation in ISSM, focusing on the inversion for the basal drag coefficient. Discuss the challenges and benefits of this approach.
%     Discuss the importance of accurate representation of grounding line dynamics in ice sheet models. What are the limitations of the current implementation in ISSM, and how can they be addressed in future development?
%     Describe the role of calving in ice sheet mass balance, and discuss the need to incorporate realistic calving laws in ice sheet models like ISSM.
%     Evaluate the potential of ISSM as a tool for projecting ice sheet contribution to future sea level rise. What are the key uncertainties and areas where the model can be improved?
% \wordcount{numerics} words in this section.
\begin{landscape}
\section*{Project timeline}
    \vspace{1cm}\hspace{-2.5em}
    \includegraphics[width=0.85\linewidth]{timeline.png}
\end{landscape}

% \include{dynamics}
% % \wordcount{dynamics} words in this section.
% %
% \include{Conclusion}
% % \wordcount{Conclusion} words in this section.
% %
\chapter{Glossary of Key Terms}\label{glossary}
% https://www.antarcticglaciers.org/glaciers-and-climate/numerical-ice-sheet-models/numerical-modelling-glossary/
\begin{enumerate}
\item Temperate ice: Ice at or near its pressure-melting point.
\item Sliding: The movement of a glacier over its bed by sliding rather than internal deformation.
\item Basal sliding: Sliding occurring at the base of a glacier.
\item Normal stress (N):The force acting perpendicular to a surface, per unit area. In the context of glaciers, it is primarily the weight of the overlying ice.
\item Shear stress (T): The force acting parallel to a surface, per unit area. In the context of glaciers, it is the force driving glacier motion.
\item Limiting static shear stress (TS): The minimum shear stress required to initiate sliding from a resting position.
\item Coefficient of static friction (µs): The ratio of the limiting static shear stress to the normal stress, indicating the resistance to sliding from rest.
\item Steady-state velocity (Vb): The constant velocity reached by a glacier or ice block when the driving shear stress is balanced by resisting forces.
\item Limiting dynamic shear stress (Tm): The shear stress at which a glacier or ice block transitions from steady-state sliding to accelerated sliding.
\item Coefficient of sliding friction (µ): The ratio of the shear stress to the normal stress during steady-state sliding.
\item Regelation: The process of melting under pressure and refreezing at lower pressure, potentially contributing to ice sliding.
\item Asperity: A protrusion or bump on a surface.RoughnessThe unevenness of a surface, characterized by the size and distribution of asperities.
\item Shallow Ice Approximation (SIA): A simplified ice flow model that considers only vertical shear stresses and neglects horizontal stress gradients. Suitable for slow-moving ice in the interior of ice sheets.
\item Shallow Shelf Approximation (SSA): A simplified ice flow model that neglects vertical shear stresses and assumes depth-independent horizontal velocity. Appropriate for modelling floating ice shelves and fast-flowing ice streams
\item Blatter-Pattyn Approximation (BP): A higher-order ice flow model that incorporates longitudinal stresses, making it suitable for simulating fast-flowing ice streams and regions with significant vertical shear.
\item Full-Stokes (FS): The most comprehensive and computationally expensive ice flow model, accounting for all stress components. Essential for accurately simulating ice flow near grounding lines.
\item Ice Sheet System Model (ISSM): A finite element, thermomechanical ice flow model that incorporates SIA, SSA, BP, and FS formulations to simulate ice sheet behaviour at various complexities and spatial resolutions. 
\item Finite Element Method (FEM): A numerical method for solving partial differential equations by dividing the domain into smaller elements and approximating the solution within each element.
\item Adaptive Mesh Refinement: A technique used to refine the mesh in regions of high variability or complexity, enhancing model accuracy and efficiency.
\item Data Assimilation: The process of incorporating observational data into a model to improve its accuracy and predictive capabilities.
\item Basal Drag Coefficient: A parameter representing the frictional resistance between the ice sheet base and the underlying bedrock.
\item Grounding Line: The boundary where the ice sheet transitions from grounded ice to floating ice (ice shelf).
\item Calving: The process by which icebergs break off from the edge of a glacier or ice sheet
\item InSAR: Interferometric Synthetic Aperture Radar, a remote sensing technique used to measure ice surface velocity.
\end{enumerate}

    WHAT ARE THESE
    \begin{itemize}
    \item{Channel incision}
    \item{Alpine style glaciation}
    \end{itemize}
    \vspace{1cm}
    TOOLS
    {\small \begin{itemize}
    \item{ICECAP aero geophysical programme}
    \item{BEDMAP}
    \end{itemize}
    }
    \vspace{1cm}
    MATHS
    {\small \begin{itemize}
    \item{Lagrangian interpolation}
    \item{natural-neighbour interpolation}
    \end{itemize}
    }




% \include{appendix}
% % \wordcount{appendix} words in this section.

\bibliographystyle{unsrturl_mod}
\bibliography{mybib}
\end{document}


