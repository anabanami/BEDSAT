\chapter{Progress}

The current objective is understanding the relationship between surface and bed topography in glaciers using ice-sheet modeling with ISSM. This simulation examines how bed undulations manifest at the surface. The simulation implements a flowband ice model across a 10 km domain with 100 m resolution. Featuring an 800 m mean ice thickness over bedrock at 1 km elevation, with a -0.1 radians downward slope and imposed cosine undulations (2.64 km wavelength, 0.1 km amplitude). The main objective of this simulation is to verify the sliding law proposed in\cite{Budd_1970}. The main model, \texttt{flowline8.py}, executes a 300-year transient Full-Stokes simulation using 1-year time steps. Supporting tools include \texttt{phase\_analysis.py} for examining bed-surface relationships and \texttt{plotting.py} for visualizing resulting simulation fields. The implementation of this simulation does not yet demonstrates how basal topography influences surface expression through a spatially varying basal friction governed by a simplified version of Budd's sliding law.

% surface undulations phase-shifted by $\pi$/2
\begin{figure}[H]
    \includegraphics[scale=0.5]{basal_friction.png}
    \caption{Slope parallel visualisation of the computational mesh with basal nodes highlighted. The gray dotted lines represent the complete finite element mesh with multiple vertical layers conforming to the undulating geometry. Basal nodes (yellow to purple) follow the periodic bed topography with a wavelength matching the dominant frequency observed in the filtered signals. The colour gradient along the basal nodes represent variations in basal friction coefficient implemented through a simplified version of Budd's sliding law, with lighter colours indicating regions of higher basal drag}
    \label{fig:Friction}
\end{figure}

\begin{figure}
    \includegraphics[scale=0.45]{overlap_xz.png}
    \caption{Mesh geometry evolution from initial state (black mesh) to final configuration at t = 300.00 years (blue mesh) shown in original coordinates. The unstructured triangular finite element mesh adapts to the changing ice geometry while maintaining numerical stability. Presently both the surface and basal boundaries preserve the wavelength of the underlying bed topography ($\lambda = 2.64$ km), with similar amplitudes and phase relationships. This is not the expected result. The final mesh appears to be moved near the base when compared to the initial mesh setup which is unphysical. This demonstrates that there may be a problem in the transient computation.}
    \label{fig:}
\end{figure}

\begin{figure}
    \includegraphics[scale=0.45]{Pressure_300yrs_xz.png}
    \caption{Pressure field distribution at t = 300.00 years shown in original coordinates. The color scale (ranging from -2000 to 8000 Pa) reveals the spatial pressure variations throughout the ice body, with higher pressures (yellow) concentrated before the peaks of the bed undulations. The triangular mesh elements display the finite element discretisation used for solving the Stokes equations. The development of low-pressure zones near the surface aligning with the undulating basal topography suggests stress transfer from the imposed variation in basal friction.}
    \label{fig:}
\end{figure}

\begin{figure}
    \includegraphics[scale=0.45]{Vx_300yrs_xz.png}
    \caption{Horizontal velocity field (Vx) at t = 300.00 years displayed in original coordinates. The color scale indicates velocity magnitude ($10^{-12}$ km/s, equivalent to $\approx 31.5$ mm/year at the maximum) with flow direction from left to right. The velocity pattern shows clear acceleration as the ice flows downslope, with highest velocities (yellow) occurring near the terminus. The fixed upstream boundary condition  constrains flow at the inlet, while the progressive acceleration downstream results from gravitational forcing along the sloped bed.}
    \label{fig:Vx}
\end{figure}

\begin{figure}
    \includegraphics[scale=0.5]{xcorr_filtered.png}
    \caption{Cross-correlation between the filtered base and filtered surface signals across spatial lags. Maximum correlation of approximately 98\% occurs at zero lag, with periodic oscillations of decreasing amplitude at increasing distances. The symmetric pattern indicates similar signal structures in both datasets, with a characteristic wavelength of approximately 2.5 km (bedrock signal is $\lambda = 2.64$ km) between correlation peaks.}
    \label{fig:xcorr_filtered}
\end{figure}

\begin{figure}
    \includegraphics[scale=0.5]{base_surf_overlap.png}
    \caption{Topographic profiles and filtered signals in slope-parallel coordinates. Upper panel: Raw elevation profiles showing surface (blue) and base (orange) interfaces with periodic undulations along a 10 km domain. Lower panel: Filtered signals ($\lambda = 2.64$km) highlighting wavelength-specific components of surface and base topography after removing the mean trend.}
    \label{fig:overlap}
\end{figure}
