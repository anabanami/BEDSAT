\chapter{Methodology}

The first phase of the project (objective 1) is to derive the BedSAT method. This will involve the integration of the Budd~\cite{Budd_1970} mathematical model relating ice surface elevation and bed topography into ISSM, and the development of a methodology for the data assimilation into ISSM. I will use a regional catchment in Antarctica for which relatively more radar data are available and has an indicative range of topography features, e.g. the Aurora Subglacial Basin, East Antarctica, extensively surveyed by the ICECAP project for airborne geophysics~\cite{Young_2011}. The second phase of the project (objective 2) will apply the methodology developed in objective 1 to the whole Antarctic continent, deriving a continent-wide bed topography dataset. Using covariance properties from existing radar surveys, I will generate a number of realisations of bed topography with unique high-resolution, and statistically-consistent topographic roughness. The third phase of the project will use the new bed topography datasets to conduct a sensitivity analysis of ice sheet model projections to 2300 CE, investigating the impact of the new topography and different realisations of roughness on ice and subglacial hydrological flow and ice mass loss from Antarctica.\\

\section*{Plan:}
\subsection*{Objective 1}
\begin{enumerate}
\item Develop a method to interpolate topography that ensures consistent surface expressions with observations
\item Reduce RMS error between observations and model predictions
\end{enumerate}

\subsection*{Key Investigation Areas}
% this is great, but perhaps make each point a sentence, and include: why it's relevant to your work, citation, how you might constrain it
\begin{enumerate}
\item\textbf{Model Development Strategy}
    \begin{itemize}
    \item Focus on invariant bed traction over the time-frame
    \item Consider thermal distribution, velocity field, and ice thickness
    \item Perform analysis to identify which topographical feartures affecting surface expression amplitude and positions
    \item Note: Ice behavior varies with thickness. Simulate the following different scenarios: (thick ice → slippery base, thin ice → sticky base)
    \end{itemize}
\item\textbf{Transfer Functions}
    \begin{itemize}
    \item Develop rapid transfer function construction methods
    \item Use for various conditions and domains
    \item Validate against ICECAP cross-section radargrams
    \item Analyze spatial covariance of existing data
    \item Consider friction roughness and high amplitude variations
    \end{itemize}
\item\textbf{Model Validation}
    \begin{itemize}
    \item Quantify error reduction
    \item Identify model breaking points
    \item Use SVD (Singular Value Decomposition)
    \item Test grid independence
    \item Evaluate sensitivity to assumptions
    \end{itemize}
\end{enumerate}


\subsection*{Confounding Factors to Consider}
% this needs some more description - recommend writing a short paragraph describing how these each influence basal friction, and what that means for what we might expect in terms of surface expressions, and how that relates to your aims of generating bed topography

Basal friction coefficient effects on:
\begin{itemize}
    \item Sliding behavior
    \item Rheological properties
    \item Thermomechanical responses (stiff vs soft ice)
    \item Slippery spots
    \end{itemize}


\subsection*{Workflow Plan}

\begin{enumerate}
\item\textbf{Initial Modeling Phase}
    \begin{itemize}
    \item Exclude surface elevation initially (reserve for control model)
    \item Use 2D Gaussian with $3\times$ ice thickness
    \item Analyze REMA spectral components at various frequencies
    \item Determine reasonable Signal-to-Noise ratio levels
    \end{itemize}

\item\textbf{Inversion Development}
    Parameters to consider:
    \begin{itemize}
        \item Basal traction
        \item Internal temperature distribution
        \item Heat flux
        \item Additional rheological parameters
    \end{itemize}

\item\textbf{Model Testing}
    \begin{itemize}
    \item Test with ensemble of topographical conditions
    \item Experiment with Gaussian features of different sizes
    \item Focus on surface expressions with meaningful results
    \item Consider Gausberg domain
    \item Use Jameson cross-flow features (ranging from shallow to deep ice, sticky to sliding bed)
    \item Incorporate ICECAP data
    \end{itemize}
\end{enumerate}
