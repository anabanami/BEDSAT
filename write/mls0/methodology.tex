\chapter{Methods}
\section{Objectives}

The overall aim of this project is to derive a new Antarctic bed topography using remote sensing data, airborne derived-estimates of the bed and ice sheet modelling. Using the new bed topography to improve understanding of the impact of fine-scale topographic roughness on ice and subglacial hydrological flow, and projections of ice mass loss under climate warming.\\
\\The specific objectives are:
\begin{enumerate}
    \item Develop an ice sheet modelling approach to assimilate satellite remote sensing datasets to improve knowledge of the bed (BedSAT) informed by mathematical models of ice flow over topography;
    \item Derive a new bed topography for Antarctica using BedSAT;
    \item Conduct sensitivity analyses to understand the impact of the improved bed topography on projections of ice mass loss from Antarctica under climate warming.
\end{enumerate}

\section{Research plan methodology}

Objective 1 is to derive the BedSAT method. We will use a regional catchment in Antarctica for which relatively more radar data are available and has an indicative range of topography features, e.g. the Aurora Subglacial Basin, East Antarctica, extensively surveyed by the Collaborative Exploration of the Cryosphere through Airborne Profiling (ICECAP)~\cite{Young_2011}. In developing our inversion approach, We will consider important factors like sliding variability, to distinguish between topographical signatures and sliding-induced features in surface expressions. We will implement a realistic sliding-law in ISSM that incorporates Budd's  determinations of ice-sliding velocities across varying roughness, normal stress, and shear stress conditions. By calibrating these advanced models against known radar-surveyed regions with varying sliding conditions and validating with other observational data, we can better isolate true bed topographical features from friction-related artifacts and create a more robust inversion framework.

The second phase of the project (objective 2) will apply the methodology developed in objective 1 to the whole Antarctic continent, deriving a continent-wide bed topography dataset. Using covariance properties from existing radar surveys, we will generate a number of realisations of bed topography with unique high-resolution, and statistically-consistent topographic roughness. The third phase of the project will use the new bed topography datasets to conduct a sensitivity analysis of ice sheet model projections to 2300 CE, investigating the impact of the new topography and different realisations of roughness on ice and subglacial hydrological flow and ice mass loss from Antarctica.


\subsection*{Specific outline for Objective 1}
\begin{enumerate}
\item Develop a method to interpolate topography that ensures surface expressions consistent with observations
\item Reduce RMS error between observations and model predictions
\end{enumerate}

\subsection*{Key Investigation Areas}
\begin{enumerate}
\item\textbf{Model Development Strategy}
    \begin{itemize}
    \item We will maintain invariant (but spatially variable )bed traction throughout our modeling timeframe to isolate topographical effects in our inversion approach.
    
    \item Our model will incorporate available thermal distribution, velocity field, and ice thickness data, as these parameters be constrained using radar observations.
    
    \item Through spectral analysis of surface expressions, we will identify the topographical features that most strongly influence surface patterns.
    
    \item To account for variations in ice behavior with thickness, we will simulate scenarios ranging from thick ice with slippery base to thin ice with sticky base, using observed velocity patterns as constraints.
    \end{itemize}

\item\textbf{Transfer Functions}
    \begin{itemize}
    \item We will develop efficient transfer function methods for rapid bed topography inversion, validating against known bed configurations from radar data.
    
    \item Our transfer functions will be tested across various ice thickness and flow conditions.
    
    \item Cross-validation against radargrams will provide direct verification of our inversion results.
    
    \item Spatial covariance analysis of existing radar data will inform our statistical framework and error propagation through the inversion process.
    
    \item We will account for friction roughness and high-amplitude variations, using observed surface velocity patterns as constraints.
    \end{itemize}

\item\textbf{Model Validation}
    \begin{itemize}
    \item We will apply quantitative error reduction metrics and compare systematically against existing bed topography products.
    
    \item Model limitations and breaking points will be identified through systematic testing across extreme scenarios, constrained by physical principles.
    
    \item Grid independence testing will ensure solution robustness across different spatial resolutions.
    
    \item Sensitivity analysis will examine the impact of our model assumptions.
    \end{itemize}
\end{enumerate}




