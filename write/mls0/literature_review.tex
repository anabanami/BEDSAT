\chapter{Antarctica's Landscape}\label{why}
\section*{Climate Impacts and Global Significance}

The polar regions are losing ice, and their oceans are changing rapidly\cite{O_C_in_changingClimate}. The consequences of this extend to the whole planet and it is crucial for us to understand them to be able to evaluate the costs and benefits of potential mitigation. 

The consequences of changes in different kinds of polar ice manifest across multiple interconnected systems. Of particular concern is the accelerating loss of continental ice sheets (permanent glacial ice masses on land) in both Greenland and Antarctica, which has become a major contributor to global sea level rise\cite{O_C_in_changingClimate}. Impacts extend beyond direct ice loss: As fresh water from melting ice sheets is added into the ocean, it increases ocean stratification. Cold freshwater can dissolve larger amounts of $\mathrm{CO_2}$ than regular ocean water. Increased $\mathrm{CO_2}$ uptake by polar oceans is creating corrosive conditions for calcifying organisms\cite{O_C_in_changingClimate}. In addition, freshwater stratification threatens to disrupt global thermohaline circulation\cite{Jacobs_2004} by decreasing the natural mixing of the ocean layers.
 
While there is high confidence in current ice loss and retreat observations in many areas, there is more uncertainty about the mechanisms driving these changes and their future progression\cite{Fox-Kemper_2021}. Uncertainty increases in regions with variable bed conditions, where characteristics like "slipperiness" and roughness are difficult to verify via direct samples. Other problematic areas involve the ice sheet's grounding line (GL). The retreat rate depends crucially on topographical features like pinning points, as these are locations where the GL is most stable and ice-sheet retreat will slow\cite{Fox-Kemper_2021}. However, major knowledge gaps persist in mapping bed topography across Antarctic ice sheet margins - with over half of all margin areas having insufficient data within 5 km of the grounding zone\cite{RINGS_2022}. Addressing this data gap through systematic mapping efforts would significantly improve both our understanding of current ice dynamics and the accuracy of ice-sheet models projecting future changes.

\chapter{Topography of Antarctica}\label{review}

% overall, i think it'd be worth thinknig about how you integrate/relate all of this information back to the overall thesis objectives. 

% the information that you include here should be sufficient for the reader to understand the problem, what previous studies have done, where the gap is, and naturally lead to what you're planning to address in your thesis. 

% hence, when you include specific case studies, e.g., ockenden study, you want to make sure that readers understand why you're focussing on that paper in particular. how does it relate to your objectives? what did they find / how was it successful? what are the big gaps that you might tackle in this project?

Bed topography is one of the most crucial boundary conditions that influences ice flow and loss from the Antarctic Ice Sheet (AIS)\cite{Morlighem_2020}. Bed topography datasets are typically generated from airborne radar surveys, which are sparse and unevenly distributed across the Antarctic continent (see figure \ref{fig:BedMAP}). Interpolation schemes to "gap fill" these sparse datasets yield bed topography estimates that have high uncertainties (i.e. multiple hundreds of metres in elevation uncertainty; Morlighem et al., 2020) which propagate in simulations of AIS evolution under climate change\cite{Castleman_2022}. Given the logistical challenges of accessing large parts of the Antarctic continent, there is a crucial need for alternative approaches, that integrate diverse and possibly more spatially complete data streams – including satellite data.
\begin{figure}[H] % Forces the figure exactly HERE
    \includegraphics[scale=0.4]{bedmap.png}
    \caption{Distribution of BedMAP\{1,2,3\} data tracks (Source: bedmap.scar.org).}
    \label{fig:BedMAP}
\end{figure}

Obtaining information about the conditions at the base of the ice-sheet often relies on indirect modelling methods like
\begin{itemize}

    \item\textbf{Inversions}
    Retrieval (or update) of bed topography and basal slipperiness from surface topography and velocity measurements\cite{deRydt_2013}. See section \ref{Ockenden_2022} for an outlined example using \textbf{linear perturbation analysis}.

        \begin{itemize}
            \item\textbf{Control method inversion}: A variational approach that minimizes mismatches between observed and simulated fields through a cost function approach, using remote sensing data and theoretical ice flow models to obtain basal conditions\cite{deRydt_2013}. Often needs regularization terms to prevent non-physical features or over-fitting\cite{Morlighem_Goldberg_2024}.

            \item\textbf{Mass conservation}: Used to constrain inversion models and fill data gaps by employing physical conservation laws, particularly effective for reconstructing bed topography where direct measurements are sparse~\cite{Morlighem_2017, Morlighem_2020}. Requires (contemporary) measurements of ice thickness at the inflow boundary to properly constrain the system\cite{Morlighem_Goldberg_2024}.

            \item\textbf{Markov Chain Monte Carlo (MCMC)}: A probabilistic method that generates sample distributions to quantify uncertainties in ice sheet parameters and models\cite{Morlighem_Goldberg_2024}. While powerful for uncertainty quantification, these methods remain computationally intensive for continental-scale ice sheet models\cite{Morlighem_Goldberg_2024}.

        \end{itemize}

    \item\textbf{4dvar}: Four-dimensional variational data assimilation - Minimizes the difference between model predictions and observations across a time window. Mainly used to optimize model parameters and initial conditions\cite{Morlighem_Goldberg_2024}. Can handle time-varying data and evolving glacier states, making it more suitable for dynamic systems unlike control methods, this makes them more computationally demanding\cite{Morlighem_Goldberg_2024}.

    \item\textbf{Geostatistics} Statistical methods specialized for analyzing spatially correlated data. In glaciology is used to interpolate between sparse measurements and characterise spatial patterns in bed properties, often employing techniques like kriging\cite{Mackie_2020}.

    \item\textbf{EnKF} Ensemble Kalman Filter. A sequential data assimilation method that uses an ensemble of model states to estimate uncertainty and update model parameters based on observations\cite{Morlighem_Goldberg_2024}.
    
\end{itemize} 
Many authors have considered the influence of the bed topography on the ice surface profile features. Understanding this relationship is crucial for accurate ice sheet bed reconstruction\cite{Budd_1970}. While each method has its strengths and limitations, the choice of approach often depends on the specific research objectives, data availability, and computational resources. Control methods and mass conservation approaches are widely used for steady-state reconstructions, while 4dvar and EnKF methods are better suited for time-evolving systems. Recent advances in computational power have made probabilistic methods like MCMC more feasible, offering valuable insights into parameter uncertainties.

\newpage
\section*{Ice Sheet Bed Reconstruction via Surface Data Inversion}\label{Ockenden_2022}

Features in the ice bed often show up as subtle patterns in the ice surface topography above them\cite{Ockenden_2022}. These bed conditions - both their shape and mechanical properties - significantly influence how ice flows. changes at the bed potentially leading to large differences in predicted ice loss rates\cite{Ockenden_2022}. To address this challenge, it is common to use inversion methods to estimate the geophysical conditions at the ice sheet bed. An example is outlined below.

\subsection*{Theoretical Framework}

Inversion is based on the principle that variations in basal topography, slipperiness, and roughness cause disturbances to the surface flow of the ice. By measuring these disturbances in surface velocity and topography, and using equations like the shelfy-stream approximation (SSA) to relate those disturbances back to their source, we can estimate the basal conditions.
The indirect estimate of basal property $x$ and surface property $y$ are related through $y=f(x)$, where $f$ is referred to as the forward model, this makes  $x=f^{-1}(y)$ the inversion. Property $y$ can be measurements of velocity and topography along the upper surface of a glacier, while the quantity $x$ to be estimated represents basal topography and basal slipperiness\cite{Gudmundsson_2008}.
The inversion method developed by Ockenden et al. in\cite{Ockenden_2022} introduces perturbations to study how small changes in ice thickness ($h$), surface elevation ($s$), basal topography ($b$), and ice velocity ($u$) affect ice flow. This means that the method is most accurate when applied to perturbations, small relative to the mean of each studied property (based on a reference state). Ockenden et al. explain in\cite{Ockenden_2022} that the shallow-ice-stream equations can be linearised and solved analytically when assuming:
\begin{enumerate}
\item A linear viscous medium ($n=1$)
\item Non-linear sliding law ($m>0$)
\item Steady-state conditions
\item Spatially constant zero-order solutions
\end{enumerate}

The SSA system described in Ockenden et al.\cite{Ockenden_2022} is as follows:
\begin{equation}\partial_{x} (4 h \eta \partial_{x} u + 2 h \eta \partial_y v) + \partial_{y}(h \eta( \partial_{x} v + \partial_{y} u)) - (u/c^{1/m}) = \rho g h ( \partial_{x} (s) \mathrm{cos}(\alpha) - \mathrm{sin}(\alpha))
\end{equation}\label{eq:2.1}\\
\begin{equation}\partial_{y} (4 h \eta \partial_{y} v + 2 h \eta \partial_x u) + \partial_{x}(h \eta( \partial_{y} u + \partial_{x} v)) - (v/c^{1/m}) = \rho g h ( \partial_{y} (s) \mathrm{cos}(\alpha)
\end{equation}\label{eq:2.2}

The linearised system described by Equations \ref{eq:2.1} and \ref{eq:2.2} is based around a reference model $(\bar{h}, \bar{s}, \bar{b}, \bar{u}, \bar{v}, \bar{c})$, leading to first-order momentum balance equations:
\begin{equation}
4 \eta \bar{h} \partial_{xx} \Delta u + 3 \eta \bar{h} \partial_{xy}^{2} \Delta v + \eta \bar{h} \partial_{yy}^{2}\Delta u -\gamma \Delta u  = \rho g \bar{h}\mathrm{cos}(\alpha) \partial_x \Delta s - \rho g \mathrm{sin}(\alpha)\Delta h
\end{equation}\\
\begin{equation}
4 \eta \bar{h} \partial_{yy} \Delta v + 3 \eta \bar{h} \partial_{xy}^{2} \Delta u + \eta \bar{h} \partial_{xx}^{2}\Delta v -\gamma \Delta v  = \rho g \bar{h}\mathrm{cos}(\alpha) \partial_y \Delta s
\end{equation}

where $h$ represents ice thickness, $s$ surface elevation, $(u, v)$ horizontal components of surface velocity, $c$ basal slipperiness, $\eta$ effective ice viscosity, $m$ sliding law parameter, $\rho$ ice density, $\alpha$ mean surface slope in $x$-direction (zero mean slope in $y$-direction), and $g$ acceleration due to gravity.

A disadvantage of the  assumptions above is that they can limit how well we can model the behavior of real ice which exhibits nonlinear rheology.

\subsection*{Transfer Functions and Implementation}

The methodology in \cite{Ockenden_2022} employs transfer functions. Transfer functions are mathematical expressions that describe how perturbations in basal topography and slipperiness affect surface topography and velocity. They are derived from the SSA equations
$$\begin{bmatrix}
\hat{s} \\
\hat{u} \\
\hat{v}
\end{bmatrix} =\begin{bmatrix}
T_{sb} & T_{sc} \\
T_{ub} & T_{uc} \\
T_{vb} & T_{vc}
\end{bmatrix}
\begin{bmatrix}
\hat{b}\\
\hat{c}
\end{bmatrix}$$
The system is solved using a weighted least-squares approach, minimizing:
\begin{equation}
\Sigma s(s_{\mathrm{obs}} - s_{\mathrm{pred}})^2 + \Sigma u(u_{\mathrm{obs}} - u_{\mathrm{pred}})^2 + \Sigma v(v_{\mathrm{obs}} - v_{\mathrm{pred}})^2
\end{equation}

Inversion combines the surface data with the transfer functions to estimate the bed properties that would cause the surface changes.

\subsection*{Application and Limitations}

Ockenden et al. successfully implemented this inversion method using two key datasets: REMA surface elevation data (8m resolution) and NASA ITS\_LIVE velocity data (120m resolution). The method demonstrates particularly strong performance in areas with moderate topographic gradients, with features not aligned with ice flow direction, and medium-wavelength (5-50km) bedrock features. However, the inversion encounters significant limitations when dealing with steep topography where the shallow-ice-stream approximation breaks down, variable slipperiness parameters, and suffers from a lack of validation data for slipperiness predictions.

Modern satellite technology has revolutionized our understanding of ice sheets by providing unprecedented detail of surface features through high-resolution elevation models and velocity measurements, as exemplified by the advanced remote sensing observations utilized in Ockenden et al.'s work. While this represents a significant advance in our ability to reconstruct bed conditions using surface data, we have yet to fully harness this wealth of information to improve bed topography estimates, particularly in regions where radar estimates are sparse. The central challenge now lies in developing systematic methods to link these detailed surface observations with underlying bed conditions - a critical step in bridging the gap between our rich surface datasets and our more limited understanding of subglacial topography.

\section*{A Forward Model: relating the bed to the surface}

The model developed by Budd in \cite{Budd_1970} relates ice flow over bedrock perturbations to surface expressions using a two-dimensional biharmonic stress equation. Budd's model simplifies the stress balance within the ice by assuming that most of the shear deformation happens at the base. 

% HOW REALISTIC IS THIS ASSUMPTION?

It also explicitely considers longitudinal stresses and strain-rates. All these assumptions are similar to those of the SSA approach in Ockenden et al. While not directly using SSA equations, the concepts and relationships described by Budd in\cite{Budd_1970} emphasize how the stress and strain components are influenced by bed topography which in turn has a strong effect on surface features and ice flow.

Similarly to Ockenden et al., Budd's model goes beyond a simple relationship between surface-slope variation and perturbation size, which is a limitation of simplified SSA models.

Interestingly, Budd's mathematical framework, despite being around for over five decades, is yet to be put to use in modern ice sheet modeling. This is especially surprising now that we have the computational power and high-resolution satellite data required for the broad scale application of the method. In this project, we aim to integrate this framework into ice sheet models using the Ice-sheet and Sea-level System Model (ISSM), 


potentially creating a systematic way to link what we can see on the surface observations to what's happening at the bed - especially in areas of sparse bed measurements.