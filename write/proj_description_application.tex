% I suggest starting from the brief I sent you of the project and put some dot points against each of the above sections. Then send it to me and I can help guide the outline (before you spend too much time on it!!). Then we can iterate back and forth. This is definitely not solely up to you to prepare and I'm happy to help with the framing, and with more background info, and to help finesse the writing. Most of the references in the brief I sent you would be a good place to start, but let me know if you want any other specific information.

% Compile with: xelatex proj_description_application.tex
\documentclass[10pt,a4paper]{article}
\usepackage{doi}
\usepackage{url}
\usepackage{amsmath}
\usepackage[no-math]{fontspec}
\usepackage{geometry}
\usepackage{setspace}
\usepackage{xcolor}
\usepackage{xspace}
\usepackage{hyperref}
\usepackage{geometry}
\usepackage{pdflscape}
\usepackage{pgfgantt}

\definecolor{icebergblue}{RGB}{102,102,255}
\definecolor{pastelblue}{RGB}{153,204,255}
% Set your name here
\def\name{Ana Cristina Fabela Hinojosa}

% The following metadata will show up in the PDF properties
\hypersetup{
  colorlinks = true,
  citecolor = icebergblue,
  urlcolor = icebergblue,
  pdfauthor = {\name},
  pdftitle = {\name: project description},
  pdfsubject = {project description},
  pdfpagemode = UseNone
}

% Customize page headers
\pagestyle{myheadings}
\thispagestyle{empty}

\begin{document}
\begin{center}
    {\Large BedSAT Antarctica: exploring what lies beneath using big data and modelling}\\ [1cm]
\end{center}
{\textbf{Background}}\\
Antarctica has been losing ice mass over the last few decades and is likely to be a significant contributor to sea level rise over the 21st Century. However, there is significant uncertainty in the timing and magnitude of Antarctica's ice loss, largely due to unknowns in ice sheet properties and associated flow processes~\cite{IPCC}. Bed topography is one of the most crucial boundary conditions that influences ice flow and loss from the Antarctic Ice Sheet (AIS)~\cite{DeepGlacialTroughs}. Bed topography datasets are typically generated from airborne radar surveys, which are sparse and unevenly distributed across the Antarctic continent,. Interpolation schemes to "gap fill" these sparse datasets yield bed topography estimates that have high uncertainties (i.e. multiple hundreds of metres uncertainty; Morlighem et al., 2020), which propagate in simulations of AIS evolution under climate change~\cite{ReductionOfUncertaintyThwaites}. Given the logistical challenges of accessing large parts of the Antarctic continent, there is a crucial need for alternative approaches that integrate diverse data streams – including satellite data – to derive bed topography.
Two commonly used alternative approaches for deriving bed topography make use of the mass conservation method (MC) and geostatistics. The MC method is used to fill data gaps by taking advantage of the fundamental physical laws of conservation of mass and momentum~\cite{DeepGlacialTroughs}. MC makes use of satellite-derived estimates of surface velocity and the adjoint method to calculate ice thickness, fields that are dynamically consistent with the ice flow physics. Corrections are made to satellite-derived ice surface elevation datasets to account for variations in ice surface layer densities (e.g. firn), and then ice surface elevation fields are used with the ice thickness data to obtain the bed topography estimates. MC is effective in regions where the ice flow is greater than $\sim$~50 m/year -- i.e. close to the coast -- which means that MC is not able to be applied for most (>80\%) of the Antarctic continent. Furthermore, the precision of the MC product is affected by the spacing between ice thickness measurements, which are used to constrain the calculation, as well as by uncertainties and errors in the ice velocity and the surface mass balance. Nevertheless, this methodology has been successfully applied in both Antarctica and Greenland, transforming our knowledge of bed topography, glacier dynamics, ocean circulation, ocean heat transfer, calving dynamics, and mechanisms of retreat~\cite{DeepGlacialTroughs, BedMachinev3}.
Geostatistical approaches have also been used to improve bed topography representation.
These methods have been used to “gap-fill” sparse bed topography data, refining the bed topography resolution or generating statistically-realistic bed roughness~\cite{GreenlandGeostatistics}. However, they typically rely on high-resolution radar-derived measurements of the bed topography, and so, like MC, tend to be most effective close to the coast, and less effective in the interior of the continental ice sheets. A drawback of MC methods is often large interpolation errors~\cite{ATR}, especially in regions with complex subglacial terrain or varying physical properties.

In this project, we propose to derive a new method to estimate bed topography for the Antarctic Ice Sheet – BedSAT. BedSAT makes use of the idea that for certain length scales, the ice surface elevation reflects the bed topography~\cite{SurfaceOndulations}. The relationship between the ice surface and the bed topography has been described mathematically~\cite{Budd_1970}, and can be integrated into ice sheet models, so that high resolution ice surface elevation datasets can be assimilated into ice sheet models to derive bed topography.\\
\\\textbf{Aims and Objectives}

The overall aim of this project is to derive a new Antarctic bed topography using remote sensing data and ice sheet modelling, and use the new bed topography in ice sheet modelling to improve understanding of the impact of fine-scale topographic roughness on ice and subglacial hydrological flow, and projections of ice mass loss under climate warming.\\
\\The specific objectives are:
\begin{enumerate}
    \item Develop an ice sheet modelling approach to assimilate satellite remote sensing datasets to improve knowledge of the bed (BedSAT) informed by mathematical models of ice flow over topography;
    \item Derive a new bed topography for Antarctica using BedSAT;
    \item Conduct sensitivity analyses to understand the impact of the improved bed topography on projections of ice mass loss from Antarctica under climate warming.
\end{enumerate}
\textbf{Approach and methodology}

The project will make use of a number of new remote sensing datasets, namely the Reference Elevation Model of Antarctica (REMA), ice surface velocities from NASA’s ITS\_LIVE, and the state-of-the-art Ice-sheet and Sea-level System Model (ISSM).

The first phase of the project (objective 1) is to derive the BedSAT method. This will involve the integration of the Budd~\cite{Budd_1970} mathematical model relating ice surface elevation and bed topography into ISSM, and the development of a methodology for the data assimilation into ISSM. I will use a regional catchment in Antarctica for which relatively more radar data are available, e.g. the Aurora Subglacial Basin, East Antarctica, whose margins have been extensively surveyed by the ICECAP project for airborne geophysics~\cite{Young2011}. The second phase of the project (objective 2) will apply the methodology developed in objective 1 to the whole Antarctic continent, deriving a continent-wide bed topography dataset. Using covariance properties from existing radar surveys, I will generate a number of realisations of bed topography with unique high-resolution, and statistically-consistent topographic roughness. The third phase of the project will use the new bed topography datasets to conduct a sensitivity analysis of ice sheet model projections to 2300 CE, investigating the impact of the new topography and different realisations of roughness on ice and subglacial hydrological flow and ice mass loss from Antarctica.\\

The datasets to be used are described in more detail below.
\begin{itemize}
    \item\textbf{Reference Elevation Model of Antarctica (REMA)}\\
    REMA provides a high-resolution (2-metre) terrain map of nearly the entire continent, allowing for precise measurements of elevation changes over time. REMA supports various remote sensing activities, such as image orthorectification and interferometry, and aids in geodynamic and ice flow modeling, grounding line mapping, and surface process studies. Constructed from hundreds of thousands of Digital Elevation Models (DEMs) derived from high-resolution Maxar satellite imagery (including WorldView and GeoEye data), REMA is calibrated with Cryosat-2 and ICESat altimetry, ensuring high accuracy with uncertainties of less than 1 meter over most areas~\cite{REMA}.

    \item\textbf{ITS\_LIVE Antarctic surface velocities and elevation}\\
    The NASA-administered ITS\_LIVE website provides automated, high-resolution datasets of Antarctic surface velocities and ice surface elevation change, derived from satellite observations. The datasets are available on annual timesteps from 1985 to present. ITS\_LIVE employs various statistical and computational methods to process data from satellites including Landsat and Sentinel, ensuring precise and timely updates for scientific research~\cite{itslive}.

    \item\textbf{Ice-sheet and Sea-level System Model (ISSM)}~\cite{ISSM}\\
    ISSM is a finite-element numerical ice sheet model. It has been used to simulate the Antarctic Ice Sheet’s response to various climate scenarios and assess future mass loss contributions to sea level rise [9, 10]. The mesh can be refined to better capture variations in ice flow and driving stresses, enhancing the simulation’s accuracy of surface elevation changes and ice dynamics. This project will involve numerical modeling using advanced mathematical approaches, including the Blatter-Pattyn approximation to the full Stokes equations for ice flow (i.e. conservation of momentum equations). The Blatter-Pattyn approximation strikes a balance between the computationally intensive full Stokes equations and the simpler shallow ice approximation (SIA), retaining vertical shearing and longitudinal stress gradients. This makes it ideal for modeling the dynamics of fast-flowing ice streams and ice shelves at the continental scale, enhancing simulation accuracy while being computationally feasible. Additionally, data assimilation, machine learning, and geostatistics will be employed, with the full Stokes equations used if necessary.
    \end{itemize}
To conduct the ice sheet model projections to 2300 CE, a similar approach to the Ice Sheet Model Intercomparison Project phase 6 (ISMIP6)~\cite{ISMIP6} will be used. Ocean and atmosphere forcing datasets are derived from the Coupled Model Intercomparison Project Phase 5 (CMIP5), and using RCP 2.6 and RCP 8.5 emissions scenarios.\\
\\\textbf{Timeline and feasibility}\\
\textit{Risk}\\
The project is highly feasible and low risk, given that it is a desk-based modelling and data assimilation project. All the data to be used in this project are freely available for download, and project supervisors are experts in ice sheet modelling using ISSM.\\
\\\textit{Resources}\\
The project will require high performance computing resources (including compute and storage) from the National Computing Infrastructure (NCI). We anticipate requiring \~250 k Service Units (SU) each quarter, and up to 500 TB of storage. These resources are already available via a Flagship between NCI and the Monash-led Australian Research Council project Securing Antarctica’s Environmental Future (SAEF).\\
\\\textit{Data management and archiving}\\
Data will be published adhering to FAIR principles (Findable, Accessible, Interoperable, Reusable), ensuring transparency and accessibility. The final bed topography datasets will be published at the Australian Antarctic Data Centre (AADC) under an open source licence. All production model outputs will be published with unique DOIs at repositories aligned with the corresponding journal articles. Model outputs – including production and other outputs – will be archived to tape at NCI using existing SAEF resources, as well as backed up to storage available through Monash MASSIVE M3 account aligned with project supervisor Dr McCormack. All journal articles published through this project will be open source, and tier 1 journals will be targeted.\\
\\\textit{Fieldwork}\\
Fieldwork is not necessary to achieve the objectives of the project; however, there may be the opportunity to participate in fieldwork through the ICECAP airborne geophysics project (led CI of ICECAP is project supervisor Dr Jason Roberts, Australian Antarctic Division), which will be instrumental in training of geophysical instruments and in developing broader expertise in the field. \\
\\\textit{Conferences}\\
At least one conference will be attended each year. An international conference relevant to the discipline, e.g. the European Geophysical Union General Assembly, will be attended in the final year of the project.
\begin{landscape}
\textbf{Project Timeline (3.5 Years)}\\
\begin{figure}[h!]
\hspace{-11em}
\vspace{-10cm}
\begin{ganttchart}[
    vgrid,
    hgrid,
    title height=0.4,
    title label anchor/.append style={below=-1.6ex},
    title left shift=.05,
    title right shift=-.05,
    title/.style={fill=none},
    bar/.style={fill=blue!60},
    bar height=0.4,
    group right shift=0,
    group top shift=0,
    group height=.1,
    group peaks width={0.1},
    progress label text={},
    bar label font=\normalsize\bfseries,
    group label font=\bfseries
]{1}{42}
% \gantttitle{}{36} \\
\gantttitlelist{1,...,42}{1}\\

\ganttgroup{Year 1}{1}{12}
\ganttbar{Literature Review}{1}{6} \\
\ganttmilestone{6-Month Milestone}{6} \\
\ganttbar{Develop BedSAT methods}{7}{12} \\
\ganttmilestone{Confirmation of Candidature}{12} \\
\ganttbar{Training (60 hours)}{6}{12} \\

\ganttgroup{Year 2}{13}{24}
\ganttbar{Training (60 hours)}{13}{18} \\
\ganttbar{Process Assessment}{19}{24} \\
\ganttmilestone{EGU Conference (April)}{16} \\
\ganttmilestone{AMOS Conference}{20} \\

\ganttgroup{Year 3}{25}{36}
\ganttbar{Model analyses/projections}{25}{30} \\
\ganttbar{Present Findings}{31}{36} \\
\ganttmilestone{AMOS Conference}{32} \\
\ganttmilestone{EGU Conference (April)}{28} \\
\ganttmilestone{Finalise Analyses}{36} \\

\ganttgroup{Year 3.5}{37}{42}
\ganttbar{Publications}{37}{42}

\end{ganttchart}
\end{figure}
\end{landscape}

\bibliographystyle{unsrturl_mod}
\bibliography{mybib}
\end{document}




