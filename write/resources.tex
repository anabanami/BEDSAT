\chapter*{Resources}

The project will require high performance computing resources (including compute and storage) from the National Computing Infrastructure (NCI). We anticipate requiring $\sim$250~k Service Units (SU) each quarter, and up to 500 TB of storage. These resources are already available via a Flagship between NCI and the Monash-led Australian Research Council project Securing Antarctica’s Environmental Future (SAEF).

\section*{Data management and archiving}

Data will be published adhering to FAIR principles (Findable, Accessible, Interoperable, Reusable), ensuring transparency and accessibility. The final bed topography datasets will be published at the Australian Antarctic Data Centre (AADC) under an open source licence. All production model outputs will be published with unique DOIs at repositories aligned with the corresponding journal articles. Model outputs – including production and other outputs – will be archived to tape at NCI using existing SAEF resources, as well as backed up to storage available through Monash MASSIVE M3 account aligned with project supervisor Dr McCormack. All journal articles published through this project will be open source, and tier 1 journals will be targeted.

\section*{Risk}

The project is highly feasible and low risk, given that it is a desk-based modelling and data assimilation project. All the data to be used in this project are freely available for download, and project supervisors are experts in ice sheet modelling using ISSM.

\section*{Career Development}

\textit{Fieldwork}\\
Fieldwork is not necessary to achieve the objectives of the project; however, there may be the opportunity to participate in fieldwork through the ICECAP airborne geophysics project (led CI of ICECAP is project supervisor Dr Jason Roberts, Australian Antarctic Division), which will be instrumental in training of geophysical instruments and in developing broader expertise in the field. \\
\textit{Conferences}\\
At least one conference will be attended each year. An international conference relevant to the discipline, e.g. the European Geophysical Union General Assembly, will be attended in the final year of the project.