\chapter{Resources}

The project will require high performance computing resources (including compute and storage) from the National Computing Infrastructure (NCI). We anticipate requiring $\sim$250~k Service Units (SU) each quarter, and up to 500 TB of storage. These resources are already available via a Flagship between NCI and the Monash-led Australian Research Council project Securing Antarctica’s Environmental Future (SAEF).

\section*{Currently available data and Framework}\label{data}
The project will make use of a number of new remote sensing datasets and software tools.

\begin{enumerate}
    \item\textbf{Reference Elevation Model of Antarctica (REMA)}\\
    REMA provides a high-resolution (2-metre) terrain map of nearly the entire continent, allowing for precise measurements of elevation changes over time. REMA supports various remote sensing activities, such as image orthorectification and interferometry, and aids in geodynamic and ice flow modeling, grounding line mapping, and surface process studies. Constructed from hundreds of thousands of Digital Elevation Models (DEMs) derived from high-resolution Maxar satellite imagery (including WorldView and GeoEye data), REMA is calibrated with Cryosat-2 and ICESat altimetry, ensuring high elevation accuracy with uncertainties of less than 1 meter over most areas\cite{REMA}.

    \item\textbf{ITS\_LIVE Antarctic surface velocities and elevation}\\
    The NASA-administered ITS\_LIVE website provides automated, high-resolution datasets of Antarctic surface velocities and ice surface elevation change, derived from satellite observations. The datasets are available on annual timesteps from 1985 to present. ITS\_LIVE employs various statistical and computational methods to process data from satellites including Landsat and Sentinel, ensuring precise and timely updates for scientific research~\cite{itslive}.

    \item\textbf{BedMachine Antarctica}\\
    A high-resolution map of Antarctic subglacial bed topography that provides unprecedented detail of basal features. The dataset combines multiple ice thickness measurements with mass conservation principles, satellite-derived ice flow velocities, and surface mass balance from regional atmospheric models. This methodology has led to significant corrections in known glacier depths (e.g., 200m deeper for Pine Island Glacier) and revealed previously unknown features, with bed slopes found to be steeper in 62\% of the mapped area compared to previous datasets\cite{Morlighem_2020}.

    \item\textbf{BedMAP}\\ 
    A suite of gridded products describing surface elevation, ice-thickness and the seafloor and subglacial bed elevation of Antarctica, based on a compilation of data collected by a large number of researchers using a variety of techniques, with the aim of representing a snap-shot of understanding of the Antarctic region\cite{Fretwell_2013}.

    % \item\textbf{ICECAP}\\ 

    \item\textbf{Ice-sheet and Sea-level System Model (ISSM)}\\
    ISSM is a finite-element numerical ice sheet model. It has been used to simulate the Antarctic Ice Sheet’s response to various climate scenarios and assess future mass loss contributions to sea level rise [9, 10]. The mesh can be refined to better capture variations in ice flow and driving stresses, enhancing the simulation’s accuracy of surface elevation changes and ice dynamics. This project will involve numerical modeling using advanced mathematical approaches, including the Blatter-Pattyn approximation to the full Stokes equations for ice flow (i.e. conservation of momentum equations). The Blatter-Pattyn approximation strikes a balance between the computationally intensive full Stokes equations and the simpler shallow ice approximation (SIA), retaining vertical shearing and longitudinal stress gradients. This makes it ideal for modeling the dynamics of fast-flowing ice streams and ice shelves at the continental scale, enhancing simulation accuracy while being computationally feasible. Additionally, data assimilation, machine learning, and geostatistics will be employed, with the full Stokes equations used if necessary~\cite{ISSM}
\end{enumerate}

\section*{Data management and archiving}

Data will be published adhering to FAIR principles (Findable, Accessible, Interoperable, Reusable), ensuring transparency and accessibility. The final bed topography datasets will be published at the Australian Antarctic Data Centre (AADC) under an open source licence. All production model outputs will be published with unique DOIs at repositories aligned with the corresponding journal articles. Model outputs – including production and other outputs – will be archived to tape at NCI using existing SAEF resources, as well as backed up to storage available through Monash MASSIVE M3 account aligned with project supervisor Dr McCormack. All journal articles published through this project will be open source, and tier 1 journals will be targeted.

\section*{Risk}

The project is highly feasible and low risk, given that it is a desk-based modelling and data assimilation project. All the data to be used in this project are freely available for download, and project supervisors are experts in ice sheet modelling using ISSM.

\textit{Fieldwork}\\
Fieldwork is not necessary to achieve the objectives of the project; however, there may be the opportunity to participate in fieldwork through the ICECAP airborne geophysics project (led CI of ICECAP is project supervisor Dr Jason Roberts, Australian Antarctic Division), which will be instrumental in training of geophysical instruments and in developing broader expertise in the field. \\

\section*{Career Development}

\textit{Conferences}\\
At least one conference will be attended each year. An international conference relevant to the discipline, e.g. the European Geophysical Union General Assembly, will be attended in the final year of the project.