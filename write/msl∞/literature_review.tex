\chapter{Antarctica's Landscape}\label{why}
\section{Climate Impacts and Global Significance}
The polar regions are losing ice, and their oceans are changing rapidly~\cite{O_C_in_changingClimate}. The consequences of this extend to the whole planet and it is crucial for us to understand them to be able to evaluate the costs and benefits of potential mitigation strategies. 
Changes in different kinds of polar ice affect many connected systems. Of particular concern is the accelerating loss of continental ice sheets (glacial ice masses on land) in both Greenland and Antarctica, which has become a major contributor to global sea level rise~\cite{O_C_in_changingClimate}. Impacts extend beyond direct ice loss: as fresh water from melting ice sheets is added into the ocean, it increases ocean stratification disrupting global thermohaline circulation~\cite{Jacobs_2004}. In addition, cold freshwater can dissolve larger amounts of $\mathrm{CO_2}$ than regular ocean water creating corrosive conditions for marine life~\cite{O_C_in_changingClimate}.
While there is high confidence in current ice loss and retreat observations in many areas, there is more uncertainty about the mechanisms driving these changes and their future progression~\cite{Fox-Kemper_2021}. Uncertainty increases in regions with variable bed conditions, where characteristics like bed slipperiness and roughness are difficult to verify via direct observations. Other problematic areas involve the ice sheet grounding line (GL): The zone that delineates ice grounded on bedrock from ice shelves floating over the ocean. The GL retreat rate depends crucially on topographical features like pinning points~\cite{Fox-Kemper_2021}, which lead to increased buttressing by the ice shelf on the upstream ice sheet. Although this mechanism is established, major knowledge gaps persist in mapping bed topography across Antarctic ice sheet margins, with over half of all margin areas having insufficient data within 5 km of the grounding zone~\cite{RINGS_2022}. Addressing this data gap through both systematic mapping and improved interpolation —utilising auxiliary data streams with more complete coverage— would significantly improve both our understanding of current ice dynamics and the accuracy of ice-sheet models projecting future changes.

\chapter{Topography of Antarctica}\label{review}
Bed topography is one of the most crucial boundary conditions that influences ice flow and loss from the Antarctic Ice Sheet (AIS)~\cite{Morlighem_2020}. Bed topography datasets are typically generated from airborne radar surveys, which are sparse and unevenly distributed across the Antarctic continent (see figure \ref{fig:BedMAP}). Interpolation schemes to ``gap fill'' these sparse datasets yield bed topography estimates that have high uncertainties (i.e. multiple hundreds of metres in elevation uncertainty; Morlighem et al. 2020) which propagate through simulations of AIS evolution under climate change~\cite{Castleman_2022}. Given the logistical challenges of accessing large parts of the Antarctic continent, there is a crucial need for alternative approaches that integrate diverse and possibly more spatially complete data streams – including satellite data.
\begin{figure}[H]
    \includegraphics[scale=0.31]{bedmap.png}
    \caption{Distribution of BedMAP\{1,2,3\} data tracks (Source: bedmap.scar.org).}
    \label{fig:BedMAP}
\end{figure}
\newpage
\section{Approaches to Bed Topography Reconstruction}
An objective of my research is to understand the bed topography itself and how it influences ice dynamics. There are two ways to infer information about this relationship: Through forward modelling, with assumptions of the bed conditions; and through inverse modelling that relies on surface observations.
\begin{itemize}
    \item\textbf{Forward models}\\
    The aim of forward models is to see how bed properties impact ice dynamics. A key example is using a large ensemble of bed topographies to investigate how bed uncertainties impact simulated ice mass loss. Geostatistical methods can be used to generate bed topographies that either preserve elevation or texture:    
    \begin{itemize}
            \item\textbf{Geostatistics} is comprised of statistical methods specialized for analyzing spatially correlated data. In glaciology, this approach is used to interpolate between sparse measurements and characterise spatial patterns in bed properties, often employing techniques like kriging~\cite{Mackie_2020} or flexural modeling—a computational method that models the earth's lithosphere response to addition or removal of materials—See subsection~\ref{Ancient_River} for more information on this technique.
    \end{itemize}
    \item\textbf{Inversion models}\\
    The aim of these models is to understand bed properties through knowledge of surface or other variables. A key example is the retrieval of bed topography or basal slipperiness from surface elevation and velocities.
        \begin{itemize}
            \item\textbf{Control method inversion}: A variational approach that minimizes mismatches between observed and simulated fields through a cost function approach. Remote sensing data and theoretical ice flow models are used to obtain basal conditions~\cite{deRydt_2013}. Often needs regularization terms to prevent non-physical features or over-fitting~\cite{Morlighem_Goldberg_2024}.
            \item\textbf{4dvar}: Four-dimensional variational data assimilation - Similar to the control method inversion algorithm, but adds a time dimension. Used to optimize model parameters and initial conditions~\cite{Morlighem_Goldberg_2024}. Can handle time-varying data and evolving glacier states, making it more suitable for dynamic systems unlike control methods. The trade-off for extra functionallity is increased computational cost~\cite{Morlighem_Goldberg_2024}.
            \item\textbf{Mass conservation}: Used to constrain inversion models and fill data gaps by employing physical conservation laws, particularly effective for reconstructing bed topography where direct measurements are sparse~\cite{Morlighem_2017, Morlighem_2020}. Requires (contemporary) measurements of ice thickness at the inflow boundary to properly constrain the system~\cite{Morlighem_Goldberg_2024}.
            \item\textbf{Markov Chain Monte Carlo (MCMC)}: A probabilistic method that generates sample distributions to quantify uncertainties in ice sheet parameters and models~\cite{Morlighem_Goldberg_2024}. While powerful for uncertainty quantification, these methods remain computationally intensive for continental-scale ice sheet models~\cite{Morlighem_Goldberg_2024}.
            \item\textbf{EnKF} Ensemble Kalman Filter. A sequential data assimilation method that uses an ensemble of model states to estimate uncertainty and update model parameters based on observations~\cite{Morlighem_Goldberg_2024}.
        \end{itemize}
\end{itemize} 
Despite revolutionary advances in satellite technology that provide unprecedented surface detail, a key challenge in glaciology remains: how to fully utilize this wealth of information in regions where our understanding of subglacial conditions is limited. My research aims to develop an integrated method combining forward and inverse modeling to improve bed topography estimates by leveraging high-resolution satellite surface data in regions where radar data is sparse.% My approach will integrate more comprehensive dynamical ice models with modern computational capabilities to develop better bed topography.

\section{Ancient Landscapes Preserved Beneath the Ice Sheet}\label{Ancient_River}
Discoveries of landscapes preserved beneath the Antarctic Ice Sheet provide crucial context and motivation for developing new bed reconstruction methods. Jamieson et al. (2023) identified a hidden river landscape ($\approx32,000~\mathrm{km}^{2}$ total area, with $800~\mathrm{m}$ mean relief) under the ice in central East Antarctica. The preserved landscape was mapped using a combined data stream approach that incorporates satellite (RADARSAT, REMA) and Radar Echo Sounding (RES) data. Their approach—using ice surface slope changes to infer buried topography— exemplifies how multiple data integration can be leveraged to develop bed topography mappings beneath the ice sheet. 
The methods used by Jamieson et al. (2023) offer a direct template for BedSAT's methods and validation. To test their hypothesis of an ancient land surface preserved beneath the modern EAIS, the researchers established a set of criteria (relevant anywhere in the continent) to be met by their observations. The geostatistical techniques they used include
\begin{itemize}
\item{\textbf{Flexural modeling}}: In the form of isostatic rebound (uplift) modeling they demonstrate how the landscape responds when the ice removes material via erosional unloading. The paper shows that the current landscape is consistent with a splitting of a single uplifted topographic feature. 
\item{\textbf{Hypsometric distributions}}: Which demonstrated that the region investigated actually satisfied a common elevation signature. 
\item{\textbf{quantitative analysis}}: This work provides a real-world quantitative target for what high-resolution bed topography should look like in this region. They identify a mean relief of $\approx 800\mathrm{m}$ and ridge-valley spacings of $2$ to $3~\mathrm{km}$. This provides a crucial benchmark for my work with BedSAT. I aim to reproduce similar physically realistic statistical properties. 
\end{itemize}
The paper arguments on landscape preservation strongly motivate the rheology and sliding conditions analysis in my own work (see Section~\ref{study1}). Since there are dramatic differences in the landscape expression depending on whether there is sliding conditions or not at the base of a glacier. Jamieson et al. find that rapid transitions between warm-based (erosive) and cold-based (preservative) ice states are necessary to explain the landscape's survival. The paper argues that the landscape predates $14~\mathrm{Ma}$ (and possibly $34~\mathrm{Ma}$), this implies that there is long-term thermal stability over these blocks of ancient terrain, with cold-based ice being the average glacial condition since landscape formation~\cite{Jamieson_2023}.  Their conclusion that the ice margin has not retreated this far inland for at least $14~\mathrm{Ma}$ provides a critical long-term stability benchmark , reinforcing the need for improved bed topography models to accurately project how this stability may change under future anthropogenic climate forcing.

\section{Features of a White Desert: Aeolian Snow Dunes}\label{Aeolian_dunes}
In a continent-wide survey using satellite imagery, researchers discovered that linear snow dunes are a ubiquitous feature of the Antarctic landscape, covering over $95\%$ of the area studied~\cite{Poizat_2024}.
The dunes range from $100$ to $1,000~\mathrm{m}$ in length~\cite{Poizat_2024} and they are predominantly longitudinal, with $61\%$ aligning with the local snow drift direction. Often developing under unidirectional wind regimes~\cite{Poizat_2024}. Poizat et al. propose that the evolution of these aeolian landforms is regulated by snow sintering—a process where snow hardens into larger ice crystals after deposition, greatly limiting erosion—they suggest the  sintering process as a mechanism of ice-sheet mass balance~\cite{Poizat_2024}.
While Antarctica's snowfields exhibit a high diversity of aeolian bedforms, the study of snow dune formation and dynamics is still in early stages~\cite{Poizat_2024}. Similar to sand dunes, snow dunes are shaped by dynamic interactions involving topography, wind patterns, and particle transport~\cite{Poizat_2024}. However the process is not as straightforward as in a regular desert, because snow sublimation contributes to snow depletion and the hardening of bedforms, unlike other sedimentary environments. Understanding these surface formations is critical for the accurate modeling of the surface mass balance (SMB), as the dunes enhance snowpack heterogeneity, which in turn reduces the accuracy of remote sensing, SMB measurements, modeling, and ice core interpretation~\cite{Poizat_2024}.

\subsection{Critical Gap: The Influence of Aeolian Surface Features on Bed Topography Inversion}
Inferring Antarctic subglacial topography from surface observations is a problem plagued with complexity. For instance, the recent work that I document in~\ref{Aeolian_dunes} successfully linked widespread snow dune orientations, observed via satellite, to continental-scale wind patterns. The research by Poizat et al. Demonstrates a real physical challenge to BedSAT. The core assumption behind BedSAT is that the dominant, coherent signals at the ice surface are expressions of the underlying bed. However, Poizat et al. (2024) demonstrate that vast regions of the Antarctic surface are covered in highly organized, large-scale patterns that are purely aeolian in origin.
The potential challenge in inverting for bed topography from surface data is distinguishing the ice-dynamic response to the bed from other surface features. The ubiquitous longitudinal snow dunes identified by Poizat et al. and their continental-scale organization could introduce a systematic roughness or 'noise' in surface elevation datasets that must be considered when isolating the glaciological signal originating from the bed.
BedSAT has to overcome the logistical challenges of direct measurement in Antarctica by leveraging the vast, continuous coverage of satellite remote sensing data and by explicitly learning to be resilient to aeolian signal interference. Rather than simply treating these features as noise to be filtered, a more robust approach is to make the model aware of the processes that create them. The most direct way to overcome this challenge is by incorporating aeolian noise in the synthetic training data.
The forward models used to generate synthetic data can be enhanced by superimposing a statistically realistic aeolian roughness layer onto the ISSM-generated ice surface. The characteristics of this roughness (e.g. wavelength, amplitude, orientation) will be directly informed by the continental-scale observations in Poizat et al. (2024). 
After taking this into account, the BedSAT framework will be better equipped to isolate the true glaciological signal, turning a significant environmental challenge into a methodological strength of the inversion process.

\newpage
\section{Theoretical Frameworks}\label{theoretical_frameworks}
 Understanding how bed features manifest in surface observations requires a theoretical framework that connects these two domains. The modelling approach used in this project relies on two different theoretical frameworks that relate bed topography and surface features. Using synthetic data, observations and these modelling frameworks, my goal is understanding the limitations of each approach and how they can be improved.

\subsection{Ice Flow Over Bedrock Perturbations - Budd 1970}
The first framework was originally developed by Budd~\cite{Budd_1970}. This model relates ice flow over bedrock perturbations to surface expressions using a two-dimensional biharmonic stress equation. The modelling carried out by Budd determined ice-sliding velocities for wide ranges of roughness, normal stress, and shear stress relevant to real glaciers~\cite{Budd_1970}. Budd's framework fits in my project as a means of verifying the validity of the physical configuration in my ice-sheet model, since it describes important effects of bedrock disturbances on the transient evolution of the transfer of basal disturbances onto the surface. The theory makes several key predictions that have been confirmed through spectral analysis of real ice cap profiles: 
\begin{enumerate}
    \item A basal disturbance wavelength of minimum damping occurs at approximately 3.3 times the ice thickness, 
    \item Surface undulations exhibit a $\pi/2$ phase lag relative to bedrock features with steepest surface slopes occurring over the highest bedrock points, and 
    \item The amplitude reduction depends systematically on ice speed, viscosity, thickness, and wavelength. 
\end{enumerate}
Importantly, Budd's theory demonstrates that energy dissipation and basal stress patterns are maximized for bedrock irregularities with wavelengths several times the ice thickness, while smaller-scale bedrock variations decay exponentially with distance into the ice and have minimal impact on overall ice motion. This selective filtering of bedrock signals provides crucial insights for understanding which scales of bed topography most significantly influence ice dynamics.
A critical aspect of Budd's theoretical framework is understanding how ice rheology affects the bed-to-surface transfer relationships. Glen's flow law typically employs a stress exponent $n\approx 3$ for ice under most natural conditions, reflecting the strongly nonlinear relationship between stress and strain rate. However, more recent research suggests that $n = 4$ may better
represent ice flow in some locations~\cite{Getraer_2025}.  Budd's analysis revealed that under certain low-stress conditions, ice deformation can behave more linearly ($n\approx 1$) than conventional wisdom suggests. 
This rheological distinction has profound implications for bed-to-surface transfer functions: because linear rheology $(n = 1)$ may produce different amplitude dampening and phase relationships compared to nonlinear rheology $(n = 4)$, particularly for wavelengths around the critical 3.3 times ice thickness scale.
My current modelling work systematically explores this by generating forward models for multiple synthetic bedrock profiles across four scenarios combining rheological assumptions $(n = 1$ vs $n = 4)$ with basal boundary conditions (no-slip vs sliding), enabling direct comparison of how these physical assumptions affect the detectability and reconstruction of bed features from surface observations. Understanding these differences is essential for developing robust inversion methods, as the choice of rheological model fundamentally determines the mathematical relationship between observable surface expressions and the underlying bed topography I seek to reconstruct.
Crucially, Budd's work established the concept of frequency-dependent transfer functions that act as "filters" between bed and surface topography. This transfer function approach, expressed as $\psi(\omega) = \frac{\text{surface amplitude}}{\text{bed amplitude}}$ for wavelength $\lambda = 2\pi/\omega$, provides a direct mathematical framework for inversion. By inverting these transfer functions, one can theoretically reconstruct bed topography from surface observations, particularly for wavelengths where the damping factor is minimal and the signal-to-noise ratio is optimal.

\subsection{Ice flow perturbation analysis - Ockenden 2023}
The second framework in my analysis builds upon these foundational concepts through some of the recent work by Ockenden et al. in~\cite{Ockenden_2023}, which uses observed surface perturbations (in velocity and elevation) to invert for unknown basal perturbations. Ockenden et al. improve from their previous work in~\cite{Ockenden_2022} by using full-Stokes transfer functions, which greatly improves their method when dealing with steep topography where the shallow-ice-stream approximation breaks down. Ockenden et al. find this is crucial for better resolving the topographic features they are interested in. The core principle of the method by Ockenden et al. (2023) relies on the fact that variations in basal topography, slipperiness, and roughness cause measurable disturbances to the surface flow of the ice. Through linear perturbation analysis, they establish a systematic relationship between surface observations and bed conditions. This relationship can be expressed as $y=f(x)$, where $y$ represents surface measurements (velocity and topography), $x$ represents bed properties (topography and slipperiness), and $f$ is the forward model transfer function refined by Gudmundsson and Raymond in 2008~\cite{Gudmundsson_2008}. 
In their work Ockenden et al apply this framework in reverse $x=f^{-1}(y)$, to infer the bedrock from modern, high-resolution satellite data estimates. A restrictive assumption in the modeling design by Ockenden et al might be their assumption of ``constant viscosity'', in glaciology this is equivalent to assuming a linear rheology, where the stress exponent, $n$, is equal to 1. This means that the strain rate is directly proportional to the stress. This is in contrast to the more commonly used non-linear Glen's Flow Law, where $n$ is typically around 3 or even 4. This earliest phase in my PhD project has as a goal to determine whether treating the rheology of ice as  linear is adequate. Ockenden et al account for a non-linear sliding law at the base of the ice, mentioning the ``sliding law parameter m''. However, the transfer of stress through the body of the ice—the core of the perturbation analysis—relies on the constant viscosity assumption from the foundational work of Gudmundsson and Raymond 2008.

% How will I use the insights from Budd's original analysis (e.g. on non-linear rheology) to inform my new inversion method?

\subsection{Bridging Classical and Modern Approaches}
While both frameworks address the fundamental bed-to-surface relationship, they operate at different levels of complexity and make different assumptions. Budd's approach provides the fundamental physical understanding of how specific wavelengths propagate through ice, establishing theoretical limits on what bed features can be detected from surface observations. Ockenden's method extends this to practical applications using real satellite data but relies on linearised assumptions that may break down under certain conditions. My research aims to bridge these approaches by combining Budd's rigorous transfer function analysis with comprehensive forward modeling that relaxes some of the restrictive assumptions inherent in linear perturbation methods. By systematically exploring how different rheological models $(n = 1$ vs $n = 4)$ and basal conditions affect the bed-to-surface transfer functions, my work aims to develop a robust inversion method that can better handle the nonlinear physics of ice flow while maintaining the theoretical rigor established by Budd's foundational analysis.

\section{Current Opportunities}
Current Antarctic bed topography reconstruction methods fail to utilize the wealth of presently available satellite-derived surface data. While mathematical models linking bed to surface through ice dynamics (such as those by Ockenden and Budd) provide a foundation for inferring bed topography from satellite data, they have significant limitations. My approach with BedSAT builds upon theoretical foundations and recent inversion methods to better understand how bed conditions—including slipperiness, roughness, and pinning points—affect both grounding line retreat rates and their surface expressions. BedSAT will connect surface observations with bed topography using more realistic rheological and geometric assumptions through an iterative process: initially inverted bed topography will feed into ice dynamics models with these improved assumptions, allowing comparison between model predictions and established datasets like NASA's ITS\_LIVE. I expect to utilise Machine learning methods to systematize this process, enhancing the analytical capabilities for the project's final phase.

\subsection{Physics-Informed Machine Learning for Bed Topography Inversion}\label{ML}
In the rheology and sliding study in section~\ref{study1}, I am establishing a ``forward problem'' investigation—how bed topography influences surface expression—under various physical assumptions. While this provides crucial physical insight, the ultimate goal of BedSAT is to solve the ``inverse problem'': inferring bed topography from surface observations. This task is computationally intensive, especially when considering very large datasets and their corresponding uncertainties. To address this challenge, I will explore the use of Physics-Informed Machine Learning (Physics-ML), leveraging the NVIDIA PhysicsNeMo platform.

PhysicsNeMo is a framework designed to create high-fidelity, deep learning surrogate models by blending the governing physics of a system—Partial Differential Equations (PDEs)—with training data~\cite{NVIDIA_NeMo_2025}. I plan to use PhysicsNemo with BedSAT in the following ways: 

\begin{enumerate}
\item{Forward model training}: PhysicsNeMo can learn the relationship between ice sheet surface velocity and elevation to bed topography, basal sliding and ice rheology ($n=1$ vs $n=4$)(see~\ref{study1}). This trained model can then generate vast amounts of synthetic training data—including variations with statistically realistic aeolian noise—orders of magnitude faster than a traditional ISSM solver.

\item{Solving the Inverse Problem}: PhysicsNeMo is explicitly designed to solve inverse problems by using observational data to infer unknown system parameters~\cite{NVIDIA_NeMo_2025}. BedSAT will rely on the PhysicsNeMo data-driven architecture to learn the mapping from surface expression to bed topography, effectively creating a fast and accurate inverse solver.
\end{enumerate}

By integrating PhysicsNeMo, BedSAT will develop into a parameterized surrogate model capable of near real-time inference, satisfying my project's third objective: Allowing for rapid sensitivity analyses of ice mass loss projections to different realisations of topographic roughness. This Physics-ML approach represents a significant step beyond traditional inversion methods, promising to enhance both the computational efficiency and physical realism of Antarctic bed topography reconstruction.