\chapter{Antarctica's Landscape}\label{why}
\section*{Climate Impacts and Global Significance}

The polar regions are losing ice, and their oceans are changing rapidly\cite{O_C_in_changingClimate}. The consequences of this polar transition extend to the whole planet and it is crucial for us to understand them to be able to evaluate the costs and benefits of potential mitigation strategies. 

The consequences of changes in different kinds of polar ice manifest across multiple interconnected systems. In both polar oceans, shifts in seasonal sea ice (both `pack-ice' that moves with ocean currents and `land-fast ice' that remains attached to the coast\cite{SeaIce}) are altering marine ecosystems, from the production of new organic matter by photosynthetic organisms like plants and algae, to species distribution\cite{O_C_in_changingClimate}. Of particular concern is the accelerating loss of continental ice sheets (permanent glacial ice masses on land) in both Greenland and Antarctica, which has become a major contributor to global sea level rise\cite{O_C_in_changingClimate}. Impact extends beyond direct ice loss: As fresh water from melting ice sheets is added into the ocean, it increases ocean stratification. Cold freshwater can dissolve larger amounts of $\mathrm{CO_2}$ than regular ocean water. Increased $\mathrm{CO_2}$ uptake by polar oceans is creating corrosive conditions for calcifying organisms\cite{O_C_in_changingClimate}. In addition, freshwater stratification threatens to disrupt global thermohaline circulation\cite{Jacobs_2004} by decreasing the natural mixing of the ocean layers.

Antarctica's climate response, however, differs markedly from the Arctic's uniform warming pattern. While West Antarctica has experienced warming in certain regions, East Antarctica has shown minimal overall temperature change in recent decades\cite{O_C_in_changingClimate}. These observations carry low confidence due to limited data availability and high variability\cite{O_C_in_changingClimate}. This asymmetric response is partly explained by the Southern Ocean's unique ability to absorb and mix heat into its depths\cite{L_T_C_C}.
 
To make matters more challenging, there is significant uncertainty in the timing and magnitude of Antarctica's ice loss, largely due to unknowns in ice sheet properties and associated flow processes\cite{IPCC}. Uncertainty increases in regions with variable bed conditions, where characteristics like "slipperiness" and roughness are difficult to verify via direct samples. Additionally, our assumptions about temperature and depth-dependent parameters like viscosity affect several key physical processes. How we use these model variables to constrain ice dynamics where we have data gaps is crucial for our ability to accurately predict ice sheet behavior\cite{Ockenden_2022}.

\chapter{Topography of Antarctica}\label{review}

Bed topography is one of the most crucial boundary conditions that influences ice flow and loss from the Antarctic Ice Sheet (AIS)\cite{Morlighem_2020}. Bed topography datasets are typically generated from airborne radar surveys, which are sparse and unevenly distributed across the Antarctic continent (see figure \ref{fig:BedMAP}). Interpolation schemes to "gap fill" these sparse datasets yield bed topography estimates that have high uncertainties (i.e. multiple hundreds of metres in elevation uncertainty; Morlighem et al., 2020), which propagate in simulations of AIS evolution under climate change\cite{Castleman_2022}. Given the logistical challenges of accessing large parts of the Antarctic continent, there is a crucial need for alternative approaches that integrate diverse and possibly more spatially complete data streams – including satellite data – to derive bed topography.
\begin{figure}[H]    % Forces the figure exactly HERE
    \includegraphics[scale=0.4]{bedmap.png}
    \caption{Distribution of BedMAP\{1,2,3\} data tracks (Source: bedmap.scar.org).}
    \label{fig:BedMAP}
\end{figure}

Obtaining information about the conditions at the base of the ice-sheet often relies on indirect modelling methods like
\begin{itemize}

    \item\textbf{Inversions}
        \begin{itemize}
            \item\textbf{Mass conservation}: Used to constrain inversion models and to fill data gaps by taking advantage of the physical laws of conservation of mass and momentum~\cite{Morlighem_2017, Morlighem_2020}. (Monte Carlo Mass Conservation: Random sampling solves mass conservation equations\cite{Brinkerhoff_2016}).
            \item\textbf{Control method inversion}: Basal conditions distribution information is obtained from remote sensing data and theoretical ice flow models\cite{deRydt_2013}.
            \item\textbf{Statistical inversion}: Study the simultaneous retrieval (or update) of bed topography and basal slipperiness from surface topography and velocity measurements\cite{deRydt_2013}. See section \ref{Ockenden_2022} for an outlined example using \textbf{linear perturbation analysis}.
        \end{itemize}

    \item\textbf{4dvar}: Four-dimensional variational data assimilation - Minimizes the difference between model predictions and observations across a time window. Mainly used to optimize model parameters and initial conditions\cite{Morlighem_Goldberg_2024}.

    \item\textbf{Geostatistics} Statistical methods specialized for analyzing spatially correlated data. In glaciology, it's used to interpolate between sparse measurements and characterize spatial patterns in bed properties, often employing techniques like kriging\cite{Mackie_2020}.

    \item\textbf{EnKF} Ensemble Kalman Filter A sequential data assimilation method that uses an ensemble of model states to estimate uncertainty and update model parameters based on observations\cite{Morlighem_Goldberg_2024}.

\end{itemize} 

\section{A case study}\label{Castleman_2022}
\subsection*{Critical Factors Influencing Thwaites Glacier's Future Evolution}

Thwaites Glacier in West Antarctica represents one of the most impactful potential contributors to the mean global sea-level rise (SLR), with an estimated contribution of 0.59 meters\cite{Holt_2006}. The glacier's future evolution is of particular concern because it could trigger a broader collapse of the West Antarctic Ice Sheet (WAIS)\cite{Holt_2006, Castleman_2022}. Understanding the factors that control its susceptibility to instabilities, including the marine ice sheet instability is therefore crucial for accurate sea-level rise projections\cite{Castleman_2022}.

Two primary factors control Thwaites Glacier's evolution: Ocean-driven basal melt rates and bedrock topography\cite{Castleman_2022}. The Castleman et al.\cite{Castleman_2022} study reveals that the glacier is highly sensitive to changes in bedrock topography within current measurement error bounds, highlighting the critical need for more accurate topographical data.

The glacier's vulnerability is further complicated by Marine Ice Sheet Instability (MISI)\cite{Castleman_2022}, a feedback mechanism where warmer ocean temperatures accelerate ice shelf melting and calving. Buttressing ice shelves have a stabilizing effect on the ice sheet and can potentially suppress or delay MISI\cite{Wernecke_2022}. For ice sheets on retrograde topographies (such as Thwaites glacier), this stabilizing effect fails. In these cases, the Grounding Line (GL)—the transition from grounded to floating ice—undergoes retreat. Once initiated on such topography, grounding line retreat can accelerate\cite{Castleman_2022}.
Ocean-driven basal melt rates present another significant source of uncertainty in ice-sheet model simulations. This uncertainty stems from multiple factors\cite{Castleman_2022}: the stochastic nature of ocean circulation patterns, temporal variability in ocean forcing, limitations in current ocean models, and limited direct ocean observations underneath ice shelves. These factors make it particularly difficult to predict how the ice sheet will respond to future ocean warming scenarios\cite{Castleman_2022}.

Current methods to generate topography datasets rely primarily on ice-penetrating radar, which presents significant challenges. While radar can provide direct measurements along specific tracks, interpolation is used to fill gaps between these tracks. This interpolation introduces uncertainties, particularly in areas where measurements are sparse. These uncertainties significantly affect our ability to model the glacier's future behavior accurately.

To quantify these uncertainties, Castleman et al.\cite{Castleman_2022} employed two-dimensional discrete wavelet transform (DWT) to systematically analyze and modify bedrock topography data. Their method decomposes a bedrock elevation map $B(x,y)$ into four distinct subarrays: $\mathbf{A}_n$ (low-frequency approximation), and three high-frequency components - $\mathbf{H}_n$ (horizontal), $\mathbf{V}_n$ (vertical), and $\mathbf{D}_n$ (diagonal), where $n$ indicates the decomposition level. 

The approach selectively amplified the high-frequency components using a multiplier $\alpha > 1$, creating modified arrays $\mathbf{H}'_n = \alpha\mathbf{H}_n$, $\mathbf{V}'_n = \alpha\mathbf{V}_n$, and $\mathbf{D}'_n = \alpha\mathbf{D}_n$. This approach allows to introduce realistic perturbations into bedrock topography models and assess how varying spatial and vertical resolutions affect SLR projections.

One of the study's most significant findings relates to the importance of "pinning points" - bedrock features that can temporarily halt or slow grounding line retreat. While their wavelet-based method could potentially bias results toward more effective pinning points, the study revealed that the location of bedrock features relative to the grounding line and their deviation from mean bed elevation were more significant than feature amplification\cite{Castleman_2022}. Through this analysis, they established crucial requirements for future bedrock measurements: 2 km spatial resolution and $\pm$8 meters vertical accuracy, particularly near the grounding line, to keep SLR uncertainty within $\pm$2 centimeters.

This study underscores the critical importance of accurate bedrock topography measurements for reliable SLR projections. The findings provide clear guidelines for future mapping efforts and highlight the need for focused attention on grounding line regions where bedrock features have the most significant impact on glacier stability.

\section{Ice Sheet Bed Reconstruction via  Surface Data Inversion}\label{Ockenden_2022}

An important observation is that features in the ice bed often show up as subtle patterns in the surface topography above them\cite{Ockenden_2022}. These bed conditions - both their shape and mechanical properties - significantly influence how ice flows, with even small changes at the bed potentially leading to large differences in predicted ice loss rates. To address this challenge, is common to use inversion methods to understand the geophysical conditions at the ice sheet bed.
The relationship between bed and surface characteristics is core to inversion methods that attempt to reconstruct bed properties from surface observations. Unfortunately, these methods cannot paint a complete picture of the ice sheet model. Inversion methods require careful tuning of parameters which are not directly observable  (e.g. the basal friction coefficient), so that modelled surface velocity matches observations. The mathematical framework for these inversions can be achieved via steady-state \textbf{linear perturbation analysis} of shallow stream approximation (SSA)\cite{Gudmundsson_2008}.

\subsection*{Theoretical Framework}

Inversion is based on the principle that variations in basal topography, slipperiness, and roughness cause disturbances to the surface flow of the ice. By measuring these disturbances in surface velocity and topography, and using equations like the shelfy-stream approximation (SSA) to relate those disturbances back to their source, we can estimate the basal conditions.
The indirect measurements of basal properties $x$ and $y$ are related through $y=f(x)$, where $f$ is referred to as the forward model, this makes  $x=f^{-1}(y)$ the inversion. The indirect measurements $y$ can be measurements of velocity and topography along the upper surface of a glacier, while the quantity $x$ to be estimated represents basal topography and basal slipperiness\cite{Gudmundsson_2008}.
The inversion method developed by Ockenden et al. introduces perturbations to study how small changes in ice thickness ($h$), surface elevation ($s$), basal topography ($b$), and ice velocity ($u$) affect ice flow. This means that the method is most accurate when applied to small perturbations, with the assumption that the perturbations are small relative to the mean of each studied property based on a reference state. The method assumes:
\begin{enumerate}
\item A linear viscous medium ($n=1$)
\item Non-linear sliding law ($m>0$)
\item Steady-state conditions
\item Spatially constant zero-order solutions
\end{enumerate}

The SSA system described in Ockenden et al.\cite{Ockenden_2022} is as follows:
\begin{equation}\partial_{x} (4 h \eta \partial_{x} u + 2 h \eta \partial_y v) + \partial_{y}(h \eta( \partial_{x} v + \partial_{y} u)) - (u/c^{1/m}) = \rho g h ( \partial_{x} (s) \mathrm{cos}(\alpha) - \mathrm{sin}(\alpha))
\end{equation}\label{eq:2.1}\\
\begin{equation}\partial_{y} (4 h \eta \partial_{y} v + 2 h \eta \partial_x u) + \partial_{x}(h \eta( \partial_{y} u + \partial_{x} v)) - (v/c^{1/m}) = \rho g h ( \partial_{y} (s) \mathrm{cos}(\alpha)
\end{equation}\label{eq:2.2}

Equations \ref{eq:2.1} and \ref{eq:2.2} are a linearised system around a reference model $(\bar{h}, \bar{s}, \bar{b}, \bar{u}, \bar{v}, \bar{c})$, leading to first-order momentum balance equations:
\begin{equation}
4 \eta \bar{h} \partial_{xx} \Delta u + 3 \eta \bar{h} \partial_{xy}^{2} \Delta v + \eta \bar{h} \partial_{yy}^{2}\Delta u -\gamma \Delta u  = \rho g \bar{h}\mathrm{cos}(\alpha) \partial_x \Delta s - \rho g \mathrm{sin}(\alpha)\Delta h
\end{equation}\\
\begin{equation}
4 \eta \bar{h} \partial_{yy} \Delta v + 3 \eta \bar{h} \partial_{xy}^{2} \Delta u + \eta \bar{h} \partial_{xx}^{2}\Delta v -\gamma \Delta v  = \rho g \bar{h}\mathrm{cos}(\alpha) \partial_y \Delta s
\end{equation}

where $h$ represents ice thickness, $s$ surface elevation, $(u, v)$ horizontal components of surface velocity, $c$ basal slipperiness, $\eta$ effective ice viscosity, $m$ sliding law parameter, $\rho$ ice density, $\alpha$ mean surface slope in $x$-direction, and $g$ acceleration due to gravity.

A disadvantage of the  assumptions above is that they can limit how well we can model the behavior of real ice which exhibits nonlinear rheology.

\subsection*{Transfer Functions and Implementation}

The methodology in \cite{Ockenden_2022} employs transfer functions. Transfer functions are mathematical expressions that describe how perturbations in basal topography and slipperiness affect surface topography and velocity. They are derived from the SSA equations

$$\begin{bmatrix}
\hat{s} \\
\hat{u} \\
\hat{v}
\end{bmatrix} =\begin{bmatrix}
T_{sb} & T_{sc} \\
T_{ub} & T_{uc} \\
T_{vb} & T_{vc}
\end{bmatrix}
\begin{bmatrix}
\hat{b}\\
\hat{c}
\end{bmatrix}$$
The system is solved using a weighted least-squares approach, minimizing:
\begin{equation}
\Sigma s(s_{\mathrm{obs}} - s_{\mathrm{pred}})^2 + \Sigma u(u_{\mathrm{obs}} - u_{\mathrm{pred}})^2 + \Sigma v(v_{\mathrm{obs}} - v_{\mathrm{pred}})^2
\end{equation}

Inversion combines the surface data with the transfer functions to estimate the bed properties that would cause the surface changes.

\subsection*{Application and Limitations}

Ockenden et al. report to have successfully implemented their inversion method using REMA surface elevation data (8m resolution) and NASA ITS\_LIVE velocity data (120m resolution). The method performs particularly well in:
\begin{itemize}
\item Areas with moderate topographic gradients in the central trunk of glaciers
\item Features not aligned with ice flow direction
\item Medium-wavelength (5-50km) bedrock features
\end{itemize}
However, the method faces limitations in cases of:
\begin{itemize}
\item Steep topography where shallow-ice-stream approximation breaks down
\item Features aligned with ice flow direction
\item Variable slipperiness parameters
\item Lack of validation data for slipperiness predictions
\end{itemize}
While this work represents a significant advance in our ability to reconstruct bed conditions using surface data. Modern satellite technology provides unprecedented detail of ice sheet surface features through high-resolution elevation models and velocity measurements. However, we have yet to fully harness this wealth of surface information to improve bed topography estimates, particularly in regions where radar measurements are sparse. This disconnect points to a critical need for robust mathematical frameworks that can systematically link bed conditions to observable surface characteristics. Work, such as Budd's model relating ice flow over bedrock perturbations to surface expressions\cite{Budd_1970}, provides foundational approaches for establishing these relationships. Building upon this existing framework while incorporating modern satellite observations could significantly enhance our ability to reconstruct bed topography in poorly sampled regions of Antarctica.