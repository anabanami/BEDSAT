\chapter{Antarctica's Landscape: Climate Impacts and Global Significance}\label{why}

The polar regions are losing ice, and their oceans are changing rapidly\cite{O_C_in_changingClimate}. The consequences of this polar transition extend to the whole planet and it is crucial for us to understand them and plan for changes. However, there is significant uncertainty in the timing and magnitude of Antarctica's ice loss, largely due to unknowns in ice sheet properties and associated flow processes~\cite{IPCC}.

\begin{itemize}
\item Climate-induced changes in seasonal ice extent and thickness are affecting sea ice and ocean layers, which impacts marine plant growth (highly likely). This alters ecosystems (moderately likely). The timing and amount of plant growth have changed in both polar oceans, varying by location (highly likely). In Antarctica, these changes relate to retreating glaciers and sea ice change (moderately likely). In the Arctic, they've affected the types, locations, and numbers of marine species, changing ecosystem structure (moderately likely)\cite{O_C_in_changingClimate}.
\item The rapid ice loss from the Greenland and Antarctic ice sheets during the early 21st century has increased into the near present day, adding to the ice sheet contribution to global sea level rise (SLR)(extremely likely)\cite{O_C_in_changingClimate}.
\item Both polar oceans will be increasingly affected by $\mathrm{CO_2}$ uptake, causing conditions corrosive for calcium carbonate shell-producing organisms (high confidence), with associated impacts on marine organisms and ecosystems (medium confidence)\cite{O_C_in_changingClimate}.
\item Thermohaline circulation changes: a large-scale system of ocean currents driven by differences in water temperature (thermo) and salinity (haline), these factors affect the density of seawater. Thermohaline circulation plays a critical role in regulating Earth's climate and distributing heat and nutrients across the globe. Warm surface waters flow from the tropics toward the poles, where they cool and sink, forming deep-water currents. These deep waters then travel along the ocean floor toward the equator, eventually rising to the surface through a process called upwelling, bringing cold, nutrient-rich waters to the surface\cite{JACOBS_2004}. Altering the salt concentration in the Antarctic ocean due to the melting of the ice sheet could result in disruptions to this circulation and could have significant consequences for global weather systems.
\end{itemize}

Unlike the Arctic, which has seen uniform warming, Antarctica's temperature changes have been less consistent. West Antarctica has warmed in some parts. East Antarctica hasn't shown significant overall change. in the past 3-5 decades. There's low confidence in these observations due to limited data and high variability\cite{O_C_in_changingClimate}. \\\\
\textbf{Atmospheric factors influencing Antarctic climate:}
\begin{itemize}
\item Southern Annular Mode (SAM)\\
    \textbf{Recent changes in the Southern Annular Mode (SAM):} The SAM has been mostly positive in recent decades during summer. This means stronger westerly winds around Antarctica. This positive phase is unprecedented in at least 600 years. It's associated with cooler conditions over Antarctica\cite{O_C_in_changingClimate}.\\
    \textbf{Causes of SAM changes:} Ozone depletion was likely the main driver of SAM changes from the late 1970s to late 1990s. Since 2000, tropical sea surface temperatures have played a stronger role in influencing SAM\cite{O_C_in_changingClimate}.\\
    \textbf{Other influences on Antarctic climate:} Tropical sea surface temperatures can affect Antarctic temperatures and Southern Hemisphere mid-latitude circulation\cite{JACOBS_2004}.
\item Pacific South American mode
\item Zonal-wave 3
\end{itemize}

\textbf{Other factors influencing Antarctic climate:}
\begin{itemize}
\item Antarctica isn't warming as much as the Arctic because the Southern Ocean surrounding Antarctica absorbs and mixes heat deep into the ocean\cite{L_T_C_C}.
\end{itemize}

\section*{The importance of bed topography}
Bed topography is one of the most crucial boundary conditions that influences ice flow and loss from the Antarctic Ice Sheet (AIS)~\cite{DeepGlacialTroughs}. Bed topography datasets are typically generated from airborne radar surveys, which are sparse and unevenly distributed across the Antarctic continent,. Interpolation schemes to "gap fill" these sparse datasets yield bed topography estimates that have high uncertainties (i.e. multiple hundreds of metres uncertainty; Morlighem et al., 2020), which propagate in simulations of AIS evolution under climate change~\cite{ReductionOfUncertaintyThwaites}. Given the logistical challenges of accessing large parts of the Antarctic continent, there is a crucial need for alternative approaches that integrate diverse data streams – including satellite data – to derive bed topography.

\section*{Topographic Models of Antarctica: A Review}\label{review}
Numerical modelling and sedimentary sequence interpretation suggest cyclical periods of ice-sheet expansion and retreat\cite{Young_2011}. Using ice-penetrating radar data to generate a new basal bed topography of the Aurora Subglacial Basin (ASB) in east Antarctica is characterised by a fjord landscape (this land is under $\sim$ under $2-4.5$ km of ice). The ASB has a potentially significant influence on the east Antarctic ice-sheet (EAIS), however there is high uncertainty in estimates of past and present global sea level changes due to the scarcity of bed data\cite{Young_2011}. This uncertainty also limits the accuracy of models used to predict future ice sheet growth or decay.\\
{\large Methods in\cite{Young_2011}}
\begin{enumerate}
    \item A ski-equipped airplane (DC-3T) carried a radar system (HiCARS), which can see through ice. HiCARS sends signals that bounce back to show the thickness of the ice and the shape of the land beneath it.
    \item The plane flew back and forth over a large area, covering distances of around 1,000 km. The flights took place over two different periods in 2008–2009 and 2009–2010.
    \item The radar data was cleaned up (processed) to improve accuracy, and they used a special radar system that helps reduce distortions (errors) in the measurements. \textbf{[HOW?]}
    \item Thickness of the ice was measured using the time it took for the radar signals to travel through the ice and back, assuming the radar signals move through the ice at a specific speed (169 meters per microsecond).
    \item The height of the land below the ice was calculated by looking at the radar-determined surface of the ice. \textbf{[WHAT?]}
    \item The radar data was combined with other existing datasets (BEDMAP) to improve the overall picture. They used a computer algorithm to fill in gaps where they didn’t have direct measurements. \textbf{[WHICH?]}
    \item Determining how rough or uneven the land under the ice was, by using a statistical measure called the ''root mean squared (rms) deviation."
\end{enumerate}
In short, Young et al. used advanced radar technology on an airplane to map the ice thickness and the landscape beneath it in a region of Antarctica, combining this data with previous maps for a better overall picture.
