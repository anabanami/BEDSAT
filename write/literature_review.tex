\chapter*{Antarctica's Landscape}\label{why}
\section*{Climate Impacts and Global Significance}

The polar regions are losing ice, and their oceans are changing rapidly\cite{O_C_in_changingClimate}. The consequences of this polar transition extend to the whole planet and it is crucial for us to understand them and plan for changes. To make matters more challenging, there is significant uncertainty in the timing and magnitude of Antarctica's ice loss, largely due to unknowns in ice sheet properties and associated flow processes~\cite{IPCC}.

These changes manifest across multiple interconnected systems. In both polar oceans, shifts in seasonal ice extent and thickness are altering marine ecosystems, from primary production to species distribution\cite{O_C_in_changingClimate}. Of particular concern is the accelerating ice loss from both Greenland and Antarctic ice sheets, which has become a major contributor to global sea level rise\cite{O_C_in_changingClimate}. The impact extends beyond direct ice loss: increasing $\mathrm{CO_2}$ uptake by polar oceans is creating corrosive conditions for calcifying organisms\cite{O_C_in_changingClimate}, while freshwater input from melting ice sheets threatens to disrupt global thermohaline circulation\cite{Jacobs_2004}.

Antarctica's climate response, however, differs markedly from the Arctic's uniform warming pattern. While West Antarctica has experienced warming in certain regions, East Antarctica has shown minimal overall temperature change in recent decades. These observations carry low confidence due to limited data availability and high variability\cite{O_C_in_changingClimate}. This asymmetric response is partly explained by the Southern Ocean's unique ability to absorb and mix heat into its depths\cite{L_T_C_C}.
 
%%%%% FELICITY HAS PAPERS ON THE EFFECT IN EAIS DENMAN'S GLACIER ETC

Several atmospheric circulation patterns govern Antarctic climate variability. The Southern Annular Mode (SAM) has maintained a predominantly positive phase during recent summer seasons, resulting in intensified westerly winds around Antarctica - a pattern unprecedented in at least six centuries. While ozone depletion primarily drove SAM variations from the late 1970s through the 1990s, tropical sea surface temperatures have emerged as a dominant influence since 2000\cite{O_C_in_changingClimate}. These tropical ocean temperatures also affect broader Antarctic temperature patterns and Southern Hemisphere mid-latitude circulation\cite{Jacobs_2004}. Additional atmospheric patterns, including the Pacific South American mode and Zonal-wave 3, further modulate Antarctic climate dynamics.

%%%% JESS MACHA EL NINO STUFF?

%%%%%%%%%%%%%%%%%%%%%%%%%%%%%%%%%%%%%%%%%%%%%%%%%%%%%%%%%%%%%%%%%%%%%%%%%%%%%%

\section*{The importance of bed topography}
Bed topography is one of the most crucial boundary conditions that influences ice flow and loss from the Antarctic Ice Sheet (AIS)~\cite{Morlighem_2020}. Bed topography datasets are typically generated from airborne radar surveys, which are sparse and unevenly distributed across the Antarctic continent. Interpolation schemes to "gap fill" these sparse datasets yield bed topography estimates that have high uncertainties (i.e. multiple hundreds of metres uncertainty; Morlighem et al., 2020), which propagate in simulations of AIS evolution under climate change~\cite{Castleman_2022}. Given the logistical challenges of accessing large parts of the Antarctic continent, there is a crucial need for alternative approaches that integrate diverse data streams – including satellite data – to derive bed topography.

\subsection*{Currently available data}\label{data}
\begin{enumerate}
\item \textbf{The Reference Elevation Model of Antarctica (REMA)} is a time-stamped Digital Surface Model (DSM) of Antarctica of high-resolution (2-meter). REMA version 4.1 DSM strips are a 13-year time series of elevation data derived from satellite imagery using photogrammetric methods\cite{strips_2022}, while REMA version 2 mosaic tiles are made by merging over 12 years of photogrammetric elevation models\cite{mosaics_2022}. 
\end{enumerate}


\section*{Topographic Models of Antarctica: A Review}\label{review}
Numerical modelling and sedimentary sequence interpretation suggest cyclical periods of ice-sheet expansion and retreat\cite{Young_2011}. Using ice-penetrating radar data to generate a new basal bed topography of the Aurora Subglacial Basin (ASB) in east Antarctica is characterised by a fjord landscape (this land is under $\sim$ under $2-4.5$ km of ice). The ASB has a potentially significant influence on the east Antarctic ice-sheet (EAIS), however there is high uncertainty in estimates of past and present global sea level changes due to the scarcity of bed data\cite{Young_2011}. This uncertainty also limits the accuracy of models used to predict future ice sheet growth or decay.\\
{\large Methods in\cite{Young_2011}}
\begin{enumerate}
    \item A ski-equipped airplane (DC-3T) carried a radar system (HiCARS), which can see through ice. HiCARS sends signals that bounce back to show the thickness of the ice and the shape of the land beneath it.
    \item The plane flew back and forth over a large area, covering distances of around 1,000 km. The flights took place over two different periods in 2008–2009 and 2009–2010.
    \item The radar data was cleaned up (processed) to improve accuracy, and they used a special radar system that helps reduce distortions (errors) in the measurements. \textbf{[HOW?]}
    \item Thickness of the ice was measured using the time it took for the radar signals to travel through the ice and back, assuming the radar signals move through the ice at a specific speed (169 meters per microsecond).
    \item The height of the land below the ice was calculated by looking at the radar-determined surface of the ice. \textbf{[WHAT?]}
    \item The radar data was combined with other existing datasets (BEDMAP) to improve the overall picture. They used a computer algorithm to fill in gaps where they didn’t have direct measurements. \textbf{[WHICH?]}
    \item Determining how rough or uneven the land under the ice was, by using a statistical measure called the ''root mean squared (rms) deviation."
\end{enumerate}
In short, Young et al. used advanced radar technology on an airplane to map the ice thickness and the landscape beneath it in a region of Antarctica, combining this data with previous maps for a better overall picture.
