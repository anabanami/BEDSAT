\chapter{Glossary of Key Terms}\label{glossary}
\begin{enumerate}
\item \textbf{Adaptive Mesh Refinement}: A technique used to refine the mesh in regions of high variability or complexity, enhancing model accuracy and efficiency.
\item \textbf{Asperity}: A protrusion or bump on a surface. Roughness The unevenness of a surface, characterized by the size and distribution of asperities.
\item \textbf{Basal Drag Coefficient}: A parameter representing the frictional resistance between the ice sheet base and the underlying bedrock.
\item \textbf{Basal Melt Rates}: The rate at which the underside of a glacier or ice sheet melts due to contact with warmer ocean water.
\item \textbf{Basal sliding}: Sliding occurring at the base of a glacier.
\item \textbf{Bed Topography}: The shape and elevation of the bedrock underlying a glacier or ice sheet.
\item \textbf{Bedrock Topography}: The shape and elevation of the solid rock surface beneath a glacier or ice sheet.
\item \textbf{Blatter-Pattyn Approximation (BP)}: A higher-order ice flow model that incorporates longitudinal stresses, making it suitable for simulating fast-flowing ice streams and regions with significant vertical shear.
\item \textbf{Calving}: The process by which icebergs break off from the edge of a glacier or ice sheet.
\item \textbf{Coefficient of sliding friction ($\mu$)}: The ratio of the shear stress to the normal stress during steady-state sliding.
\item \textbf{Coefficient of static friction ($\mu_s$)}: The ratio of the limiting static shear stress to the normal stress, indicating the resistance to sliding from rest.
\item \textbf{Data Assimilation}: The process of incorporating observational data into a model to improve its accuracy and predictive capabilities.
\item \textbf{Finite Element Method (FEM)}: A numerical method for solving partial differential equations by dividing the domain into smaller elements and approximating the solution within each element.
\item \textbf{Fourier Transform}: A mathematical tool used to decompose a signal, such as surface elevation data, into its constituent frequencies. This allows for analysis of specific spatial scales and features.
\item \textbf{Full-Stokes (FS)}: The most comprehensive and computationally expensive ice flow model, accounting for all stress components. Essential for accurately simulating ice flow near grounding lines.
\item \textbf{Global Mean Sea-Level Rise (SLR)}: The average increase in sea level across the globe due to various factors, including melting of glaciers and ice sheets, and thermal expansion of ocean water.
\item \textbf{Grounding Line}: The boundary where the ice sheet transitions from grounded ice to floating ice (ice shelf).
\item \textbf{Ice-Penetrating Radar}: A remote sensing technique used to map the bedrock topography beneath glaciers and ice sheets by transmitting radar waves through the ice.
\item \textbf{Ice Sheet System Model (ISSM)}: A finite element, thermomechanical ice flow model that incorporates SIA, SSA, BP, and FS formulations to simulate ice sheet behaviour at various complexities and spatial resolutions.
\item \textbf{InSAR}: Interferometric Synthetic Aperture Radar, a remote sensing technique used to measure ice surface velocity.
\item \textbf{Inversion}: A mathematical technique used to infer unknown parameters, such as bed topography, from observations of other variables, such as surface elevation and velocity.
\item \textbf{Limiting dynamic shear stress ($T_m$)}: The shear stress at which a glacier or ice block transitions from steady-state sliding to accelerated sliding.
\item \textbf{Kriging}: A geostatistical interpolation method that estimates unknown values at specific points by calculating weighted averages of known values from surrounding points, while accounting for spatial correlation and providing uncertainty estimates.
\item \textbf{Limiting static shear stress ($T_S$)}: The minimum shear stress required to initiate sliding from a resting position.
\item \textbf{Linear Perturbation Analysis}: A technique that examines the response of a system to small perturbations in its parameters, assuming a linear relationship between the perturbation and the response.
\item \textbf{Marine Ice Sheet Instability (MISI)}: A process where the grounding line of an ice sheet retreats into deeper water, leading to accelerated ice discharge and potentially unstoppable collapse.
\item \textbf{Momentum Balance}: The fundamental physical principle describing how forces control ice motion, expressed through the Navier-Stokes equations. In ice sheet modeling, it accounts for the balance between internal stresses, gravitational driving forces, and resistive forces (including drag at the bed and lateral margins).
\item \textbf{Normal stress ($N$)}: The force acting perpendicular to a surface, per unit area. In the context of glaciers, it is primarily the weight of the overlying ice.
\item \textbf{Null Space}: The set of all possible solutions to an inverse problem that do not contribute to the observed data. In the case of the work of \ref{Ockenden_2022}, features aligned with ice flow fall within the null space and cannot be resolved by the inversion.
\item \textbf{Pinning Point}: A topographic feature, such as a ridge or mountain, that can slow or temporarily halt the retreat of a glacier's grounding line.
\item \textbf{Regelation}: The process of melting under pressure and refreezing at lower pressure, potentially contributing to ice sliding.
\item \textbf{Retrograde Bedrock Slope}: A bedrock slope that deepens inland, making the ice sheet more susceptible to marine ice sheet instability.
\item \textbf{Rheology}: The study of how materials deform and flow under stress. In glaciology, it refers to the flow properties of ice.
\item \textbf{Shallow Ice Approximation (SIA)}: A simplified ice flow model that considers only vertical shear stresses and neglects horizontal stress gradients. Suitable for slow-moving ice in the interior of ice sheets.
\item \textbf{Shallow Shelf Approximation (SSA)}: A simplified ice flow model that neglects vertical shear stresses and assumes depth-independent horizontal velocity. Appropriate for modelling floating ice shelves and fast-flowing ice streams.
\item \textbf{Shallow-Ice-Stream Approximation}: A simplification of the ice flow equations that assumes the ice thickness is much smaller than the horizontal extent of the glacier, allowing for analytical solutions.
\item \textbf{Shear stress ($T$)}: The force acting parallel to a surface, per unit area. In the context of glaciers, it is the force driving glacier motion.
\item \textbf{Sliding}: The movement of a glacier over its bed by sliding rather than internal deformation.
\item \textbf{Slipperiness}: A measure of the ease with which ice can slide over its bed. It encompasses the influence of basal conditions like geology, hydrology, and sediment characteristics.
\item \textbf{Steady-state}: A condition where the glacier's flow and properties are constant over time, assuming a balance between ice accumulation and loss.
\item \textbf{Steady-state velocity ($V_b$)}: The constant velocity reached by a glacier or ice block when the driving shear stress is balanced by resisting forces.
\item \textbf{Stress Balance}: The equilibrium between the forces acting on an ice sheet, including gravity, basal friction, and internal ice stresses.
\item \textbf{Temperate ice}: Ice at or near its pressure-melting point.
\item \textbf{Transfer Functions}: Mathematical equations that describe the relationship between perturbations in bed properties and the resulting changes in surface variables.
\item \textbf{Volume Above Floatation (VAF)}: The volume of an ice sheet that is grounded on bedrock and contributes to sea-level rise if it melts or slides into the ocean.
\item \textbf{Wavelet Decomposition}: A mathematical technique that analyzes a signal by decomposing it into different frequency components at various spatial scales.
\end{enumerate}