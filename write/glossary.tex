\chapter{Glossary of Key Terms}\label{glossary}
\begin{enumerate}
\item Temperate ice: Ice at or near its pressure-melting point.
\item Sliding: The movement of a glacier over its bed by sliding rather than internal deformation.
\item Basal sliding: Sliding occurring at the base of a glacier.
\item Normal stress (N):The force acting perpendicular to a surface, per unit area. In the context of glaciers, it is primarily the weight of the overlying ice.
\item Shear stress (T): The force acting parallel to a surface, per unit area. In the context of glaciers, it is the force driving glacier motion.
\item Limiting static shear stress (TS): The minimum shear stress required to initiate sliding from a resting position.
\item Coefficient of static friction (µs): The ratio of the limiting static shear stress to the normal stress, indicating the resistance to sliding from rest.
\item Steady-state velocity (Vb): The constant velocity reached by a glacier or ice block when the driving shear stress is balanced by resisting forces.
\item Limiting dynamic shear stress (Tm): The shear stress at which a glacier or ice block transitions from steady-state sliding to accelerated sliding.
\item Coefficient of sliding friction (µ): The ratio of the shear stress to the normal stress during steady-state sliding.
\item Regelation: The process of melting under pressure and refreezing at lower pressure, potentially contributing to ice sliding.
\item Asperity: A protrusion or bump on a surface.RoughnessThe unevenness of a surface, characterized by the size and distribution of asperities.
\item
\end{enumerate}

    WHAT ARE THESE
    \begin{itemize}
    \item{Channel incision}
    \item{Alpine style glaciation}
    \end{itemize}
    \vspace{1cm}
    TOOLS
    {\small \begin{itemize}
    \item{ICECAP aero geophysical programme}
    \item{BEDMAP}
    \end{itemize}
    }
    \vspace{1cm}
    MATHS
    {\small \begin{itemize}
    \item{Lagrangian interpolation}
    \item{natural-neighbour interpolation}
    \end{itemize}
    }