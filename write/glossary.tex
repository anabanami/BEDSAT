\chapter{Glossary of Key Terms}\label{glossary}
% https://www.antarcticglaciers.org/glaciers-and-climate/numerical-ice-sheet-models/numerical-modelling-glossary/
\begin{enumerate}
\item Temperate ice: Ice at or near its pressure-melting point.
\item Sliding: The movement of a glacier over its bed by sliding rather than internal deformation.
\item Basal sliding: Sliding occurring at the base of a glacier.
\item Normal stress (N):The force acting perpendicular to a surface, per unit area. In the context of glaciers, it is primarily the weight of the overlying ice.
\item Shear stress (T): The force acting parallel to a surface, per unit area. In the context of glaciers, it is the force driving glacier motion.
\item Limiting static shear stress (TS): The minimum shear stress required to initiate sliding from a resting position.
\item Coefficient of static friction (µs): The ratio of the limiting static shear stress to the normal stress, indicating the resistance to sliding from rest.
\item Steady-state velocity (Vb): The constant velocity reached by a glacier or ice block when the driving shear stress is balanced by resisting forces.
\item Limiting dynamic shear stress (Tm): The shear stress at which a glacier or ice block transitions from steady-state sliding to accelerated sliding.
\item Coefficient of sliding friction (µ): The ratio of the shear stress to the normal stress during steady-state sliding.
\item Regelation: The process of melting under pressure and refreezing at lower pressure, potentially contributing to ice sliding.
\item Asperity: A protrusion or bump on a surface.RoughnessThe unevenness of a surface, characterized by the size and distribution of asperities.
\item Shallow Ice Approximation (SIA): A simplified ice flow model that considers only vertical shear stresses and neglects horizontal stress gradients. Suitable for slow-moving ice in the interior of ice sheets.
\item Shallow Shelf Approximation (SSA): A simplified ice flow model that neglects vertical shear stresses and assumes depth-independent horizontal velocity. Appropriate for modelling floating ice shelves and fast-flowing ice streams
\item Blatter-Pattyn Approximation (BP): A higher-order ice flow model that incorporates longitudinal stresses, making it suitable for simulating fast-flowing ice streams and regions with significant vertical shear.
\item Full-Stokes (FS): The most comprehensive and computationally expensive ice flow model, accounting for all stress components. Essential for accurately simulating ice flow near grounding lines.
\item Ice Sheet System Model (ISSM): A finite element, thermomechanical ice flow model that incorporates SIA, SSA, BP, and FS formulations to simulate ice sheet behaviour at various complexities and spatial resolutions. 
\item Finite Element Method (FEM): A numerical method for solving partial differential equations by dividing the domain into smaller elements and approximating the solution within each element.
\item Adaptive Mesh Refinement: A technique used to refine the mesh in regions of high variability or complexity, enhancing model accuracy and efficiency.
\item Data Assimilation: The process of incorporating observational data into a model to improve its accuracy and predictive capabilities.
\item Basal Drag Coefficient: A parameter representing the frictional resistance between the ice sheet base and the underlying bedrock.
\item Grounding Line: The boundary where the ice sheet transitions from grounded ice to floating ice (ice shelf).
\item Calving: The process by which icebergs break off from the edge of a glacier or ice sheet
\item InSAR: Interferometric Synthetic Aperture Radar, a remote sensing technique used to measure ice surface velocity.
\end{enumerate}

    WHAT ARE THESE
    \begin{itemize}
    \item{Channel incision}
    \item{Alpine style glaciation}
    \end{itemize}
    \vspace{1cm}
    TOOLS
    {\small \begin{itemize}
    \item{ICECAP aero geophysical programme}
    \item{BEDMAP}
    \end{itemize}
    }
    \vspace{1cm}
    MATHS
    {\small \begin{itemize}
    \item{Lagrangian interpolation}
    \item{natural-neighbour interpolation}
    \end{itemize}
    }



